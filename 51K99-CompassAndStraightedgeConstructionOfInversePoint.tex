\documentclass[12pt]{article}
\usepackage{pmmeta}
\pmcanonicalname{CompassAndStraightedgeConstructionOfInversePoint}
\pmcreated{2013-03-22 17:13:17}
\pmmodified{2013-03-22 17:13:17}
\pmowner{Wkbj79}{1863}
\pmmodifier{Wkbj79}{1863}
\pmtitle{compass and straightedge construction of inverse point}
\pmrecord{12}{39547}
\pmprivacy{1}
\pmauthor{Wkbj79}{1863}
\pmtype{Algorithm}
\pmcomment{trigger rebuild}
\pmclassification{msc}{51K99}
\pmclassification{msc}{53A30}
\pmclassification{msc}{51M15}

\endmetadata

\usepackage{amssymb}
\usepackage{amsmath}
\usepackage{amsfonts}
\usepackage{pstricks}
\usepackage{psfrag}
\usepackage{graphicx}
\usepackage{amsthm}
%%\usepackage{xypic}

\begin{document}
\PMlinkescapeword{order}

Let $c$ be a circle in the Euclidean plane with center $O$ and let $P \neq O$.  One can construct the inverse point $P'$ of $P$ using compass and straightedge.

If $P \in c$, then $P=P'$.  Thus, it will be assumed that $P \notin c$.

The construction of $P'$ depends on whether $P$ is in the interior of $c$ or not.  The case that $P$ is in the interior of $c$ will be dealt with first.

\begin{enumerate}
\item Draw the ray $\overrightarrow{OP}$.

\begin{center}
\begin{pspicture}(-3,-3)(6,3)
\rput[a](0,3){.}
\rput[l](-3,0){.}
\rput[b](0,-3){.}
\rput[r](6,0){.}
\pscircle(0,0){3}
\psline[linecolor=blue]{*->}(0,0)(6,0)
\psdots(0,0)(2,0)
\rput[a](0,-0.3){$O$}
\rput[a](2,-0.3){$P$}
\end{pspicture}
\end{center}

\item Determine $Q \in \overrightarrow{OP}$ such that $Q \neq O$ and $\overline{OP} \cong \overline{PQ}$.

\begin{center}
\begin{pspicture}(-3,-3)(6,3)
\rput[a](0,3){.}
\rput[l](-3,0){.}
\rput[b](0,-3){.}
\rput[r](6,0){.}
\pscircle(0,0){3}
\psline{*->}(0,0)(6,0)
\psline[linecolor=blue](2,0)(4,0)
\psdots(0,0)(2,0)(4,0)
\rput[a](0,-0.3){$O$}
\rput[a](2,-0.3){$P$}
\rput[a](4,-0.3){$Q$}
\end{pspicture}
\end{center}

\item Construct the perpendicular bisector of $\overline{OQ}$ in order to find one point $T$ where it intersects $c$.

\begin{center}
\begin{pspicture}(-3,-3)(6,3)
\rput[a](0,3){.}
\rput[l](-3,0){.}
\rput[b](0,-3){.}
\rput[r](6,0){.}
\pscircle(0,0){3}
\psline{*->}(0,0)(6,0)
\psarc[linecolor=blue](0,0){2.5}{-40}{40}
\psarc[linecolor=blue](4,0){2.5}{140}{220}
\psline[linecolor=blue](2,-2)(2,2.236)
\psdots(0,0)(2,0)(4,0)(2,2.236)
\rput[a](0,-0.3){$O$}
\rput[a](2,-0.3){$P$}
\rput[a](4,-0.3){$Q$}
\rput[b](2.1,2.4){$T$}
\end{pspicture}
\end{center}

\item Draw the ray $\overrightarrow{OT}$.

\begin{center}
\begin{pspicture}(-3,-3)(6,5)
\rput[a](4.4723,5){.}
\rput[l](-3,0){.}
\rput[b](0,-3){.}
\rput[r](6,0){.}
\pscircle(0,0){3}
\psline{*->}(0,0)(6,0)
\psarc(0,0){2.5}{-40}{40}
\psarc(4,0){2.5}{140}{220}
\psline(2,-2)(2,2.236)
\psline[linecolor=blue]{*->}(0,0)(4.4723,5)
\psdots(0,0)(2,0)(4,0)(2,2.236)
\rput[a](0,-0.3){$O$}
\rput[a](2,-0.3){$P$}
\rput[a](4,-0.3){$Q$}
\rput[b](2,2.4){$T$}
\end{pspicture}
\end{center}

\item Determine $U \in \overrightarrow{OP}$ such that $U \neq O$ and $\overline{OT} \cong \overline{TU}$.

\begin{center}
\begin{pspicture}(-3,-3)(6,5)
\rput[a](4.4723,5){.}
\rput[l](-3,0){.}
\rput[b](0,-3){.}
\rput[r](6,0){.}
\pscircle(0,0){3}
\psline{*->}(0,0)(6,0)
\psarc(0,0){2.5}{-40}{40}
\psarc(4,0){2.5}{140}{220}
\psline(2,-2)(2,2.236)
\psline{*->}(0,0)(4.4723,5)
\psline[linecolor=blue](2,2.236)(4,4.472)
\psdots(0,0)(2,0)(4,0)(2,2.236)(4,4.472)
\rput[a](0,-0.3){$O$}
\rput[a](2,-0.3){$P$}
\rput[a](4,-0.3){$Q$}
\rput[b](2,2.4){$T$}
\rput[a](4,4.2){$U$}
\end{pspicture}
\end{center}

\item Construct the perpendicular bisector of $\overline{OU}$ in order to find the point where it intersects $\overrightarrow{OP}$.  This is $P'$.

\begin{center}
\begin{pspicture}(-3,-3)(6,5)
\rput[a](4.4723,5){.}
\rput[l](-3,0){.}
\rput[b](0,-3){.}
\rput[r](6,0){.}
\pscircle(0,0){3}
\psline{*->}(0,0)(6,0)
\psarc(0,0){2.5}{-40}{40}
\psarc(4,0){2.5}{140}{220}
\psline(2,-2)(2,2.236)
\psline{*->}(0,0)(4.4723,5)
\psline(2,2.236)(4,4.472)
\psarc[linecolor=blue](0,0){3.5}{10}{90}
\psarc[linecolor=blue](4,4.472){3.5}{190}{270}
\psline[linecolor=blue]{*->}(2,2.236)(6,-1.3416)
\psdots(0,0)(2,0)(4,0)(2,2.236)(4,4.472)(4.5,0)
\rput[a](0,-0.3){$O$}
\rput[a](2,-0.3){$P$}
\rput[a](4,-0.3){$Q$}
\rput[b](2,2.4){$T$}
\rput[a](4,4.2){$U$}
\rput[a](4.5,-0.3){$P'$}
\end{pspicture}
\end{center}

\end{enumerate}

Now the case in which $P$ is not in the interior of $c$ will be dealt with.

\begin{enumerate}

\item Connect $O$ and $P$ with a line segment.

\begin{center}
\begin{pspicture}(-2,-2)(4,2)
\rput[b](0,-2){.}
\rput[l](-2,0){.}
\rput[a](0,2){.}
\pscircle(0,0){2}
\psline[linecolor=blue](0,0)(4,0)
\psdots(0,0)(4,0)
\rput[a](0,-0.3){$O$}
\rput[a](4,-0.3){$P$}
\end{pspicture}
\end{center}

\item Construct the perpendicular bisector of $\overline{OP}$ in order to determine the midpoint $M$ of $\overline{OP}$.

\begin{center}
\begin{pspicture}(-2,-2)(4,2)
\rput[b](0,-2){.}
\rput[l](-2,0){.}
\rput[a](0,2){.}
\pscircle(0,0){2}
\psline(0,0)(4,0)
\psarc[linecolor=blue](0,0){2.5}{-40}{40}
\psarc[linecolor=blue](4,0){2.5}{140}{220}
\psline[linecolor=blue]{<->}(2,-2)(2,2)
\psdots(0,0)(4,0)(2,0)
\rput[a](0,-0.3){$O$}
\rput[a](4,-0.3){$P$}
\rput[a](2.3,-0.3){$M$}
\end{pspicture}
\end{center}

\item Draw an arc of the circle with center $M$ and radius $\overline{OM}$ in order to find one point $T$ where it intersects $C$.  By Thales' theorem, the angle $\angle OTP$ is a right angle; however, it does not need to be drawn.

\begin{center}
\begin{pspicture}(-2,-2)(4,2)
\rput[b](0,-2){.}
\rput[l](-2,0){.}
\rput[a](0,2){.}
\pscircle(0,0){2}
\psline(0,0)(4,0)
\psarc(0,0){2.5}{-40}{40}
\psarc(4,0){2.5}{140}{220}
\psline{<->}(2,-2)(2,2)
\psarc[linecolor=blue](2,0){2}{110}{130}
\psdots(0,0)(4,0)(2,0)(1,1.732)
\rput[a](0,-0.3){$O$}
\rput[a](4,-0.3){$P$}
\rput[a](2.3,-0.3){$M$}
\rput[l](1.12,1.83){$T$}
\end{pspicture}
\end{center}

\item Drop the perpendicular from $T$ to $\overline{OP}$.  The point of intersection is $P'$.

\begin{center}
\begin{pspicture}(-2,-2)(4,2)
\rput[b](0,-2){.}
\rput[l](-2,0){.}
\rput[a](0,2){.}
\pscircle(0,0){2}
\psline(0,0)(4,0)
\psarc(0,0){2.5}{-40}{40}
\psarc(4,0){2.5}{140}{220}
\psline{<->}(2,-2)(2,2)
\psarc(2,0){2}{110}{130}
\psarc[linecolor=blue](1,1.732){1.9}{240}{300}
\psarc[linecolor=blue](0.35,0){0.8}{-45}{45}
\psarc[linecolor=blue](1.65,0){0.8}{135}{225}
\psline[linecolor=blue]{<->}(1,2)(1,-1.5)
\psdots(0,0)(1,0)(1,1.732)(2,0)(4,0)
\rput[a](0,-0.3){$O$}
\rput[a](4,-0.3){$P$}
\rput[a](2.3,-0.3){$M$}
\rput[l](1.12,1.83){$T$}
\rput[a](0.6,0.2){$P'$}
\end{pspicture}
\end{center}

\end{enumerate}

A justification for these constructions is supplied in the entry inversion of plane.

If you are interested in seeing the rules for compass and straightedge constructions, click on the \PMlinkescapetext{link} provided.
%%%%%
%%%%%
\end{document}
