\documentclass[12pt]{article}
\usepackage{pmmeta}
\pmcanonicalname{CongruenceAxioms}
\pmcreated{2013-03-22 15:31:59}
\pmmodified{2013-03-22 15:31:59}
\pmowner{CWoo}{3771}
\pmmodifier{CWoo}{3771}
\pmtitle{congruence axioms}
\pmrecord{16}{37426}
\pmprivacy{1}
\pmauthor{CWoo}{3771}
\pmtype{Axiom}
\pmcomment{trigger rebuild}
\pmclassification{msc}{51F20}
\pmsynonym{axioms of congruence}{CongruenceAxioms}
\pmdefines{congruence relation}

\usepackage{amssymb,amscd}
\usepackage{amsmath}
\usepackage{amsfonts}

% used for TeXing text within eps files
%\usepackage{psfrag}
% need this for including graphics (\includegraphics)
%\usepackage{graphicx}
% for neatly defining theorems and propositions
%\usepackage{amsthm}
% making logically defined graphics
%%%\usepackage{xypic}

% define commands here

\renewcommand{\line}[1]{\overleftrightarrow{#1}}
\newcommand{\ray}[1]{\overrightarrow{#1}}
\begin{document}
\PMlinkescapeword{congruence}

\textbf{General Congruence Relations}.  Let $A$ be a set and
$X=A\times A$. A relation on $X$ is said to be a \emph{congruence
relation} on $X$, denoted $\cong$, if the following three conditions
are satisfied:
\begin{enumerate}
\item $(a,b)\cong (b,a)$, for all $a,b\in A$,
\item if $(a,a)\cong (b,c)$, then $b=c$, where $a,b,c\in A$,
\item if $(a,b)\cong (c,d)$ and $(a,b)\cong (e,f)$, then $(c,d)\cong
(e,f)$, for any $a,b,c,d,e,f\in A$.
\end{enumerate}
By applying $(b,a)\cong (a,b)$ twice, we see that $\cong$ is reflexive according to the third condition.  From this, it is easy to that $\cong$ is symmetric, since $(a,b)\cong (c,d)$ and $(a,b)\cong (a,b)$ imply $(c,d)\cong (a,b)$.  Finally, $\cong$ is transitive, for if $(a,b)\cong (c,d)$ and $(c,d)\cong
(e,f)$, then $(c,d)\cong (a,b)$ because $\cong$ is symmetric and so
$(a,b)\cong (e,f)$ by the third condition.  Therefore, the
congruence relation is an equivalence relation on pairs of elements
of $A$.
\\\\
\textbf{Congruence Axioms in Ordered Geometry}. Let $(A,B)$ be an
 ordered geometry with strict betweenness relation $B$.
We say that the ordered geometry $(A,B)$ satisfies the \emph{congruence
axioms} if
\begin{enumerate}
\item there is a congruence relation $\cong$ on $A\times A$;
\item if $(a,b,c)\in B$ and $(d,e,f)\in B$ with
\begin{itemize}
\item $(a,b)\cong (d,e)$, and
\item $(b,c)\cong (e,f),$
\end{itemize}
then $(a,c)\cong (d,f)$;
\item given $(a,b)$ and a ray $\rho$ emanating from $p$,
there exists a unique point $q$ lying on $\rho$ such that
$(p,q)\cong (a,b)$;
\item given the following:
\begin{itemize}
\item three rays emanating from $p_1$ such that they intersect with a
line $\ell_1$ at $a_1,b_1,c_1$ with $(a_1,b_1,c_1)\in B$, and
\item three rays emanating from $p_2$ such that they intersect with a
line $\ell_2$ at $a_2,b_2,c_2$ with $(a_2,b_2,c_2)\in B$,
\item $(a_1,b_1)\cong (a_2,b_2)$ and $(b_1,c_1)\cong (b_2,c_2)$,
\item $(p_1,a_1)\cong (p_2,a_2)$ and $(p_1,b_1)\cong (p_2,b_2)$,
\end{itemize}
then $(p_1,c_1)\cong (p_2,c_2)$;
\item given three distinct points $a,b,c$ and two distinct points $p,q$ such that $(a,b)\cong (p,q)$.  Let $H$ be a closed half plane with boundary
$\line{pq}$.  Then there exists a unique point $r$ lying on $H$ such
that $(a,c)\cong (p,r)$ and $(b,c)\cong (q,r)$.
\end{enumerate}
\textbf{Congruence Relations on line segments, triangles, and
angles}. With the above five congruence axioms, one may define a
congruence relation (also denoted by $\cong$ by abuse of notation)
on the set $S$ of closed line segments of $A$ by
$$\overline{ab}\cong\overline{cd}\qquad \mbox{ iff }\qquad (a,b)\cong (c,d),$$
where $\overline{ab}$ (in this entry) denotes the closed line
segment with endpoints $a$ and $b$.

It is obvious that the congruence relation defined on line segments
of $A$ is an equivalence relation.  Next, one defines a congruence
relation on triangles in $A$: $\triangle abc\cong \triangle pqr$ if
their sides are congruent:
\begin{enumerate}
\item $\overline{ab}\cong\overline{pq}$,
\item $\overline{bc}\cong\overline{qr}$, and
\item $\overline{ca}\cong\overline{rp}$.
\end{enumerate}
With this definition, Axiom 5 above can be restated as: given a
triangle $\triangle abc$, such that $\overline{ab}$ is congruent to
a given line segment $\overline{pq}$.  Then there is exactly one
point $r$ on a chosen side of the line $\line{pq}$ such that
$\triangle abc\cong\triangle pqr$.  Not surprisingly, the congruence
relation on triangles is also an equivalence relation.
\\\\
The last major congruence relation in an ordered geometry to be
defined is on angles: $\angle abc$ is \emph{congruent to} $\angle
pqr$ if there are
\begin{enumerate}
\item a point $a_1$ on $\ray{ba}$,
\item a point $c_1$ on $\ray{bc}$,
\item a point $p_1$ on $\ray{qp}$, and
\item a point $r_1$ on $\ray{qr}$
\end{enumerate}
such that $\triangle a_1bc_1\cong \triangle p_1qr_1$.
\\\\
It is customary to also write $\angle abc\cong \angle pqr$ to mean
that $\angle abc$ is congruent to $\angle pqr$.  Clearly for any
points $x\in\ray{ba}$ and $y\in\ray{bc}$, we have $\angle xby\cong
\angle abc$, so that $\cong$ is reflexive.  $\cong$ is also
symmetric and transitive (as the properties are inherited from the
congruence relation on triangles).  Therefore, the congruence
relation on angles also defines an equivalence relation.

\begin{thebibliography}{6}
\bibitem{dh} D. Hilbert, {\it Foundations of Geometry}, Open Court Publishing Co. (1971)
\bibitem{bs} K. Borsuk and W. Szmielew, {\it Foundations of Geometry}, North-Holland Publishing Co. Amsterdam (1960)
\bibitem{mg} M. J. Greenberg, {\it Euclidean and Non-Euclidean Geometries, Development and History}, W. H. Freeman and Company, San Francisco (1974)
\end{thebibliography}
%%%%%
%%%%%
\end{document}
