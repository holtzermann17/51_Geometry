\documentclass[12pt]{article}
\usepackage{pmmeta}
\pmcanonicalname{ProofOfPappussTheorem}
\pmcreated{2013-03-22 13:47:40}
\pmmodified{2013-03-22 13:47:40}
\pmowner{mathcam}{2727}
\pmmodifier{mathcam}{2727}
\pmtitle{proof of Pappus's theorem}
\pmrecord{6}{34509}
\pmprivacy{1}
\pmauthor{mathcam}{2727}
\pmtype{Proof}
\pmcomment{trigger rebuild}
\pmclassification{msc}{51A05}

\usepackage{amssymb}
\usepackage{amsmath}
\usepackage{amsfonts}
\usepackage{graphicx}
\begin{document}
\PMlinkescapeword{vertices}
\PMlinkescapeword{opposite}
\PMlinkescapeword{intersection}
\PMlinkescapeword{order}
\PMlinkescapeword{contain}
Pappus's theorem says that if the six vertices of a hexagon lie
alternately on two lines, then
the three points of intersection of opposite sides are collinear.
In the figure, the given lines are $A_{11}A_{13}$ and $A_{31}A_{33}$,
but we have omitted the letter $A$.

\includegraphics{pappus}

The appearance of the diagram will depend on the order in which the
given points appear on the two lines; two possibilities are shown.

Pappus's theorem is true in the affine plane over any (commutative) field.
A tidy proof is available with the aid of homogeneous coordinates.

No three of the four points $A_{11}$, $A_{21}$, $A_{31}$, and $A_{13}$ are
collinear, and therefore we can choose homogeneous coordinates such that
$$A_{11}=(1,0,0)\qquad A_{21}=(0,1,0)$$
$$A_{31}=(0,0,1)\qquad A_{13}=(1,1,1)$$
That gives us equations for three of the lines in the figure:
$$A_{13}A_{11}:y=z\qquad A_{13}A_{21}:z=x\qquad A_{13}A_{31}:x=y\;.$$
These lines contain $A_{12}$, $A_{32}$, and $A_{22}$ respectively, so
$$A_{12}=(p,1,1)\qquad A_{32}=(1,q,1)\qquad A_{22}=(1,1,r)$$
for some scalars $p,q,r$. So, we get equations for six more lines:
\begin{equation} \label{I1}
A_{31}A_{32}:y=qx\qquad A_{11}A_{22}:z=ry\qquad A_{12}A_{21}:x=pz
\end{equation}
\begin{equation} \label{I2}
A_{31}A_{12}:x=py\qquad A_{11}A_{32}:y=qz\qquad A_{21}A_{22}:z=rx
\end{equation}
By hypothesis, the three lines (\ref{I1}) are concurrent, and therefore
$prq=1$. But that implies $pqr=1$, and therefore the three lines
(\ref{I2}) are concurrent, QED.
%%%%%
%%%%%
\end{document}
