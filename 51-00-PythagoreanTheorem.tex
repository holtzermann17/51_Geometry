\documentclass[12pt]{article}
\usepackage{pmmeta}
\pmcanonicalname{PythagoreanTheorem}
\pmcreated{2013-03-22 11:43:09}
\pmmodified{2013-03-22 11:43:09}
\pmowner{drini}{3}
\pmmodifier{drini}{3}
\pmtitle{Pythagorean theorem}
\pmrecord{28}{30098}
\pmprivacy{1}
\pmauthor{drini}{3}
\pmtype{Theorem}
\pmcomment{trigger rebuild}
\pmclassification{msc}{51-00}
\pmclassification{msc}{20G42}
\pmclassification{msc}{16S40}
\pmclassification{msc}{57T05}
\pmclassification{msc}{16W35}
\pmclassification{msc}{81R50}
\pmclassification{msc}{16W30}
\pmsynonym{Pythagoreas' theorem}{PythagoreanTheorem}
\pmsynonym{Pythagoras theorem}{PythagoreanTheorem}
%\pmkeywords{Pythagoras}
%\pmkeywords{triangle}
%\pmkeywords{right}
%\pmkeywords{hypotenuse}
%\pmkeywords{leg}
\pmrelated{Triangle}
\pmrelated{CosinesLaw}
\pmrelated{Hypotenuse}
\pmrelated{RightTriangle}
\pmrelated{PythagoreanTriple}
\pmrelated{ProofOfPythagoreasTheorem}
\pmrelated{PythagorasTheorem}
\pmrelated{PtolemysTheorem}
\pmrelated{FirstPrimitivePythagoreanTriplets}
\pmrelated{PythagoreanTheoremInInnerProductSpaces}
\pmrelated{GeneralizedPythagoreanTheorem}

\usepackage{amssymb}
\usepackage{amsmath}
\usepackage{amsfonts}
\usepackage{graphicx}
%%%%%%%%%%%\usepackage{xypic}
\begin{document}
\textbf{Pythagorean theorem} states:

If $\triangle ABC$ is a right triangle, then the \PMlinkid{square}{1087} of the length of the hypothenuse
is equal to the sum of the squares of the two legs. In the following picture, the purple squares add up to the same area as the orange one.
\begin{center}
\includegraphics{pyth.eps}
\end{center}
$$AC^2=AB^2+BC^2.$$

Cosines law is a generalization of Pythagorean theorem for any triangle.
It implies that the converse of  Pythagorean theorem  also holds: if
the sides of a triangle satisfy $a^2+b^2=c^2$ then the triangle is a 
right triangle.
%%%%%
%%%%%
%%%%%
%%%%%
%%%%%
%%%%%
%%%%%
%%%%%
%%%%%
%%%%%
%%%%%
\end{document}
