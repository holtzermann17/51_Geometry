\documentclass[12pt]{article}
\usepackage{pmmeta}
\pmcanonicalname{ProofOfPtolemysTheorem}
\pmcreated{2013-03-22 12:38:31}
\pmmodified{2013-03-22 12:38:31}
\pmowner{drini}{3}
\pmmodifier{drini}{3}
\pmtitle{proof of Ptolemy's theorem}
\pmrecord{11}{32906}
\pmprivacy{1}
\pmauthor{drini}{3}
\pmtype{Proof}
\pmcomment{trigger rebuild}
\pmclassification{msc}{51-00}
\pmrelated{PtolemysTheorem}
\pmrelated{CyclicQuadrilateral}

\endmetadata

\usepackage{graphicx}
%%%\usepackage{xypic} 
\usepackage{bbm}
\newcommand{\Z}{\mathbbmss{Z}}
\newcommand{\C}{\mathbbmss{C}}
\newcommand{\R}{\mathbbmss{R}}
\newcommand{\Q}{\mathbbmss{Q}}
\newcommand{\mathbb}[1]{\mathbbmss{#1}}
\begin{document}
Let $ABCD$ be a cyclic quadrialteral. We will prove that 
$$AC\cdot BD=AB\cdot CD + BC\cdot DA.$$

\begin{center}
\includegraphics{ptolomeo}
\end{center}

Find a point $E$ on $BD$ such that $\angle BCA=\angle ECD$. Since $\angle BAC=\angle BDC$ for opening the same arc, we have triangle similarity
$\triangle ABC\sim \triangle DEC$ and so
$$\frac{AB}{DE}=\frac{CA}{CD}$$
which implies $AC\cdot ED = AB\cdot CD$.

Also notice that $\triangle ADC \sim \triangle BEC$ since have two pairs of equal angles. The similarity implies 
$$\frac{AC}{BC}=\frac{AD}{BE}$$
which implies $AC\cdot BE = BC\cdot DA$.

So we finally have $AC\cdot BD=AC(BE+ED)=AB\cdot CD+BC\cdot DA$.
%%%%%
%%%%%
\end{document}
