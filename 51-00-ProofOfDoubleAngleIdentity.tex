\documentclass[12pt]{article}
\usepackage{pmmeta}
\pmcanonicalname{ProofOfDoubleAngleIdentity}
\pmcreated{2013-03-22 12:50:30}
\pmmodified{2013-03-22 12:50:30}
\pmowner{drini}{3}
\pmmodifier{drini}{3}
\pmtitle{proof of double angle identity}
\pmrecord{4}{33168}
\pmprivacy{1}
\pmauthor{drini}{3}
\pmtype{Proof}
\pmcomment{trigger rebuild}
\pmclassification{msc}{51-00}

\usepackage{graphicx}
%%%\usepackage{xypic} 
\usepackage{bbm}
\newcommand{\Z}{\mathbbmss{Z}}
\newcommand{\C}{\mathbbmss{C}}
\newcommand{\R}{\mathbbmss{R}}
\newcommand{\Q}{\mathbbmss{Q}}
\newcommand{\mathbb}[1]{\mathbbmss{#1}}
\newcommand{\figura}[1]{\begin{center}\includegraphics{#1}\end{center}}
\newcommand{\figuraex}[2]{\begin{center}\includegraphics[#2]{#1}\end{center}}
\begin{document}
\textbf{Sine:}\\
\begin{eqnarray*}
\sin(2a)&=&\sin(a+a)\\
&=&\sin(a)\cos(a)+\cos(a)\sin(a)\\
&=&2\sin(a)\cos(a).
\end{eqnarray*}

\textbf{Cosine:}
\begin{eqnarray*}
\cos(2a)&=&\cos(a+a)\\
&=&\cos(a)\cos(a)+\sin(a)\sin(a)\\
&=&\cos^2(a)-\sin^2(a).
\end{eqnarray*}

By using the identity $$\sin^2(a)+\cos^2(a)=1$$ we can change the expression above into the alternate forms
$$\cos(2a)=2\cos^2(a)-1 = 1-2\sin^2(a).$$

\textbf{Tangent:}
\begin{eqnarray*}
\tan(2a)&=&\tan(a+a)\\
&=&\frac{\tan(a)+\tan(a)}{1-\tan(a)\tan(a)}\\
&=&\frac{2\tan(a)}{1-\tan^2(a)}.
\end{eqnarray*}
%%%%%
%%%%%
\end{document}
