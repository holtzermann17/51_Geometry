\documentclass[12pt]{article}
\usepackage{pmmeta}
\pmcanonicalname{TangentPlaneOfQuadraticSurface}
\pmcreated{2013-03-22 14:58:48}
\pmmodified{2013-03-22 14:58:48}
\pmowner{pahio}{2872}
\pmmodifier{pahio}{2872}
\pmtitle{tangent plane of quadratic surface}
\pmrecord{7}{36681}
\pmprivacy{1}
\pmauthor{pahio}{2872}
\pmtype{Result}
\pmcomment{trigger rebuild}
\pmclassification{msc}{51N20}
\pmrelated{TangentOfConicSection}
\pmrelated{QuadraticSurfaces}

% this is the default PlanetMath preamble.  as your knowledge
% of TeX increases, you will probably want to edit this, but
% it should be fine as is for beginners.

% almost certainly you want these
\usepackage{amssymb}
\usepackage{amsmath}
\usepackage{amsfonts}

% used for TeXing text within eps files
%\usepackage{psfrag}
% need this for including graphics (\includegraphics)
%\usepackage{graphicx}
% for neatly defining theorems and propositions
%\usepackage{amsthm}
% making logically defined graphics
%%%\usepackage{xypic}

% there are many more packages, add them here as you need them

% define commands here
\begin{document}
The common equation of all quadratic surfaces in the rectangular $(x,\,y,\,z)$-coordinate system is
\begin{align}
 Ax^2+By^2+Cz^2+2A'yz+2B'zx+2C'xy+2A''x+2B''y+2C''z+D = 0
\end{align}
where $A,\,B,\,C,\,A',\,B',\,C',\,A'',\,B'',\,C'',\,D$ are constants and at least one of the six first is distinct from zero.\, The equation of the tangent plane of the surface, with $(x_0,\,y_0,\,z_0)$ as the point of tangency, is
$$Ax_0x+By_0y+Cz_0z+A'(z_0y+y_0z)+B'(x_0z+z_0x)+C'(y_0x+x_0y)+
  A''(x+x_0)+B''(y+y_0)+C''(z+z_0)+D = 0.$$
This is said to be obtained from (1) by polarizing it.

\textbf{Example.} \,The tangent plane of the {\em elliptic paraboloid}\, $4x^2+9y^2 = 2z$\, set in the point\, $(x_0,\,y_0,\,z_0)$\, of the surface is\, $4x_0x+9y_0y = z+z_0$,\, and especially in the point\, $(\frac{1}{2},\,\frac{1}{3},\,1)$\, it is\, $2x+3y -z-1 = 0$.
%%%%%
%%%%%
\end{document}
