\documentclass[12pt]{article}
\usepackage{pmmeta}
\pmcanonicalname{CoordinateSystems}
\pmcreated{2013-03-22 15:12:53}
\pmmodified{2013-03-22 15:12:53}
\pmowner{matte}{1858}
\pmmodifier{matte}{1858}
\pmtitle{coordinate systems}
\pmrecord{16}{36977}
\pmprivacy{1}
\pmauthor{matte}{1858}
\pmtype{Definition}
\pmcomment{trigger rebuild}
\pmclassification{msc}{51-01}
\pmrelated{StereographicProjection}

% this is the default PlanetMath preamble.  as your knowledge
% of TeX increases, you will probably want to edit this, but
% it should be fine as is for beginners.

% almost certainly you want these
\usepackage{amssymb}
\usepackage{amsmath}
\usepackage{amsfonts}
\usepackage{amsthm}

\usepackage{mathrsfs}

% used for TeXing text within eps files
%\usepackage{psfrag}
% need this for including graphics (\includegraphics)
%\usepackage{graphicx}
% for neatly defining theorems and propositions
%
% making logically defined graphics
%%%\usepackage{xypic}

% there are many more packages, add them here as you need them

% define commands here

\newcommand{\sR}[0]{\mathbb{R}}
\newcommand{\sC}[0]{\mathbb{C}}
\newcommand{\sN}[0]{\mathbb{N}}
\newcommand{\sZ}[0]{\mathbb{Z}}

 \usepackage{bbm}
 \newcommand{\Z}{\mathbbmss{Z}}
 \newcommand{\C}{\mathbbmss{C}}
 \newcommand{\F}{\mathbbmss{F}}
 \newcommand{\R}{\mathbbmss{R}}
 \newcommand{\Q}{\mathbbmss{Q}}



\newcommand*{\norm}[1]{\lVert #1 \rVert}
\newcommand*{\abs}[1]{| #1 |}



\newtheorem{thm}{Theorem}
\newtheorem{defn}{Definition}
\newtheorem{prop}{Proposition}
\newtheorem{lemma}{Lemma}
\newtheorem{cor}{Corollary}
\begin{document}
\subsubsection*{Plane}
\begin{enumerate}
\item Cartesian coordinates
\item Polar coordinates
\end{enumerate}

\subsubsection*{$\R^3$}
\begin{enumerate}
\item \PMlinkname{Cartesian coordinates}{CartesianCoordinates}
\item Spherical coordinates
\item Cylindrical coordinates
\end{enumerate}

\subsubsection*{Sphere $S^2$}
\begin{enumerate}
\item Stereographic projection
\item Mercator projection
\item ..
\end{enumerate}


\subsubsection*{Miscellaneous}
\begin{enumerate}
\item Barycentric coordinates
\item Homogeneous coordinates $(x,\,y,\,t)$ in plane, $(x,\,y,\,z,\,t)$ in $\R^3$
\end{enumerate}

\subsubsection*{Differential geometry}
\begin{enumerate}
\item \PMlinkname{Normal coordinates}{RiemannNormalCoordinates} in Riemann geometry
\item \PMlinkname{Darboux coordinates}{DarbouxsTheoremSymplecticGeometry} in symplectic geometry
\item Gaussian coordinates
\item Fermi coordinates for a submanifold
\item Boundary coordinates
\end{enumerate}


\subsubsection*{Coordinate transformations}
\begin{enumerate}
\item \PMlinkname{affine transformations}{AffineTransformation} 
\item \PMlinkname{continuous transformations}{HomotopyOfPaths} in algebraic topology (homotopy theory)
\item Lorentz group transformations
\item Cartesian-to-polar coordinate transformations
\item Cartesian-to-spherical coordinate transformations
\item A.
\item B.
\end{enumerate}

%%%%%
%%%%%
\end{document}
