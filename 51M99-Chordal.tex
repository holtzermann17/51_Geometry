\documentclass[12pt]{article}
\usepackage{pmmeta}
\pmcanonicalname{Chordal}
\pmcreated{2013-03-22 15:07:50}
\pmmodified{2013-03-22 15:07:50}
\pmowner{PrimeFan}{13766}
\pmmodifier{PrimeFan}{13766}
\pmtitle{chordal}
\pmrecord{9}{36873}
\pmprivacy{1}
\pmauthor{PrimeFan}{13766}
\pmtype{Result}
\pmcomment{trigger rebuild}
\pmclassification{msc}{51M99}
\pmclassification{msc}{51N20}
\pmsynonym{radical axis}{Chordal}

\endmetadata

% this is the default PlanetMath preamble.  as your knowledge
% of TeX increases, you will probably want to edit this, but
% it should be fine as is for beginners.

% almost certainly you want these
\usepackage{amssymb}
\usepackage{amsmath}
\usepackage{amsfonts}

% used for TeXing text within eps files
%\usepackage{psfrag}
% need this for including graphics (\includegraphics)
%\usepackage{graphicx}
% for neatly defining theorems and propositions
%\usepackage{amsthm}
% making logically defined graphics
%%%\usepackage{xypic}

% there are many more packages, add them here as you need them

% define commands here
\begin{document}
By the \PMlinkescapetext{parent} entry, the power of the point $(a,\,b)$ with respect to the circle
            $$K_1(x,\,y) := (x-x_1)^2+(y-y_1)^2-r_1^2 =0$$
is equal to \,$K_1(a,\,b)$\, and with respect to the circle 
            $$K_2(x,\,y) := (x-x_2)^2+(y-y_2)^2-r_2^2 =0$$
equal to \,$K_2(a,\,b)$. \,Thus the locus of all points $(x,\,y)$ having the same \PMlinkescapetext{power} with respect to both circles is characterized by the equation
                 $$K_1(x,\,y) = K_2(x,\,y).$$
This reduces to the form
          $$2(x_2-x_2)x+2(y_2-y_1)y+k = 0,$$
and hence the locus is a straight line perpendicular to the \PMlinkescapetext{{\em centre line}} of the circles. \,This locus is called the {\em chordal} or the {\em radical axis} of the circles.
%%%%%
%%%%%
\end{document}
