\documentclass[12pt]{article}
\usepackage{pmmeta}
\pmcanonicalname{Slope}
\pmcreated{2013-03-22 14:48:10}
\pmmodified{2013-03-22 14:48:10}
\pmowner{pahio}{2872}
\pmmodifier{pahio}{2872}
\pmtitle{slope}
\pmrecord{14}{36457}
\pmprivacy{1}
\pmauthor{pahio}{2872}
\pmtype{Definition}
\pmcomment{trigger rebuild}
\pmclassification{msc}{51N20}
\pmsynonym{angle coefficient (?)}{Slope}
%\pmkeywords{parallelism of lines}
\pmrelated{Derivative}
\pmrelated{ExampleOfRotationMatrix}
\pmrelated{ParallellismInEuclideanPlane}
\pmrelated{SlopeAngle}
\pmrelated{LineInThePlane}
\pmrelated{DifferenceQuotient}
\pmrelated{DerivationOfWaveEquation}
\pmrelated{IsogonalTrajectory}
\pmrelated{TangentOfHyperbola}

\endmetadata

\usepackage{amssymb}
\usepackage{amsmath}
\usepackage{amsfonts}

\usepackage{graphicx}
\begin{document}
\PMlinkescapeword{represent}
The {\em slope} of a \PMlinkescapetext{straight} line in the $xy$-plane expresses how great is the change of the ordinate $y$ of the point of the line per a unit-change of the abscissa $x$ of the point; it requires that the line is not vertical.

The slope $m$ of the line may be determined by taking the changes of the coordinates between two arbitrary points $(x_1,\,y_1)$ and $(x_2,\,y_2)$ of the line:
              $$m = \frac{y_2-y_1}{x_2-x_1}$$

The equation of the line is
                    $$y = mx+b,$$
where $b$ indicates the intersection point of the line and the $y$-axis (one speaks of {\em y-intercept}).

The slope is equal to the \PMlinkname{tangent}{DefinitionsInTrigonometry} of the slope angle of the line.

Two non-vertical lines of the plane are parallel if and only if their slopes are equal.


\begin{center}
\includegraphics{slope}
\end{center}

In the previous picture, the blue line given by\, $3x-y+1 = 0$\, has slope $3$, whereas the red one given by\, 
$2x+y+2 = 0$\, has slope $-2$.\, Also notice that positive slopes represent ascending graphs and negative slopes represent descending graphs.
%%%%%
%%%%%
\end{document}
