\documentclass[12pt]{article}
\usepackage{pmmeta}
\pmcanonicalname{Rhomboid}
\pmcreated{2013-03-22 17:12:35}
\pmmodified{2013-03-22 17:12:35}
\pmowner{Wkbj79}{1863}
\pmmodifier{Wkbj79}{1863}
\pmtitle{rhomboid}
\pmrecord{9}{39532}
\pmprivacy{1}
\pmauthor{Wkbj79}{1863}
\pmtype{Definition}
\pmcomment{trigger rebuild}
\pmclassification{msc}{51-00}
\pmrelated{Parallelogram}

\endmetadata

\usepackage{amssymb}
\usepackage{amsmath}
\usepackage{amsfonts}
\usepackage{pstricks}
\usepackage{psfrag}
\usepackage{graphicx}
\usepackage{amsthm}
%%\usepackage{xypic}

\begin{document}
\PMlinkescapeword{word}

A \emph{rhomboid} is a parallelogram that is neither a rhombus nor a rectangle.  The word rhomboid comes from the Greek word $\rho o \mu \beta o \varepsilon \iota \delta \acute\eta \varsigma$, which is transliterated as `rhomboeidis'.  In \emph{The Elements}, Euclid has this definition:

\begin{quote}
``Of quadrilateral figures....a rhomboid (is) that which has its opposite sides and angles equal to each other but is neither \PMlinkname{equilateral}{Equilateral} nor \PMlinkname{right-angled}{RightAngle}.''
\end{quote}

Below is a picture of a rhomboid.

\begin{center}
\begin{pspicture}(0,0)(5,2)
\pspolygon(0,0)(1,2)(5,2)(4,0)
\end{pspicture}
\end{center}

The \PMlinkescapetext{formula} for the area of a rhomboid is the same as that for all parallelograms:  If $b$ is the length of its \PMlinkescapetext{base} and $h$ is the length of its \PMlinkescapetext{height}, then $A=bh$.
%%%%%
%%%%%
\end{document}
