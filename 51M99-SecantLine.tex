\documentclass[12pt]{article}
\usepackage{pmmeta}
\pmcanonicalname{SecantLine}
\pmcreated{2013-03-22 14:50:34}
\pmmodified{2013-03-22 14:50:34}
\pmowner{Mathprof}{13753}
\pmmodifier{Mathprof}{13753}
\pmtitle{secant line}
\pmrecord{15}{36512}
\pmprivacy{1}
\pmauthor{Mathprof}{13753}
\pmtype{Definition}
\pmcomment{trigger rebuild}
\pmclassification{msc}{51M99}
\pmsynonym{secant}{SecantLine}
\pmsynonym{secant of the curve}{SecantLine}
\pmsynonym{secant to the curve}{SecantLine}
\pmrelated{curve}
\pmdefines{cubic parabola}

% this is the default PlanetMath preamble.  as your knowledge
% of TeX increases, you will probably want to edit this, but
% it should be fine as is for beginners.

% almost certainly you want these
\usepackage{amssymb}
\usepackage{amsmath}
\usepackage{amsfonts}

% used for TeXing text within eps files
%\usepackage{psfrag}
% need this for including graphics (\includegraphics)
\usepackage{graphicx}
% for neatly defining theorems and propositions
%\usepackage{amsthm}
% making logically defined graphics
%%%\usepackage{xypic}

% there are many more packages, add them here as you need them

% define commands here
\begin{document}
The {\em secant line} (or simply the {\em secant}) of  a curve is a straight line intersecting the curve in at least two distinct points.\, [The name is initially a participial form of the Latin verb {\em secare} `\PMlinkescapetext{to cut}'.]

If one sets a secant e.g. to the ``cubic parabola''\, $y = x^3$\, through its points\, $(0,\,0)$\, and\, $(1,\,1)$,\, there is also a third common point\, 
$(-1,\,-1)$.

Notice that a secant line can also be tangent to the curve at some point, given that tangency is only a local property.\, In the following picture, $l$ is a secant line for the curve $C$ (since it intersects the curve at points $A$ and $B$), yet it is also a tangent line at the point $A$.
\begin{center}
\includegraphics{secant}
\end{center}
%%%%%
%%%%%
\end{document}
