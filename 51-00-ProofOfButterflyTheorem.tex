\documentclass[12pt]{article}
\usepackage{pmmeta}
\pmcanonicalname{ProofOfButterflyTheorem}
\pmcreated{2013-03-22 13:10:09}
\pmmodified{2013-03-22 13:10:09}
\pmowner{drini}{3}
\pmmodifier{drini}{3}
\pmtitle{proof of butterfly theorem}
\pmrecord{6}{33613}
\pmprivacy{1}
\pmauthor{drini}{3}
\pmtype{Proof}
\pmcomment{trigger rebuild}
\pmclassification{msc}{51-00}

% this is the default PlanetMath preamble.  as your knowledge
% of TeX increases, you will probably want to edit this, but
% it should be fine as is for beginners.

% almost certainly you want these
\usepackage{amssymb}
\usepackage{amsmath}
\usepackage{amsfonts}

% used for TeXing text within eps files
%\usepackage{psfrag}
% need this for including graphics (\includegraphics)
\usepackage{graphicx}
% for neatly defining theorems and propositions
%\usepackage{amsthm}
% making logically defined graphics
%%%\usepackage{xypic}

% there are many more packages, add them here as you need them

% define commands here
\begin{document}
\PMlinkescapeword{cut}
\PMlinkescapeword{onto}
\PMlinkescapeword{obviously}
Given that $M$ is the midpoint of a chord $PQ$ of a circle and $AB$ and $CD$ are two other chords passing through $M$, we will prove that $M$ is the midpoint of $XY,$ where $X$ and $Y$ are the points where $AD$ and $BC$ cut $PQ$ respectively.

\begin{center}
\includegraphics{but1.eps}
\end{center}

Let $O$ be the center of the circle. Since $OM$ is perpendicular to $XY$ (the line from the center of the circle to the midpoint of a chord is perpendicular to the chord), to show that $XM = MY,$ we have to prove that $\angle XOM = \angle YOM.$
Drop perpendiculars $OK$ and $ON$ from $O$ onto $AD$ and $BC$, respectively.
Obviously, $K$ is the midpoint of $AD$ and $N$ is the midpoint of $BC$. Further,
$$\angle DAB = \angle DCB$$ and $$\angle ADC = \angle ABC$$ as angles subtending equal arcs.
Hence triangles $ADM$ and $CBM$ are similar and hence
$$\frac{AD}{AM} = \frac{BC}{CM}$$   or   $$\frac{AK}{KM} = \frac{CN}{NM}$$
In other words, in triangles $AKM$ and $CNM, $ two pairs of sides are proportional. Also the angles between the corresponding sides are equal. We infer that the triangles $AKM$ and $CNM$ are similar.
Hence $\angle AKM = \angle CNM.$

Now we find that quadrilaterals $OKXM$ and $ONYM$ both have a pair of opposite straight angles.
This implies that they are both cyclic quadrilaterals.

In $OKXM,$ we have $\angle AKM = \angle XOM$ and in $ONYM,$ we have $\angle CNM = \angle YOM.$
From these two, we get $$\angle XOM = \angle YOM.$$
Therefore $M$ is the midpoint of $XY.$
%%%%%
%%%%%
\end{document}
