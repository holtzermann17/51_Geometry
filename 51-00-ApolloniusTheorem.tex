\documentclass[12pt]{article}
\usepackage{pmmeta}
\pmcanonicalname{ApolloniusTheorem}
\pmcreated{2013-03-22 11:44:10}
\pmmodified{2013-03-22 11:44:10}
\pmowner{yark}{2760}
\pmmodifier{yark}{2760}
\pmtitle{Apollonius theorem}
\pmrecord{14}{30146}
\pmprivacy{1}
\pmauthor{yark}{2760}
\pmtype{Theorem}
\pmcomment{trigger rebuild}
\pmclassification{msc}{51-00}
\pmclassification{msc}{18-00}
\pmclassification{msc}{81-00}
%\pmkeywords{Triangle}
%\pmkeywords{median}
\pmrelated{Triangle}
\pmrelated{Median}
\pmrelated{StewartsTheorem}
\pmrelated{ProofOfStewartsTheorem}
\pmrelated{ProofOfApolloniusTheorem2}
\pmrelated{ParallelogramLaw}
\pmrelated{ProofOfParallelogramLaw}

\endmetadata

\usepackage{amssymb}
\usepackage{amsmath}
\usepackage{amsfonts}
\usepackage{graphicx}
%%%%%%%\usepackage{xypic}
\begin{document}
\PMlinkescapeword{length}
\PMlinkescapeword{theorem}

Let $a,b,c$ the sides of a triangle and $m$ the length of the median to the side with length $a$.
Then $b^2+c^2=2m^2+\frac{a^2}{2}$.
\begin{center}
\includegraphics{apollonius}
\end{center}

If $b=c$ (the triangle is isosceles), then the theorem reduces to the
Pythagorean theorem,
$$
   m^2 + (a/2)^2 = b^2.
$$
%%%%%
%%%%%
%%%%%
%%%%%
%%%%%
%%%%%
%%%%%
\end{document}
