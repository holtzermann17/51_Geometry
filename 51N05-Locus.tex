\documentclass[12pt]{article}
\usepackage{pmmeta}
\pmcanonicalname{Locus}
\pmcreated{2013-03-22 17:07:52}
\pmmodified{2013-03-22 17:07:52}
\pmowner{pahio}{2872}
\pmmodifier{pahio}{2872}
\pmtitle{locus}
\pmrecord{7}{39436}
\pmprivacy{1}
\pmauthor{pahio}{2872}
\pmtype{Definition}
\pmcomment{trigger rebuild}
\pmclassification{msc}{51N05}
\pmrelated{SetMembership}
\pmdefines{locus}
\pmdefines{loci}

\endmetadata

% this is the default PlanetMath preamble.  as your knowledge
% of TeX increases, you will probably want to edit this, but
% it should be fine as is for beginners.

% almost certainly you want these
\usepackage{amssymb}
\usepackage{amsmath}
\usepackage{amsfonts}

% used for TeXing text within eps files
%\usepackage{psfrag}
% need this for including graphics (\includegraphics)
%\usepackage{graphicx}
% for neatly defining theorems and propositions
 \usepackage{amsthm}
% making logically defined graphics
%%%\usepackage{xypic}

% there are many more packages, add them here as you need them

% define commands here

\theoremstyle{definition}
\newtheorem*{thmplain}{Theorem}

\begin{document}
{\em Locus} (from Latin {\em locus} `\PMlinkescapetext{place, region}') is a set of points satisfying a certain condition and such that no point outside the set satisfies this condition.

For example, in the plane the locus of such point that is at a fixed distance $r$ from a fixed point $O$ is the circle with centre $O$ and radius $r$; the locus of such point which is equidistant from both ends of a line segment is the perpendicular bisector of this line segment.

In general, the loci in the plane are geometric curves; thus one may speak of {\em locus curves}.
%%%%%
%%%%%
\end{document}
