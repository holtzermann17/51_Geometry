\documentclass[12pt]{article}
\usepackage{pmmeta}
\pmcanonicalname{SNCFMetric}
\pmcreated{2013-03-22 15:17:25}
\pmmodified{2013-03-22 15:17:25}
\pmowner{GrafZahl}{9234}
\pmmodifier{GrafZahl}{9234}
\pmtitle{SNCF metric}
\pmrecord{5}{37083}
\pmprivacy{1}
\pmauthor{GrafZahl}{9234}
\pmtype{Example}
\pmcomment{trigger rebuild}
\pmclassification{msc}{51M05}
\pmclassification{msc}{97A20}
\pmclassification{msc}{54E35}
%\pmkeywords{French railway}
\pmrelated{RealTree}

% this is the default PlanetMath preamble.  as your knowledge
% of TeX increases, you will probably want to edit this, but
% it should be fine as is for beginners.

% almost certainly you want these
\usepackage{amssymb}
\usepackage{amsmath}
\usepackage{amsfonts}
\usepackage[latin1]{inputenc}

% used for TeXing text within eps files
%\usepackage{psfrag}
% need this for including graphics (\includegraphics)
%\usepackage{graphicx}
% for neatly defining theorems and propositions
\usepackage{amsthm}
% making logically defined graphics
%%%\usepackage{xypic}

% there are many more packages, add them here as you need them

% define commands here
\newcommand{\Bigcup}{\bigcup\limits}
\newcommand{\Prod}{\prod\limits}
\newcommand{\Sum}{\sum\limits}
\newcommand{\mbb}{\mathbb}
\newcommand{\mbf}{\mathbf}
\newcommand{\mc}{\mathcal}
\newcommand{\mmm}[9]{\left(\begin{array}{rrr}#1&#2&#3\\#4&#5&#6\\#7&#8&#9\end{array}\right)}
\newcommand{\ol}{\overline}

% Math Operators/functions
\DeclareMathOperator{\Frob}{Frob}
\DeclareMathOperator{\cwe}{cwe}
\DeclareMathOperator{\we}{we}
\DeclareMathOperator{\wt}{wt}
\begin{document}
The following two examples of a metric space (one of which is a real
tree) obtained their name from the \PMlinkescapetext{operator} of the French railway
system. Especially \emph{malicious} rumour has it that if you want to
go by train from $x$ to $y$ in France, the most efficient solution is
to reduce the problem to going from $x$ to Paris and then from Paris
to $y$.

Since their discovery, the intrinsic laws of the French way of going
by train have made it around the world and reached the late-afternoon
tutorials of first-term mathematics courses in an effort to lighten the
moods in the guise of the following definition:
\theoremstyle{definition}
\newtheorem{defn}{Definition}
\begin{defn}[SNCF metric]
Let $P$ be a point in a metric space $(F,d)$. Then the \emph{SNCF
metric} $d_P$ with respect to $P$ is defined by
\begin{equation*}
d_P(x,y):=\begin{cases}
0&\text{if }x=y\\
d(x,P)+d(P,y)&\text{otherwise.}
\end{cases}
\end{equation*}
\end{defn}
It is easy to see that $d_P$ is a metric.

Now, what if the train from $x$ to Paris stops over in $y$ during the
ride (or the other way round)? Sure, Paris is a beautiful city, but
you wouldn't \emph{always} want to go there and back again. To
implement this, the geometric notion of ``$y$ lies on the straight
line defined by $x$ and $P$'' is required, so the definition becomes
more specialised:
\begin{defn}[SNCF metric, enhanced version]
Let $P$ be the origin in the space $\mbb{R}^n$ with Euclidean norm
$\|\cdot\|_2$. Then the \emph{SNCF metric} $d_P$ is defined by
\begin{equation*}
d_P(x,y):=\begin{cases}
\|x-y\|_2&\text{if }x\text{ and }y\text{ lie on the same ray from the origin}\\
\|x\|_2+\|y\|_2&\text{otherwise}
\end{cases}.
\end{equation*}
\end{defn}
The metric space $(\mbb{R}^n,d_P)$ is, in addition, a real tree since
if $x$ and $y$ do not lie on the same \PMlinkid{ray}{6962} from $P$, the only arc in
$(\mbb{R}^n,d_P)$ joining $x$ and $y$ consists of the two ray \PMlinkid{segments}{5783}
$xP$ and $yP$. Other injections which are arcs in Euclidean
$\mbb{R}^n$ do not remain continuous in $(\mbb{R}^n,d_P)$.
%%%%%
%%%%%
\end{document}
