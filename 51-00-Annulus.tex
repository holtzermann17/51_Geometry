\documentclass[12pt]{article}
\usepackage{pmmeta}
\pmcanonicalname{Annulus}
\pmcreated{2013-03-22 12:26:23}
\pmmodified{2013-03-22 12:26:23}
\pmowner{akrowne}{2}
\pmmodifier{akrowne}{2}
\pmtitle{annulus}
\pmrecord{12}{32535}
\pmprivacy{1}
\pmauthor{akrowne}{2}
\pmtype{Definition}
\pmcomment{trigger rebuild}
\pmclassification{msc}{51-00}

\usepackage{amssymb}
\usepackage{amsmath}
\usepackage{amsfonts}

%\usepackage{psfrag}
\usepackage{graphicx}
%%%\usepackage{xypic}
\begin{document}
An \emph{annulus} is a two-dimensional shape which can be thought of as a disc with a smaller disc removed from its center.  An annulus looks like:

\begin{center}
\includegraphics[scale=.7]{annulus}
\end{center}

Note that both the inner and outer radii may take on any values, so long as the outer radius is larger than the inner.
%%%%%
%%%%%
\end{document}
