\documentclass[12pt]{article}
\usepackage{pmmeta}
\pmcanonicalname{CorrespondingAnglesInTransversalCutting}
\pmcreated{2013-03-22 17:15:12}
\pmmodified{2013-03-22 17:15:12}
\pmowner{Wkbj79}{1863}
\pmmodifier{Wkbj79}{1863}
\pmtitle{corresponding angles in transversal cutting}
\pmrecord{12}{39588}
\pmprivacy{1}
\pmauthor{Wkbj79}{1863}
\pmtype{Theorem}
\pmcomment{trigger rebuild}
\pmclassification{msc}{51M04}
\pmclassification{msc}{51-01}
%\pmkeywords{corresponding angles}
\pmrelated{EuclideanAxiomByHilbert}
\pmrelated{HarmonicMeanInTrapezoid}
\pmdefines{transversal}
\pmdefines{vertical angle}
\pmdefines{alternate interior angle}

\endmetadata

\usepackage{amssymb}
\usepackage{amsmath}
\usepackage{amsfonts}
\usepackage{pstricks}
\usepackage{amsthm}

\theoremstyle{definition}
\newtheorem{thm}{Theorem}
\newtheorem{cor}{Corollary}
\newtheorem*{rem*}{Remark}

\begin{document}
\PMlinkescapeword{cut}
\PMlinkescapeword{right}

\begin{center}
\begin{pspicture}(-3,-3)(3,3)
\rput[b](5,-2){.}
\psline(-3,-3)(3,3)
\psline(-1,-3)(5,2)
\psline(-2,2)(5,-2)
\rput[r](1.12,0.65){$\alpha$}
\rput[r](2.68,-0.3){$\beta$}
\rput[r](1.9,-0.35){$\beta_1$}
\rput[l](3.1,3){$\ell$}
\rput[l](5.2,2){$m$}
\rput[r](-2.2,2){$t$}
\end{pspicture}
\end{center}

The following theorem is valid in Euclidean geometry:

\begin{thm}
If two lines ($\ell$ and $m$) are cut by a third line, called a \emph{transversal} ($t$), and one pair of corresponding angles (\PMlinkname{e.g.}{Eg} $\alpha$ and $\beta$) are congruent, then the cut lines are parallel.
\end{thm}

Its converse theorem is also valid in Euclidean geometry:

\begin{thm}
If two parallel lines ($\ell$ and $m$) are cut by a transversal ($t$), then each pair of corresponding angles (e.g. $\alpha$ and $\beta$) are congruent.
\end{thm}

\begin{rem*}
The angle $\beta$ in both theorems may be replaced with its \emph{vertical angle} $\beta_1$.\, The angles $\alpha$ and $\beta_1$ are called \emph{alternate interior angles} of each other.
\end{rem*}

\begin{cor}
Two lines that are perpendicular to the same line are parallel to each other.
\end{cor}

\begin{cor}
If a line is perpendicular to one of two parallel lines, then it is also perpendicular to the other.
\end{cor}

\begin{cor}
If the left sides of two convex angles are parallel (or alternatively perpendicular) as well as their right sides, then the angles are congruent.
\end{cor}

\begin{thebibliography}{9}
\bibitem{VG}{\sc K. V\"ais\"al\"a:} {\em Geometria}. Kolmas painos. Werner S\"oderstr\"om Osakeyhti\"o, Porvoo ja Helsinki (1971).
\end{thebibliography}
%%%%%
%%%%%
\end{document}
