\documentclass[12pt]{article}
\usepackage{pmmeta}
\pmcanonicalname{piAndpi2AreIrrational}
\pmcreated{2013-03-22 14:44:00}
\pmmodified{2013-03-22 14:44:00}
\pmowner{mathcam}{2727}
\pmmodifier{mathcam}{2727}
\pmtitle{$\pi$ and $\pi^2$ are irrational}
\pmrecord{15}{36365}
\pmprivacy{1}
\pmauthor{mathcam}{2727}
\pmtype{Theorem}
\pmcomment{trigger rebuild}
\pmclassification{msc}{51-00}
\pmclassification{msc}{11-00}

\endmetadata

% this is the default PlanetMath preamble.  as your knowledge
% of TeX increases, you will probably want to edit this, but
% it should be fine as is for beginners.

% almost certainly you want these
\usepackage{amssymb}
\usepackage{amsmath}
\usepackage{amsfonts}

% used for TeXing text within eps files
%\usepackage{psfrag}
% need this for including graphics (\includegraphics)
%\usepackage{graphicx}
% for neatly defining theorems and propositions
\usepackage{amsthm}
% making logically defined graphics
%%%\usepackage{xypic}

% there are many more packages, add them here as you need them

% define commands here
\begin{document}
\newtheorem{thm}{Theorem}
\begin{thm}
$\pi$ and $\pi^2$ are irrational.
\end{thm}

\begin{proof}
For any strictly positive integer $n$ ,$x\in (0,1)$ we define:
$$f=f(x)=\frac{x^n(1-x)^n}{n!}=\frac{1}{n!}\sum_{m=n}^{2n}c_mx^m$$
where $c_m$ are integers. For $0<x<1$ we have

\begin{equation}
\label{firseq}
0<f(x)<\frac{1}{n!}
\end{equation}

For a contradiction, suppose $\pi^2$ is rational, so that $\pi^2=\frac{a}{b}$, where $a,b$ are positive integers.

For $x\in (0,1)$ let us define
$$G(x)=b^n[\pi^{2n}f(x)-\pi^{2n-2}f''(x)+\pi^{2n-4}f^{(4)}(x)-...+(-1)^nf^{(2n)}(x)].$$
We have that $f(0)=0$ and $f^{(m)}(0)=0$ if $m<n$ or $m>2n$. But, if $n \leq m \leq 2n$, then
$$f^{(m)}(0)=\frac{m!}{n!}c_m,$$
an integer. Hence $f(x)$ and all its derivates take integral values at $x=0$.Since $f(1-x)=f(x)$, the same is true at $x=1$

so that $G(0)$ and $G(1)$ are integers. We have

\begin{eqnarray*}
\frac{d}{dx}[G'(x)\sin{\pi x}-\pi G(x)\cos{\pi x}]
&=& [G''(x)+\pi^2G(x)]\sin{\pi x} \\
&=& b^n\pi^{2n+2}f(x)\sin{\pi x} \\
&=& \pi^2a^n \sin{\pi x}f(x).
\end{eqnarray*}

Hence
$$\pi\int_0^1a^n \sin{\pi x}f(x)dx=[\frac{G'(x)\sin{\pi x}}{\pi}-G(x)\cos{\pi x}]_0^1$$
$$=G(0)+G(1),$$
witch is an integer. But by equation \ref{firseq},
$$0<\pi\int_0^1a^n \sin{\pi x}f(x)dx<\frac{\pi a^n}{n!}<1.$$
For a large enough $n$, we obtain a contradiction.

For any integer $n$, if $a^n$ is irrational then a is irrational \PMlinkexternal{(proof)}{http://planetmath.org/?op=getobj&from=objects&id=5779}, 
and since $\pi^2$ is irrational $\sqrt{\pi^2}=\pi$ is also irrational.
\end{proof}

The irrationality of $\pi$ was Proved by Lambert in 1761. The above proof is not the original proof due to Lambert.
\begin{thebibliography}{99}
\bibitem a G.H.Hardy and E.M.Wright \emph{An Introduction to the Theory of
Numbers}, Oxford University Press, 1959
\end{thebibliography}

\subsection*{See also}
\begin{itemize}
\item The MacTutor History of Mathematics Archive,
\PMlinkexternal{A history of
Pi}{http://www-gap.dcs.st-and.ac.uk/~history/HistTopics/Pi_through_the_ages.html}
\item The MacTutor History of Mathematics Archive,
\PMlinkexternal{Johann Heinrich
Lambert}{http://www-history.mcs.st-andrews.ac.uk/Mathematicians/Lambert.html}
\item \PMlinkexternal{Irrationality proofs}{http://numbers.computation.free.fr/Constants/Miscellaneous/irrationality.html}
\end{itemize}
%%%%%
%%%%%
\end{document}
