\documentclass[12pt]{article}
\usepackage{pmmeta}
\pmcanonicalname{PowerOfPoint}
\pmcreated{2013-03-22 15:07:02}
\pmmodified{2013-03-22 15:07:02}
\pmowner{PrimeFan}{13766}
\pmmodifier{PrimeFan}{13766}
\pmtitle{power of point}
\pmrecord{15}{36854}
\pmprivacy{1}
\pmauthor{PrimeFan}{13766}
\pmtype{Theorem}
\pmcomment{trigger rebuild}
\pmclassification{msc}{51M99}
\pmsynonym{power of the point}{PowerOfPoint}
\pmsynonym{power of a point}{PowerOfPoint}
\pmrelated{InversionOfPlane}
\pmrelated{VolumeOfSphericalCapAndSphericalSector}

% this is the default PlanetMath preamble.  as your knowledge
% of TeX increases, you will probably want to edit this, but
% it should be fine as is for beginners.

% almost certainly you want these
\usepackage{amssymb}
\usepackage{amsmath}
\usepackage{amsfonts}
\usepackage{pstricks}
% used for TeXing text within eps files
%\usepackage{psfrag}
% need this for including graphics (\includegraphics)
%\usepackage{graphicx}
% for neatly defining theorems and propositions
 \usepackage{amsthm}
% making logically defined graphics
%%%\usepackage{xypic}

% there are many more packages, add them here as you need them

% define commands here

\theoremstyle{definition}
\newtheorem*{thmplain}{Theorem}

\begin{document}
\begin{thmplain}
 \,If a secant of the circle is drawn through a point ($P$), then the product of the line segments on the secant between the point and the perimeter of the circle is \PMlinkescapetext{independent} on the direction of the secant. \,The product is called {\em the power of the point with respect to the circle}.
\end{thmplain}

{\em Proof.} \,Let $PA$ and $PB$ be the segments of a secant and $PA'$ and $PB'$ the segments of another secant. \,Then the triangles $PAB'$ and $PA'B$ are similar since they have equal angles, namely the central angles $\angle APB'$ and $\angle BPA'$ and the inscribed angles $\angle PAB'$ and $\angle PA'B$. \,Thus we have the \PMlinkname{proportion}{ProportionEquation}
$$\frac{PA}{PA'} = \frac{PB'}{PB},$$
which implies the asserted equation
$$PA\cdot PB = PA'\cdot PB'.$$

\begin{center}
\begin{pspicture}(-3,-2)(3,3)
\psline{-}(-3,-2)(2.6,3)
\psline{-}(-3,-2)(2.8,-1.2)
\rput[b](-3,-2){.}
\rput[a](2.6,3){.}
\rput[r](2.8,-1.2){.}
\rput[b](0,-2.5){.}
\pscircle[linecolor=blue](0,0){2.5}
\psdots(0,0)(-3,-2)
\rput[a](-3.4,-2){$P$}
\rput[r](-1.6,-1.6){$A$}
\rput[r](-1.8,-0.9){$A'$}
\rput[a](2,-1.1){$B$}
\rput[a](1.9,2){$B'$}
\end{pspicture}
\end{center}

\textbf{Notes.} \,If the point $P$ is outside a circle, then value of the power of the point with respect to the circle is equal to the square of the limited tangent of the circle from $P$; this \PMlinkname{square}{SquareOfANumber} may be considered as the limit case of the power of point where the both intersection points of the secant with the circle coincide. \,Another \PMlinkescapetext{extension} of the notion power of point is got when the line through $P$ does not intersect the circle; we can think that then the intersecting points are imaginary; also now the product of the ``imaginary line segments'' is the same.

Denote by $p^2$ the power of the point \,$P := (a,\,b)$\, with respect to circle
$$K(x,\,y) := (x-x_0)^2+(y-y_0)^2-r^2 = 0.$$
Then, by the Pythagorean theorem, we obtain
\begin{align}
p^2 = (a-x_0)^2+(b-y_0)^2-r^2
\end{align}
if $P$ is outside the circle and
\begin{align}
p^2 = r^2-((a-x_0)^2+(b-y_0)^2)
\end{align}
if $P$ is inside of the circle. \,If in the latter case, we change the definition of the power of point to be the negative value $-p^2$ for a point inside the circle, then in both cases the power of the point $(a,\,b)$ is equal to
$$K(a,\,b) \equiv (a-x_0)^2+(b-y_0)^2-r^2.$$
%%%%%
%%%%%
\end{document}
