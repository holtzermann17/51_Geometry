\documentclass[12pt]{article}
\usepackage{pmmeta}
\pmcanonicalname{ProofOfCosinesLaw}
\pmcreated{2013-03-22 18:27:13}
\pmmodified{2013-03-22 18:27:13}
\pmowner{pahio}{2872}
\pmmodifier{pahio}{2872}
\pmtitle{proof of cosines law}
\pmrecord{4}{41116}
\pmprivacy{1}
\pmauthor{pahio}{2872}
\pmtype{Proof}
\pmcomment{trigger rebuild}
\pmclassification{msc}{51M04}
\pmrelated{DerivationOfCosinesLaw}

% this is the default PlanetMath preamble.  as your knowledge
% of TeX increases, you will probably want to edit this, but
% it should be fine as is for beginners.

% almost certainly you want these
\usepackage{amssymb}
\usepackage{amsmath}
\usepackage{amsfonts}

% used for TeXing text within eps files
%\usepackage{psfrag}
% need this for including graphics (\includegraphics)
%\usepackage{graphicx}
% for neatly defining theorems and propositions
 \usepackage{amsthm}
% making logically defined graphics
%%%\usepackage{xypic}

% there are many more packages, add them here as you need them

% define commands here

\theoremstyle{definition}
\newtheorem*{thmplain}{Theorem}

\begin{document}
Let $a$, $b$, $c$ be the sides of a triangle and $\alpha$, $\beta$, $\gamma$ its angles, respectively.\, By the projection formula, one may write the equalities
\begin{align*}
\begin{cases}
a = b\cos\gamma+c\cos\beta\\
b = c\cos\alpha+a\cos\gamma\\
c = a\cos\beta+b\cos\alpha.
\end{cases}
\end{align*}
Multiplying the equalities by $a$, $-b$ and $-c$, respectively, they read
\begin{align*}
\begin{cases}
a^2 \;=\; ab\cos\gamma+ca\cos\beta\\
-b^2 = -bc\cos\alpha-ab\cos\gamma\\
-c^2 = -ca\cos\beta -bc\cos\alpha.
\end{cases}
\end{align*}
Addition of these yields the sum equation
$$a^2-b^2-c^2 = -2bc\cos\alpha,$$
i.e. 
$$a^2 \;=\; b^2+c^2-2bc\cos\alpha,$$
which is the cosines law.

%%%%%
%%%%%
\end{document}
