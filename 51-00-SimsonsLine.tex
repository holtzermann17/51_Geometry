\documentclass[12pt]{article}
\usepackage{pmmeta}
\pmcanonicalname{SimsonsLine}
\pmcreated{2013-03-22 12:24:34}
\pmmodified{2013-03-22 12:24:34}
\pmowner{drini}{3}
\pmmodifier{drini}{3}
\pmtitle{Simson's line}
\pmrecord{17}{32277}
\pmprivacy{1}
\pmauthor{drini}{3}
\pmtype{Theorem}
\pmcomment{trigger rebuild}
\pmclassification{msc}{51-00}
\pmrelated{Circumcircle}
\pmrelated{Triangle}

\usepackage{graphicx}
%%%\usepackage{xypic} 
\usepackage{bbm}
\newcommand{\Z}{\mathbbmss{Z}}
\newcommand{\C}{\mathbbmss{C}}
\newcommand{\R}{\mathbbmss{R}}
\newcommand{\Q}{\mathbbmss{Q}}
\newcommand{\mathbb}[1]{\mathbbmss{#1}}
\begin{document}
Let $ABC$ a triangle and $P$ a point on its circumcircle (other than $A,B,C$).
Then the feet of the perpendiculars drawn from P to the sides $AB,BC,CA$ (or their prolongations) are collinear.

\begin{center}
\includegraphics{simson.eps}
\end{center}

In the picture, the line passing through $U,V,W$ is a Simson line for $\triangle ABC$.

An interesting result form the realm of analytic geometry states that the envelope formed by Simson's lines when P  varies is a circular hypocycloid of three points.

\begin{center}
\includegraphics{simson-env}
\end{center}
%%%%%
%%%%%
\end{document}
