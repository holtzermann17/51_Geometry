\documentclass[12pt]{article}
\usepackage{pmmeta}
\pmcanonicalname{Triangle}
\pmcreated{2013-03-22 11:43:51}
\pmmodified{2013-03-22 11:43:51}
\pmowner{Wkbj79}{1863}
\pmmodifier{Wkbj79}{1863}
\pmtitle{triangle}
\pmrecord{54}{30139}
\pmprivacy{1}
\pmauthor{Wkbj79}{1863}
\pmtype{Definition}
\pmcomment{trigger rebuild}
\pmclassification{msc}{51-00}
\pmclassification{msc}{51M05}
\pmclassification{msc}{00A05}
\pmclassification{msc}{51M10}
\pmclassification{msc}{55-00}
\pmclassification{msc}{55-01}
%\pmkeywords{Geometry}
%\pmkeywords{Polygon}
%\pmkeywords{Angle}
\pmrelated{SinesLaw}
\pmrelated{EulerLine}
\pmrelated{Median}
\pmrelated{PythagorasTheorem}
\pmrelated{Hypotenuse}
\pmrelated{Orthocenter}
\pmrelated{OrthicTriangle}
\pmrelated{IsoscelesTriangle}
\pmrelated{CevasTheorem}
\pmrelated{Cevian}
\pmrelated{SinesLawProof}
\pmrelated{FundamentalTheoremOnIsogonalLines}
\pmrelated{Incenter}
\pmrelated{EquilateralTriangle}
\pmrelated{TrigonometricVersionOfCevasTheorem}
\pmrelated{HeronsFo}
\pmdefines{acute triangle}
\pmdefines{right triangle}
\pmdefines{obtuse triangle}

\endmetadata

\usepackage{amssymb}
\usepackage{amsmath}
\usepackage{amsfonts}
\usepackage{graphicx}
%%%%%%%\usepackage{xypic}
\begin{document}
A \emph{triangle} is a \PMlinkescapetext{bounded} planar region delimited by three \PMlinkescapetext{straight} lines, i.e. it is a polygon with three angles.

\begin{center}
\includegraphics{triangulo}
\end{center}

In Euclidean geometry, the angle sum of a triangle is always equal to $180^\circ$. In the figure: $A+B+C=180^\circ$.

In hyperbolic geometry, the angle sum of a triangle is always strictly positive and strictly less than $180^\circ$.  In the figure: $0^\circ<A+B+C<180^\circ$.

In spherical geometry, the angle sum of a triangle is always strictly greater than $180^\circ$ and strictly less than $540^\circ$.  In the figure:  $180^\circ<A+B+C<540^\circ$.

Also in spherical geometry, a triangle has these additional requirements:  It must be strictly contained in a hemisphere of the sphere that is serving as the model for spherical geometry, and all of its angles must have a measure strictly less that $180^{\circ}$.

Triangles can be classified according to the number of their equal sides. So, a triangle with 3 equal sides is called \PMlinkname{\emph{equilateral}}{RegularTriangle}, a triangle with 2 equal sides is called \emph{isosceles}, and finally a triangle with no equal sides is called \emph{scalene}. Notice that an \PMlinkescapetext{equilateral triangle} is also isosceles, but there are isosceles triangles that are not equilateral.

\begin{center}
\includegraphics{trianglebyside}
\end{center}

In Euclidean geometry, triangles can also be classified according to the \PMlinkescapetext{size} of the greatest of its three (inner) angles. If the greatest of these is acute (and therefore all three are acute), the triangle is called an \emph{acute triangle}. If the triangle has a right angle, it is a \emph{right triangle}. If the triangle has an obtuse angle, it is an \emph{obtuse triangle}.

\begin{center}
\includegraphics{trianglebyangle}
\end{center}

\subsubsection*{Area of a triangle}
There are several ways to \PMlinkescapetext{calculate} a triangle's area.

In hyperbolic and spherical \PMlinkescapetext{geometry}, the area of a triangle is equal to its defect (measured in radians).

For the rest of this entry, only Euclidean geometry will be 
considered.

Many \PMlinkescapetext{formulas} for the area of a triangle exist.  The most basic one is $\displaystyle A=\frac{1}{2}bh$, where $b$ is its base and $h$ is its height.  Following is a \PMlinkescapetext{derivation} of another \PMlinkescapetext{formula} for the area of a triangle.
 
Let $a,b,c$ be the sides and $A,B,C$ the interior angles \PMlinkescapetext{opposite} to them. Let $h_a,h_b,h_c $ be the heights drawn upon $a,b,c$ respectively, $r$ the inradius and $R$ the circumradius. Finally, let $\displaystyle s=\frac{a+b+c}{2}$ be the semiperimeter. Then

\begin{eqnarray*}
\text{Area} &=& \frac{a h_a}{2}=\frac{b h_b}{2}=\frac{c h_c}{2}\\
&=& \frac{ab\sin C}{2}=\frac{bc\sin A}{2}=\frac{ca\sin B}{2}\\
&=& \frac{abc}{4R}\\
&=& sr\\
&=& \sqrt{s(s-a)(s-b)(s-c)}
\end{eqnarray*}
The last \PMlinkescapetext{formula} is known as Heron's formula.

With the coordinates of the vertices \,$(x_1,\,y_1)$,\, $(x_2,\,y_2)$,\, $(x_3,\,y_3)$\, of the triangle, the area may be expressed as
$$ 
\pm\frac{1}{2}\left|\begin{matrix}
x_1 & y_1 & 1\\
x_2 & y_2 & 1\\
x_3 & y_3 & 1
\end{matrix}\right|
$$
(cf. the volume of \PMlinkname{tetrahedron}{Tetrahedron}).

Inequalities for the area are Weizenbock's inequality and the Hadwiger-Finsler inequality.

\subsubsection*{Angles in a triangle}
\begin{enumerate}
\item the sum of the angles in a triangle is $\pi$ radians ($180^\circ$)
\item sines law 
\item cosines law 
\item Mollweide's equations 
\end{enumerate}

\subsubsection*{Special geometric objects for a triangle}
\begin{enumerate}
\item incenter 
\item inscribed circle
\item circumcenter
\item circumscribed circle 
\item centroid
\item orthocenter
\item Lemoine point, Lemoine circle  
\item Gergonne point, Gergonne triangle 
\item orthic triangle 
\item pedal triangle 
\item medial triangle 
\item Euler Line 
\end{enumerate}
%%%%% 1
% 12
%%%%%
%%%%%
%%%%%
%%%%%
%%%%%
%%%%%
\end{document}
