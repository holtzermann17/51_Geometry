\documentclass[12pt]{article}
\usepackage{pmmeta}
\pmcanonicalname{SectorOfACircle}
\pmcreated{2013-03-22 13:10:20}
\pmmodified{2013-03-22 13:10:20}
\pmowner{CWoo}{3771}
\pmmodifier{CWoo}{3771}
\pmtitle{sector of  a circle}
\pmrecord{6}{33617}
\pmprivacy{1}
\pmauthor{CWoo}{3771}
\pmtype{Definition}
\pmcomment{trigger rebuild}
\pmclassification{msc}{51-00}
\pmsynonym{sector}{SectorOfACircle}

% this is the default PlanetMath preamble.  as your knowledge
% of TeX increases, you will probably want to edit this, but
% it should be fine as is for beginners.

% almost certainly you want these
\usepackage{amssymb}
\usepackage{amsmath}
\usepackage{amsfonts}

% used for TeXing text within eps files
%\usepackage{psfrag}
% need this for including graphics (\includegraphics)
\usepackage{graphicx}
% for neatly defining theorems and propositions
%\usepackage{amsthm}
% making logically defined graphics
%%%\usepackage{xypic}

% there are many more packages, add them here as you need them

% define commands here
\begin{document}
\PMlinkescapeword{interior}
\PMlinkescapeword{complete}

A sector is a fraction of the interior of a circle, described by a central angle $\theta$.
If $\theta = 2 \pi,$ the sector becomes a complete circle. 

\begin{center}
\includegraphics{sector.eps}
\end{center}

If the central angle is $\theta,$ and the radius of the circle is $r,$  then the area of the sector is given by
$$\frac{1}{2}r^2\theta$$

This is obvious from the fact that the area of a sector is $\frac{\theta}{2 \pi}$ times the area of the circle (which is $\pi r^2$).
Note that, in the formula,  $\theta$ is in radians.

\textbf{Remark}.  Since the length $a$ of the arc of the sector is $r\theta$, the area of the sector is $\frac{1}{2}ar$, which is equal to the area of a triangle with base $=a$ and the height $=r$.
%%%%%
%%%%%
\end{document}
