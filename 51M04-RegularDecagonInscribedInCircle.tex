\documentclass[12pt]{article}
\usepackage{pmmeta}
\pmcanonicalname{RegularDecagonInscribedInCircle}
\pmcreated{2013-03-22 17:34:26}
\pmmodified{2013-03-22 17:34:26}
\pmowner{pahio}{2872}
\pmmodifier{pahio}{2872}
\pmtitle{regular decagon inscribed in circle}
\pmrecord{10}{39985}
\pmprivacy{1}
\pmauthor{pahio}{2872}
\pmtype{Theorem}
\pmcomment{trigger rebuild}
\pmclassification{msc}{51M04}
\pmsynonym{regular decagon}{RegularDecagonInscribedInCircle}
\pmrelated{RegularPolygonAndCircles}
\pmrelated{HomogeneousEquation}
\pmrelated{Pentagon}
\pmdefines{golden section}

% this is the default PlanetMath preamble.  as your knowledge
% of TeX increases, you will probably want to edit this, but
% it should be fine as is for beginners.

% almost certainly you want these
\usepackage{amssymb}
\usepackage{amsmath}
\usepackage{amsfonts}

% used for TeXing text within eps files
%\usepackage{psfrag}
% need this for including graphics (\includegraphics)
%\usepackage{graphicx}
% for neatly defining theorems and propositions
 \usepackage{amsthm}
% making logically defined graphics
%%%\usepackage{xypic}

% there are many more packages, add them here as you need them

\usepackage{pstricks}

% define commands here

\theoremstyle{definition}
\newtheorem*{thmplain}{Theorem}

\begin{document}
If a line segment has been divided into two parts such that the greater part is the central proportional of the whole segment and the smaller part, then one has performed the {\em golden section} (Latin {\em sectio aurea}) of the line segment.\\


\textbf{Theorem.}  The side of the \PMlinkname{regular}{RegularPolygon} \PMlinkname{decagon}{Polygon}, inscribed in a circle, is equal to the greater part of the radius divided with the \PMlinkescapetext{golden section}.

{\em Proof.}  A \PMlinkname{regular polygon can be inscribed in a circle}{RegularPolygonAndCircles}.  In the picture below, there is seen an isosceles central triangle $OAB$ of a regular decagon with the central angle \,$O = 360^\circ\!:\!10 = 36^\circ$; the base angles are\, $(180^\circ\!-\!36^\circ)\!:\!2 = 72^\circ$.  One of the base angles is halved with the line $AC$, when one gets a smaller isosceles triangle $ABC$ with equal angles as in the triangle $OAB$.  From these similar triangles we obtain the proportion equation
\begin{align}                    
               r:s \,=\, s:(r\!-\!s),
\end{align}
which shows that the side $s$ of the regular decagon is the central proportional of the radius $r$ of the circle and the difference $r\!-\!s$.  

\begin{center}
\begin{pspicture}(0,0)(8,5)
\psarc(0,0){7}{-4}{40}
\pspolygon[linecolor=blue](0,0)(7,0)(5.663,4.114)
\psline[linecolor=red]{-}(4.326,0)(5.663,4.114)
\psdots(0,0)(7,0)(5.663,4.114)(4.326,0)
\rput[a](0,-0.3){$O$}
\rput[a](7.2,-0.2){$B$}
\rput[a](5.9,4.14){$A$}
\rput[a](4.3,-0.3){$C$}
\rput[a](2.35,1.9){$r$}
\rput[a](2.4,-0.2){$s$}
\rput[a](4.65,1.8){$s$}
\rput[a](6.17,1.8){$s$}
\rput[a](5.6,-0.2){$r\!-\!s$}
\rput[a](0.67,0.22){$36^\circ$}
\rput[a](6.6,0.22){$72^\circ$}
\rput[a](5.1,3.3){$36^\circ$}
\rput[a](5.65,3.1){$36^\circ$}
\rput[a](4.8,0.22){$72^\circ$}
\end{pspicture}
\end{center}


\textbf{Note.}  (1) can be simplified to the \PMlinkname{quadratic equation}{QuadraticFormula}
                   $$s^2\!+\!rs\!-\!r^2 = 0$$
which yields the positive solution
      $$s\; =\; \frac{-1\!+\!\sqrt{5}}{2}\,r\; \approx\; 0.618\,r.$$
Cf. also the golden ratio.
%%%%%
%%%%%
\end{document}
