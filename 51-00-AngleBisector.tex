\documentclass[12pt]{article}
\usepackage{pmmeta}
\pmcanonicalname{AngleBisector}
\pmcreated{2013-03-22 13:10:38}
\pmmodified{2013-03-22 13:10:38}
\pmowner{giri}{919}
\pmmodifier{giri}{919}
\pmtitle{angle bisector}
\pmrecord{4}{33623}
\pmprivacy{1}
\pmauthor{giri}{919}
\pmtype{Definition}
\pmcomment{trigger rebuild}
\pmclassification{msc}{51-00}
\pmsynonym{interior angle bisector}{AngleBisector}
\pmsynonym{exterior angle bisector}{AngleBisector}

% this is the default PlanetMath preamble.  as your knowledge
% of TeX increases, you will probably want to edit this, but
% it should be fine as is for beginners.

% almost certainly you want these
\usepackage{amssymb}
\usepackage{amsmath}
\usepackage{amsfonts}

% used for TeXing text within eps files
%\usepackage{psfrag}
% need this for including graphics (\includegraphics)
\usepackage{graphicx}
% for neatly defining theorems and propositions
%\usepackage{amsthm}
% making logically defined graphics
%%%\usepackage{xypic}

% there are many more packages, add them here as you need them

% define commands here
\begin{document}
\PMlinkescapeword{interior}
\PMlinkescapeword{meet}
\PMlinkescapeword{divides}
For every angle, there exists a line that divides the angle into two equal parts.
This line is called the \textbf{angle bisector}.

\begin{center}
\includegraphics{angb.eps}
\end{center}

The interior bisector of an angle is the line or line segment that divides it into two equal angles on the same side as the angle.

The exterior bisector of an angle is the line or line segment that divides it into two equal angles on the opposite side as the angle.

For a triangle, the point where the angle bisectors of the three angles meet is called the incenter.
%%%%%
%%%%%
\end{document}
