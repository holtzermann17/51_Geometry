\documentclass[12pt]{article}
\usepackage{pmmeta}
\pmcanonicalname{ParallelAndPerpendicularPlanes}
\pmcreated{2013-04-19 15:18:51}
\pmmodified{2013-04-19 15:18:51}
\pmowner{pahio}{2872}
\pmmodifier{pahio}{2872}
\pmtitle{parallel and perpendicular planes}
\pmrecord{10}{41600}
\pmprivacy{1}
\pmauthor{pahio}{2872}
\pmtype{Theorem}
\pmcomment{trigger rebuild}
\pmclassification{msc}{51M04}
\pmrelated{PlaneNormal}
\pmrelated{NormalOfPlane}
\pmrelated{ParallelismOfTwoPlanes}

% this is the default PlanetMath preamble.  as your knowledge
% of TeX increases, you will probably want to edit this, but
% it should be fine as is for beginners.

% almost certainly you want these
\usepackage{amssymb}
\usepackage{amsmath}
\usepackage{amsfonts}

% used for TeXing text within eps files
%\usepackage{psfrag}
% need this for including graphics (\includegraphics)
%\usepackage{graphicx}
% for neatly defining theorems and propositions
 \usepackage{amsthm}
% making logically defined graphics
%%%\usepackage{xypic}

% there are many more packages, add them here as you need them

% define commands here

\theoremstyle{definition}
\newtheorem*{thmplain}{Theorem}

\begin{document}
\textbf{Theorem 1.}\, If a plane ($\pi$) intersects two parallel planes ($\varrho$, $\sigma$), the intersection lines are parallel.

{\em Proof.}\, The intersection lines cannot have common points, because $\varrho$ and $\sigma$ have no such ones.\, Since the lines are in a same plane $\pi$, they are parallel.\\


\textbf{Theorem 2.}\, If a plane ($\pi$) contains the \PMlinkname{normal}{PlaneNormal} ($n$) of another plane ($\varrho$), the planes are \PMlinkname{perpendicular}{DihedralAngle} to each other.

{\em Proof.}\, Draw in the plane $\varrho$ the line $l$ cutting the intersection line perpendicularly and cutting also $n$.\, Then $l$ must be perpendicular to $n$ and thus to the whole plane $\pi$ (see the Theorem in the entry normal of plane).\, Consequently, the right angle formed by the lines $n$ and $l$ is the normal section of the dihedral angle formed by the planes $\pi$ and $\varrho$.\, Therefore,\, $\pi \bot \varrho$.
%%%%%
%%%%%
\end{document}
