\documentclass[12pt]{article}
\usepackage{pmmeta}
\pmcanonicalname{Pentagon}
\pmcreated{2013-03-22 12:10:19}
\pmmodified{2013-03-22 12:10:19}
\pmowner{rspuzio}{6075}
\pmmodifier{rspuzio}{6075}
\pmtitle{pentagon}
\pmrecord{8}{31397}
\pmprivacy{1}
\pmauthor{rspuzio}{6075}
\pmtype{Definition}
\pmcomment{trigger rebuild}
\pmclassification{msc}{51-00}
\pmrelated{Triangle}
\pmrelated{Polygon}
\pmrelated{Hexagon}
\pmrelated{RegularDecagonInscribedInCircle}

\usepackage{amssymb}
\usepackage{amsmath}
\usepackage{amsfonts}
\usepackage{graphicx}
%%%\usepackage{xypic}

\begin{document}
A \emph{pentagon} is a 5-sided planar polygon.

Regular pentagons are of particular interest for geometers.
On a regular pentagon, the inner angles are equal to $108^\circ$.
All ten diagonals have the same length. If $s$ is the length of a side and $d$ is the length of a diagonal, then 
$$\frac{d}{s}=\frac{1+\sqrt{5}}{2};$$
that is, the ratio between a diagonal and a side is the Golden Number.

A regular pentagon (along with its diagonals) can also be obtained as
the projection of a regular pentahedron in four dimensional space
onto a plane determined by two opposite edges.
This is analogous to the way a square with its diagonals can be obtained 
as the projection of a tetrahedrononto a plane determined by two opposite edges.
%%%%%
%%%%%
%%%%%
\end{document}
