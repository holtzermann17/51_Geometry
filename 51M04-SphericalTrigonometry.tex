\documentclass[12pt]{article}
\usepackage{pmmeta}
\pmcanonicalname{SphericalTrigonometry}
\pmcreated{2013-03-22 17:08:13}
\pmmodified{2013-03-22 17:08:13}
\pmowner{fernsanz}{8869}
\pmmodifier{fernsanz}{8869}
\pmtitle{spherical trigonometry}
\pmrecord{12}{39443}
\pmprivacy{1}
\pmauthor{fernsanz}{8869}
\pmtype{Topic}
\pmcomment{trigger rebuild}
\pmclassification{msc}{51M04}
%\pmkeywords{Spherical Trigonometry}
\pmrelated{AreaOfASphericalTriangle}

% this is the default PlanetMath preamble.  as your knowledge
% of TeX increases, you will probably want to edit this, but
% it should be fine as is for beginners.

% almost certainly you want these
\usepackage{amssymb}
\usepackage{amsmath}
\usepackage{amsfonts}
\usepackage{amsthm}
\usepackage{graphicx}
\usepackage{psfrag}
%%\usepackage{xypic}

% there are many more packages, add them here as you need them

% define commands here
\newcommand{\norm}[1]{\left\Vert#1\right\Vert}
\newcommand{\bd}[1]{\mathbf{#1}}
\begin{document}
% we don't sign articles -- attribution is generated automatically by PlanetMath
%\title{Spherical Trigonometry}%
%\author{Fernando Sanz Gamiz}%
{\bf Cosine law.}

\medskip
In the following we deduce the cosine law for a spherical trihedron.

Let $\mathbf {e_1, e_2, e_3}$ be the vertex unitary vectors as shown
in the figure. \bigskip

\begin{figure}[h]
\begin{center}
\scalebox{.7}{\includegraphics{TrigonometriaEsferica.epsi}}
\caption{Spherical Trihedron used to deduce trigonometric relations}
\label{fig1}
\end{center}
\end{figure}

\noindent The cosine of the angle $\alpha$ formed by the plane
defined by $\mathbf {e_1, e_2}$ and the plane defined by $\mathbf
{e_1, e_3}$ is:

$$\cos \alpha=\frac{( \bd{e_1} \times \bd{e_3}) \cdot (\bd{e_1} \times \bd{e_2})}{\norm{\bd{e_1} \times
\bd{e_3}}\norm{\bd{e_1} \times \bd{e_2}}} = \frac{( \bd{e_1} \times
\bd{e_3}) \cdot (\bd{e_1} \times \bd{e_2})}{\sin b \sin c}$$ 

\noindent Now, using the cyclic property of the triple vector product and \PMlinkname{Lagrange's formula}{TripleCrossProduct}, we can write:

\medskip

$$\cos \alpha=\frac{ \bd{e_1} \cdot (\bd{e_3} \times (\bd{e_1}
\times \bd{e_2}))}{\sin b \sin c} = \frac{ \bd{e_1} \cdot
((\bd{e_3}\cdot\bd{e_2})\bd{e_1}-(\bd{e_3}\cdot\bd{e_1})\bd{e_2})
}{\sin b \sin c}=\frac{\cos a -\cos b \cos c}{\sin b \sin c}$$

Hence: $$\cos a=\cos b \cos c + \sin b \sin c \cos \alpha$$
%%%%%
%%%%%
\end{document}
