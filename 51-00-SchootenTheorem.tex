\documentclass[12pt]{article}
\usepackage{pmmeta}
\pmcanonicalname{SchootenTheorem}
\pmcreated{2013-03-22 14:05:50}
\pmmodified{2013-03-22 14:05:50}
\pmowner{mathcam}{2727}
\pmmodifier{mathcam}{2727}
\pmtitle{Schooten theorem}
\pmrecord{7}{35482}
\pmprivacy{1}
\pmauthor{mathcam}{2727}
\pmtype{Theorem}
\pmcomment{trigger rebuild}
\pmclassification{msc}{51-00}
\pmsynonym{Ptolemy's theorem}{SchootenTheorem}

\usepackage{amssymb}
\usepackage{amsmath}
\usepackage{amsfonts}
\usepackage{graphicx}
\usepackage{amsthm}
%%\usepackage{xypic}
\begin{document}
\textbf{Theorem:} Let $ABC$ be a equilateral triangle. If $M$ is a
point on the circumscribed circle then the equality $$AM=BM+CM$$
holds.

\textbf{Proof:} Let $B'\in\ (MA)$ so that $MB'=B'B$. Because
$\widehat{BMA}=\widehat{BCA}=60^\circ$, the triangle $MBB'$ is
equilateral, so $BB'=MB=MB'$. Because $AB=BC, BB'=BM$ and
$\widehat{ABB'}\equiv\widehat{MBC}$ we have that the triangles
$ABB'$ and $CBM$ are equivalent. Since $MC=AB'$ we have that
$AM=AB'+B'M=MC+MB$. $\square$

\begin{center}
\includegraphics{schooten}
\end{center}

\begin{thebibliography}{9}
\bibitem{Pritchard}[Pritchard] Pritchard, Chris (ed.) \emph{The Changing Shape of Geometry : Celebrating a Century of Geometry and Geometry Teaching.}  Cambridge University Press, 2003.
\end{thebibliography}
%%%%%
%%%%%
\end{document}
