\documentclass[12pt]{article}
\usepackage{pmmeta}
\pmcanonicalname{ProvingThalesTheoremWithVectors}
\pmcreated{2013-03-22 17:47:45}
\pmmodified{2013-03-22 17:47:45}
\pmowner{pahio}{2872}
\pmmodifier{pahio}{2872}
\pmtitle{proving Thales' theorem with vectors}
\pmrecord{5}{40257}
\pmprivacy{1}
\pmauthor{pahio}{2872}
\pmtype{Proof}
\pmcomment{trigger rebuild}
\pmclassification{msc}{51F20}
\pmclassification{msc}{51F20}
\pmsynonym{vector proof of Thales' theorem}{ProvingThalesTheoremWithVectors}
\pmrelated{ThalesTheorem}
\pmrelated{SumOfVectors}
\pmrelated{DifferenceOfVectors}
\pmrelated{ScalarSquare}
\pmrelated{ParallelogramPrinciple}

\endmetadata

% this is the default PlanetMath preamble.  as your knowledge
% of TeX increases, you will probably want to edit this, but
% it should be fine as is for beginners.

% almost certainly you want these
\usepackage{amssymb}
\usepackage{amsmath}
\usepackage{amsfonts}

% used for TeXing text within eps files
%\usepackage{psfrag}
% need this for including graphics (\includegraphics)
%\usepackage{graphicx}
% for neatly defining theorems and propositions
 \usepackage{amsthm}
% making logically defined graphics
%%%\usepackage{xypic}

% there are many more packages, add them here as you need them

\usepackage{pstricks}

% define commands here

\theoremstyle{definition}
\newtheorem*{thmplain}{Theorem}

\begin{document}
\begin{center}
\begin{pspicture}(-2.5,-0.5)(2.5,2.5)
\psdot[linecolor=black](0,0)
\rput[a](0,-0.3){$O$}
\rput[a](-2.2,-0.1){$B$}
\rput[a](2.15,-0.1){$A$}
\rput[a](1.33,1.8){$P$}
\psarc[linecolor=blue](0,0){2}{0}{180}
\psline[linecolor=blue](-2,0)(0,0)
\psline[arrows=->,arrowsize=5pt,linecolor=blue](1.2,1.6)(0,0)
\psline[arrows=->,arrowsize=5pt,linecolor=blue](1.2,1.6)(-2,0)
\psline[arrows=->,arrowsize=5pt,linecolor=blue](1.2,1.6)(2,0)
\psline[arrows=->,arrowsize=5pt,linecolor=red](0,0)(2,0)
\rput[a](-1,-0.2){$r$}
\rput[a](+1,-0.2){$r$}
\rput[a](0.75,0.7){$r$}
\end{pspicture}
\end{center}
Let the radius of the circle be $r$ and $AB$ a diameter of the circle.  We make the dot product calculation
$$\overrightarrow{PA}\cdot\overrightarrow{PB} = (\overrightarrow{PO}+\overrightarrow{OA})\cdot(\overrightarrow{PO}+\overrightarrow{OB}) =
(\overrightarrow{PO}+\overrightarrow{OA})\cdot(\overrightarrow{PO}-\overrightarrow{OA}) =
\overrightarrow{PO}\cdot\overrightarrow{PO}-\overrightarrow{OA}\cdot\overrightarrow{OA} =
r^2-r^2 = 0.$$
The result shows that\, $\overrightarrow{PA} \perp \overrightarrow{PB}$, i.e. the circumferential angle $APB$ of the half-circle is a right angle.

%%%%%
%%%%%
\end{document}
