\documentclass[12pt]{article}
\usepackage{pmmeta}
\pmcanonicalname{AngleSumIdentity}
\pmcreated{2013-03-22 12:50:36}
\pmmodified{2013-03-22 12:50:36}
\pmowner{mathcam}{2727}
\pmmodifier{mathcam}{2727}
\pmtitle{angle sum identity}
\pmrecord{14}{33170}
\pmprivacy{1}
\pmauthor{mathcam}{2727}
\pmtype{Theorem}
\pmcomment{trigger rebuild}
\pmclassification{msc}{51-00}
\pmrelated{ProofOfDeMoivreIdentity}
\pmrelated{DoubleAngleIdentity}

\endmetadata

% this is the default PlanetMath preamble.  as your knowledge
% of TeX increases, you will probably want to edit this, but
% it should be fine as is for beginners.

% almost certainly you want these
\usepackage{amssymb}
\usepackage{amsmath}
\usepackage{amsfonts}

% used for TeXing text within eps files
%\usepackage{psfrag}
% need this for including graphics (\includegraphics)
\usepackage{graphicx}
% for neatly defining theorems and propositions
%\usepackage{amsthm}
% making logically defined graphics
%%%\usepackage{xypic}

% there are many more packages, add them here as you need them

% definitions 

%delete-creep work-around.........................
\begin{document}
It is desired to prove the identities
\[ \sin(\theta+\phi) = \sin\theta\cos\phi + \cos\theta\sin\phi \]
and
\[ \cos(\theta+\phi) = \cos\theta\cos\phi - \sin\theta\sin\phi \]

Consider the figure

\begin{center}
\includegraphics[width = 5.3cm, height = 5.0cm]{sclaw1.eps}
\end{center}

where we have
\begin{list}{$\circ$}{}
\item $\triangle Aad \equiv \triangle Ccb$
\item $\triangle Bba \equiv \triangle Ddc$
\item $ad = dc = 1$.
\end{list}

Also, everything is Euclidean, and in particular, the interior angles of any triangle sum to $\pi$.

Call $\angle Aad = \theta$ and $\angle baB = \phi$.
From the triangle \PMlinkescapetext{sum rule}, we have $\angle Ada = \frac{\pi}{2}-\theta$ and $\angle Ddc = \frac{\pi}{2} - \phi$, while the degenerate
angle $\angle AdD = \pi$, so that
\[ \angle adc = \theta + \phi\]
We have, therefore, that the area of the pink parallelogram is $\sin(\theta + \phi)$.  On the other hand, we can rearrange things thus:

\begin{center}
\includegraphics[width = 5.6cm, height=5.1cm]{sclaw2.eps}
\end{center}

In this figure we see an equal pink area, but it is composed of two pieces, of areas $\sin \phi \cos\theta$ and $\cos\phi \sin\theta$.  Adding, 
we have 
$$ \sin(\theta + \phi) = \sin\phi\cos\theta + \cos\phi\sin\theta$$
which gives us the first.
From definitions, it then also follows that $\sin(\theta + \pi/2) = \cos(\theta)$, and $\sin(\theta+\pi) = - \sin(\theta)$.
Writing
\[ \begin{array}{rcl}
\cos(\theta + \phi)&=& \sin(\theta+\phi+\pi/2) \\
&=& \sin(\theta)\cos(\phi+\pi/2) + \cos(\theta) \sin(\phi + \pi/2)\\
&=& \sin(\theta)\sin(\phi + \pi) + \cos(\theta) \cos(\phi) \\
&=& \cos\theta\cos\phi - \sin\theta\sin\phi
\end{array} \]
%delete-creep work-around.................................................
%%%%%
%%%%%
\end{document}
