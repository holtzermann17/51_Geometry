\documentclass[12pt]{article}
\usepackage{pmmeta}
\pmcanonicalname{Biangle}
\pmcreated{2013-03-22 17:06:10}
\pmmodified{2013-03-22 17:06:10}
\pmowner{Wkbj79}{1863}
\pmmodifier{Wkbj79}{1863}
\pmtitle{biangle}
\pmrecord{8}{39400}
\pmprivacy{1}
\pmauthor{Wkbj79}{1863}
\pmtype{Definition}
\pmcomment{trigger rebuild}
\pmclassification{msc}{51M10}
\pmclassification{msc}{51-00}

\usepackage{amssymb}
\usepackage{amsmath}
\usepackage{amsfonts}

\usepackage{psfrag}
\usepackage{graphicx}
\usepackage{amsthm}
%%\usepackage{xypic}

\begin{document}
In spherical geometry, it is possible to form a polygon with only two sides.  Thus, we have the following definition:

A \emph{biangle} is a two-sided polygon that is strictly contained in one hemisphere of the sphere that is serving as the model for spherical geometry.

Given a biangle, its vertices must be antipodal points, and its two angles must be congruent.  Therefore, every biangle is equiangular.  Since each side of a biangle is half of a great circle, every biangle is equilateral.  Hence, every biangle is regular.

Let $\theta$ be the radian \PMlinkname{measure}{AngleMeasure} of each angle of a biangle.  Then the biangle \PMlinkname{covers}{Cover} $\displaystyle \frac{\theta}{2\pi}$ of the sphere.  Since the area of the sphere is $4\pi$, the area of the biangle is $2\theta$.  Note that this equals the angle sum of the biangle.
%%%%%
%%%%%
\end{document}
