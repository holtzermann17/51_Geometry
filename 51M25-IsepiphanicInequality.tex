\documentclass[12pt]{article}
\usepackage{pmmeta}
\pmcanonicalname{IsepiphanicInequality}
\pmcreated{2013-03-22 19:19:08}
\pmmodified{2013-03-22 19:19:08}
\pmowner{pahio}{2872}
\pmmodifier{pahio}{2872}
\pmtitle{isepiphanic inequality}
\pmrecord{6}{42257}
\pmprivacy{1}
\pmauthor{pahio}{2872}
\pmtype{Theorem}
\pmcomment{trigger rebuild}
\pmclassification{msc}{51M25}
\pmclassification{msc}{51M16}

\endmetadata

% this is the default PlanetMath preamble.  as your knowledge
% of TeX increases, you will probably want to edit this, but
% it should be fine as is for beginners.

% almost certainly you want these
\usepackage{amssymb}
\usepackage{amsmath}
\usepackage{amsfonts}

% used for TeXing text within eps files
%\usepackage{psfrag}
% need this for including graphics (\includegraphics)
%\usepackage{graphicx}
% for neatly defining theorems and propositions
 \usepackage{amsthm}
% making logically defined graphics
%%%\usepackage{xypic}

% there are many more packages, add them here as you need them

% define commands here

\theoremstyle{definition}
\newtheorem*{thmplain}{Theorem}

\begin{document}
The classical \emph{isepiphanic inequality} 
 $$36\pi V^2 \;\leqq\; A^3,$$
concerns the volume $V$ and the area $A$ of any solid in $\mathbb{R}^3$.\, It asserts that the ball has the greatest volume among the solids having a given area.\\

For a ball with radius $r$, we have
$$V \;=\; \frac{4}{3}\pi r^3, \qquad A \;=\; 4\pi r^2,$$
whence it follows the equality
 $$36\pi V^2 \;=\; A^3.$$\\
Cf. the isoperimetric inequality and the isoperimetric problem.

\begin{thebibliography}{8}
\bibitem{nord}{\sc Patrik Nordbeck}: {\em Isoperimetriska problemet eller Varf\"or ser man s\aa\, f\aa\, fyrkantiga tr\"ad?}\, Examensarbete.\, Lund University (1995). 
\end{thebibliography}

%%%%%
%%%%%
\end{document}
