\documentclass[12pt]{article}
\usepackage{pmmeta}
\pmcanonicalname{ParallelismOfLineAndPlane}
\pmcreated{2013-03-22 18:47:58}
\pmmodified{2013-03-22 18:47:58}
\pmowner{pahio}{2872}
\pmmodifier{pahio}{2872}
\pmtitle{parallelism of line and plane}
\pmrecord{6}{41599}
\pmprivacy{1}
\pmauthor{pahio}{2872}
\pmtype{Theorem}
\pmcomment{trigger rebuild}
\pmclassification{msc}{51M04}
\pmrelated{ParallelismOfTwoPlanes}

% this is the default PlanetMath preamble.  as your knowledge
% of TeX increases, you will probably want to edit this, but
% it should be fine as is for beginners.

% almost certainly you want these
\usepackage{amssymb}
\usepackage{amsmath}
\usepackage{amsfonts}

% used for TeXing text within eps files
%\usepackage{psfrag}
% need this for including graphics (\includegraphics)
%\usepackage{graphicx}
% for neatly defining theorems and propositions
 \usepackage{amsthm}
% making logically defined graphics
%%%\usepackage{xypic}

% there are many more packages, add them here as you need them

% define commands here

\theoremstyle{definition}
\newtheorem*{thmplain}{Theorem}

\begin{document}
Parallelity of a line and a plane means that the angle between line and plane is 0, i.e. the line and the plane have either no or infinitely many common points.\\


\textbf{Theorem 1.}\, If a line ($l$) is parallel to a line ($m$) contained in a plane ($\pi$), then it is parallel to the plane or is contained in the plane.

{\em Proof.}\, 
So,\, $l \,||\, m \subset \pi$.\, If\, $l \not\subset \pi$,\, we can set a set along the parallel lines $l$ and $m$ another plane $\varrho$.\, The common points of $\pi$ and $\varrho$ are on the intersection line $m$ of the planes.\, If $l$ would intersect the plane $\pi$, then it would intersect also the line $m$, contrary to the assumption.\, Thus\, $l \,||\, \pi$.\\


\textbf{Theorem 2.}\, If a plane is set along a line ($l$) which is parallel to another plane ($\pi$), then the intersection line ($m$) of the planes is parallel to the first-mentioned line.

{\em Proof.}\, The lines $l$ and $m$ are in a same plane, and they cannot intersect each other since otherwise $l$ would intersect the plane $\pi$ which would contradict the assumption.\, Accordingly,\, $m \,||\, l$.
%%%%%
%%%%%
\end{document}
