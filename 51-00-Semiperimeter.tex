\documentclass[12pt]{article}
\usepackage{pmmeta}
\pmcanonicalname{Semiperimeter}
\pmcreated{2013-03-22 16:58:13}
\pmmodified{2013-03-22 16:58:13}
\pmowner{Wkbj79}{1863}
\pmmodifier{Wkbj79}{1863}
\pmtitle{semiperimeter}
\pmrecord{9}{39245}
\pmprivacy{1}
\pmauthor{Wkbj79}{1863}
\pmtype{Definition}
\pmcomment{trigger rebuild}
\pmclassification{msc}{51-00}
\pmrelated{Perimeter2}
\pmrelated{BasicPolygon}

\usepackage{amssymb}
\usepackage{amsmath}
\usepackage{amsfonts}

\usepackage{psfrag}
\usepackage{graphicx}
\usepackage{amsthm}
%%\usepackage{xypic}

\begin{document}
The \emph{semiperimeter} of a polygon is half of its \PMlinkname{perimeter}{Perimeter2}.

Semiperimeters are most commonly used for determining the area of a triangle.  The most famous example of this usage is Heron's formula.  There is also a \PMlinkescapetext{formula} for the area of a quadrilateral which involves semiperimeter.
%%%%%
%%%%%
\end{document}
