\documentclass[12pt]{article}
\usepackage{pmmeta}
\pmcanonicalname{GeometryAsTheStudyOfInvariantsUnderCertainTransformations}
\pmcreated{2013-03-22 18:00:29}
\pmmodified{2013-03-22 18:00:29}
\pmowner{rspuzio}{6075}
\pmmodifier{rspuzio}{6075}
\pmtitle{geometry as the study of invariants under certain transformations}
\pmrecord{6}{40522}
\pmprivacy{1}
\pmauthor{rspuzio}{6075}
\pmtype{Topic}
\pmcomment{trigger rebuild}
\pmclassification{msc}{51-01}
\pmclassification{msc}{51-00}
%\pmkeywords{geometric algebra}

\endmetadata

\usepackage{amssymb,amscd}
\usepackage{amsmath}
\usepackage{amsfonts}
\usepackage{mathrsfs}

% used for TeXing text within eps files
%\usepackage{psfrag}
% need this for including graphics (\includegraphics)
%\usepackage{graphicx}
% for neatly defining theorems and propositions
\usepackage{amsthm}
% making logically defined graphics
%%\usepackage{xypic}
\usepackage{pst-plot}

% define commands here
\newcommand*{\abs}[1]{\left\lvert #1\right\rvert}
\newtheorem{prop}{Proposition}
\newtheorem{thm}{Theorem}
\newtheorem{ex}{Example}
\newcommand{\real}{\mathbb{R}}
\newcommand{\pdiff}[2]{\frac{\partial #1}{\partial #2}}
\newcommand{\mpdiff}[3]{\frac{\partial^#1 #2}{\partial #3^#1}}
\begin{document}
An approach to geometry first formulated by Felix Klein in his
Erlangen lectures is to describe it as the study of invariants under
certain allowed transformations. This involves taking our space as a
set $S$, and considering a subgroup $G$ of the group $Bij(S)$, the set
of bijections of $S$.  Objects are subsets of $S$, and we consider two
objects $A,B \subset S$ to be equivalent if there is an $f \in G$ such
that $f(A) = B$.

A property $P$ of subsets of $S$ is said to be a \emph{geometric
property} if it is invariant under the action of the group $G$, which
is to say that $P(S)$ is true (or false) if and only if $P(g(S))$ is
true (or false) for every transformation $g \in G$.  For example, the
property of being a straight line is a geometric property in Euclidean
geometry.  Note that the question whether or not a certin property is
geometric depends on the choice of group.  For instance, in the case
of Euclidean geometry, the property of orthogonality is geometric
because, given two lines $L_1$ and $L_2$ and any transformation $g$
which belongs to the Euclidean group, the lines $g(L_1)$ and $g(L_2)$
are orthogonal if and only if $L_1$ and $L_2$ are orthogonal.
However, if we consider affine geometry, orthogonality is no longer a
geometric property because, given two orthogonal lines $L_1$ and
$L_2$, one can find a transformation $f$ which belongs to the affine
group such that $f(L_1)$ is not orthogonal to $f(L_2)$.

Invariants can also be numbers.  A real-valued function $f$ whose
domain consists of subsets of $S$ is an invariant, or a
\emph{geometrical quantity} if the domain of $X$ is invariant under
the action of $G$ and $f(X) = f(g(X))$ for all subsets $X$ in the
domain of $f$ and all transformations $g \in G$.  Familiar examples
from Euclidean geometry are the length of line segments, areas of
triangles, and angles.  An important feature of the group-theoretic
approach to geometry is that one one can use the techniques of
invariant theory to systematically find and classify the invariants of
a geometrical system.  Using this approach, one can start with the
description of a geometrical system in terms of a set and a group and
rediscover geometric quantities which were originally found by trial
and error.

One is not always interested in considering all possible subsets of
$S$.  For instance, in algebraic geometry, one only cares about
subsets which can be defined by sytems of algebraic equations.  To
accommodate this desire, one may revise Klein's definition by
replacing the set $S$ with a suitable category (such as the category
of algebraic subsets) to obtain the definition ``geometry is the study
of the invariants of a category $C$ under the action of a group $G$
which acts upon this category.''  Not only is such an approach popular
in contemporary algebraic geometry, it is also useful when discussing
such phenomena as duality transforms which map a point in one space to
a line in another space and vice-versa.  Such a phenomenon is not
easily accomodated in a set-theoretic framework, but in terms of
category theory, the duality transform can be described as a
contravariant functor.

Klein's definition provides an organizing principle for classifying
geometries.  Ever since the discovery of non-Euclidean geometry,
geometers have been defined and studied many different geometries.
Without an organizing principle, the discussion and comparison of
these geometries could become confusing.  In the next section, we
shall describe several familiar geometric systems from the standpoint
of Klein's definition.

\subsection{Basic examples}
\subsubsection{Euclidean geometry}
Euclidean geometry deals with $\mathbb{R}^n$ as a vector space along with a metric $d$. The allowed transformations are 
bijections $f\colon \mathbb{R}^n \to \mathbb{R}^n$ that preserve the metric, that is, $d(\boldsymbol{x},\boldsymbol{y}) 
= d(f(\boldsymbol{x}),f(\boldsymbol{y}))$ for all $\boldsymbol{x},\boldsymbol{y} \in \mathbb{R}^n$. Such maps are called 
\emph{isometries}, and the group is often denoted by $\operatorname{Iso}(\mathbb{R}^n)$. Defining a norm by $|x| = 
d(\boldsymbol{x},\boldsymbol{0})$, for $\boldsymbol{x} \in \mathbb{R}^n$, we obtain a notion of length or distance.
We can also define an inner product $\langle \boldsymbol{x},\boldsymbol{y}\rangle = \boldsymbol{x} \cdot \boldsymbol{y}$ on $\mathbb{R}^n$ using the standard dot
product (this induces the same norm which can now be defined as
$|x| = \sqrt{\langle \boldsymbol{x},\boldsymbol{x}\rangle}$).
An inner product leads to a definition of
the angle between two vectors $\boldsymbol{x},\boldsymbol{y} \in \mathbb{R}^n$ to be 
$\displaystyle \angle{\boldsymbol{x}}{\boldsymbol{y}} = \cos^{-1}\left(\frac{\langle \boldsymbol{x},\boldsymbol{y}\rangle}
{|\boldsymbol{x}| \cdot |\boldsymbol{y}|}\right).$  It is clear that since isometries preserve the metric, they preserve distance and angle. As an example, it can be shown 
that the group $\operatorname{Iso}(\mathbb{R}^2)$ consists of translations, reflections, glides, and rotations.  In 
general, a member $f$ of $\operatorname{Iso}(\mathbb{R}^n)$ has the form $f(\boldsymbol{x})=\boldsymbol{Ux}+
\boldsymbol{c}$, where $\boldsymbol{U}$ is an orthogonal $n\times n$ matrix and $\boldsymbol{c}\in\mathbb{R}^n$.

\subsubsection{Affine geometry} 
 Unlike Euclidean geometry, we are no longer bound to ``rigid motion''
transformations in affine geometry.  Here, we are interested in what
happens to geometric objects when they undergo a finite series of
``parallel projections''.  For example, imagine two Euclidean planes
($\mathbb{R}^2$) in $\mathbb{R}^3$.  Loosely speaking, Euclidean
geometry deals with transformations that take objects from one plane
to the other, when the planes are \emph{parallel} to each other.  In
affine geometry, the transformation is between two copies of
$\mathbb{R}^2$, but they are no longer required to be parallel to each
other anymore.  Objects from one plane will appear to be ``stretched''
in the other.  A circle will turn into an ellipse, etc...

For $\mathbb{R}^2$, in terms of the Kleinian view of geometry, affine
geometry consists of the ordinary Euclidean plane, together with a
group of transformations that
\begin{enumerate}
\item map straight lines to straight lines,
\item map parallel lines to parallel lines, and
\item preserve ratios of lengths of line segments along a given
straight line.  
\end{enumerate}
Of course, the properties can be generalized to $\mathbb{R}^n$ and
$n-1$ dimensional hyperplanes.  A typical tranformation in an affine
geometry is called an \PMlinkname{affine
transformation}{AffineTransformation}:
$T(\boldsymbol{x})=\boldsymbol{Ax}+\boldsymbol{b}$, where
$x\in\mathbb{R}^n$ and $\boldsymbol{A}$ is an invertible $n\times n$
real matrix.

\subsubsection{Projective geometry}
Projective geometry was motivated by how we see objects in everyday
life. For example, parallel train tracks appear to meet at a point far
away, even though they are always the same distance apart. In
projective geometry, the primary invariant is that of incidence. The
notion of parallelism and distance is not present as with Euclidean
geometry. There are different ways of approaching projective
geometry. One way is to add points of infinity to Euclidean space. For
example, we may form the projective line by adding a point of infinity
$\infty$, called the ideal point, to $\mathbb{R}$. We can then create
the projective plane where for each line $l \in \mathbb{R}^2$, we
attach an ideal point, and two ordinary lines have the same ideal
point if and only if they are parallel. The projective plane then
consists of the regular plane $\mathbb{R}^2$ along with the ideal
line, which consists of all ideal points of all ordinary lines. The
idea here is to make central projections from a point sending a line
to another a bijective map.

Another approach is more algebraic, where we form $P(V)$ where V is a
vector space. When $V = \mathbb{R}^n$, we take the quotient of $
\mathbb{R}^{n+1} - \{0\}$ where $v \sim \lambda \cdot v$ for $v \in
\mathbb{R}^n, \lambda \in \mathbb{R}$.  The allowed transformations is
the group $PGL(\mathbb{R}^{n+1})$, which is the general linear group
modulo the subgroup of scalar matrices.

\subsubsection{Spherical geometry}
Spherical geometry deals with restricting our attention in Euclidean
space to the unit sphere $S^n$. The role of straight lines is taken by
great circles. Notions of distance and angles can be easily developed,
as well as spherical laws of cosines, the law of sines, and spherical
triangles.
%%%%%
%%%%%
\end{document}
