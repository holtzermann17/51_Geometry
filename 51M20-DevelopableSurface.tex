\documentclass[12pt]{article}
\usepackage{pmmeta}
\pmcanonicalname{DevelopableSurface}
\pmcreated{2013-03-22 15:29:29}
\pmmodified{2013-03-22 15:29:29}
\pmowner{pahio}{2872}
\pmmodifier{pahio}{2872}
\pmtitle{developable surface}
\pmrecord{8}{37348}
\pmprivacy{1}
\pmauthor{pahio}{2872}
\pmtype{Topic}
\pmcomment{trigger rebuild}
\pmclassification{msc}{51M20}
\pmclassification{msc}{51M04}
\pmsynonym{torsal surface}{DevelopableSurface}
\pmrelated{Area2}
\pmrelated{RiemannMultipleIntegral}
\pmdefines{developable}
\pmdefines{torsal generatrix}
\pmdefines{torsal}
\pmdefines{tangential surface}

\endmetadata

% this is the default PlanetMath preamble.  as your knowledge
% of TeX increases, you will probably want to edit this, but
% it should be fine as is for beginners.

% almost certainly you want these
\usepackage{amssymb}
\usepackage{amsmath}
\usepackage{amsfonts}

% used for TeXing text within eps files
%\usepackage{psfrag}
% need this for including graphics (\includegraphics)
\usepackage{graphicx}
% for neatly defining theorems and propositions
 \usepackage{amsthm}
% making logically defined graphics
%%%\usepackage{xypic}

% there are many more packages, add them here as you need them

% define commands here

\theoremstyle{definition}
\newtheorem*{thmplain}{Theorem}
\begin{document}
A generatrix of a ruled surface is {\em torsal}, if in each of its points there is one and the same tangent plane of the surface.

A ruled surface is {\em torsal} iff it only has torsal generatrices.

A surface is {\em developable}, if one can spread it out on a plane without any stretching or tearing.

K. F. Gauss has proved that a surface is developable if and only if it is a torsal  ruled surface.

One may divide the developable surfaces into three \PMlinkescapetext{types}:
\begin{enumerate}
 \item Cylindrical surfaces
 \item Conical surfaces
 \item {\em Tangential surfaces} of a space curve; they can be expressed by
$$\vec{r} = \vec{\gamma}(t)+ s\,\frac{d\vec{\gamma}(t)}{dt}$$
where\, $\vec{r} = \vec{\gamma}(t)$\, is the equation of the space curve, $s$ and $t$ are parameters.
\end{enumerate}

\begin{figure}
\begin{center}
\includegraphics{developable.eps}
\end{center}
\caption{Tangential surface of a circular helix}
\end{figure}

%%%%%
%%%%%
\end{document}
