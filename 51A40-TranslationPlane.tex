\documentclass[12pt]{article}
\usepackage{pmmeta}
\pmcanonicalname{TranslationPlane}
\pmcreated{2013-03-22 19:15:15}
\pmmodified{2013-03-22 19:15:15}
\pmowner{CWoo}{3771}
\pmmodifier{CWoo}{3771}
\pmtitle{translation plane}
\pmrecord{10}{42180}
\pmprivacy{1}
\pmauthor{CWoo}{3771}
\pmtype{Definition}
\pmcomment{trigger rebuild}
\pmclassification{msc}{51A40}
\pmclassification{msc}{51A35}
\pmrelated{MoufangPlane}
\pmdefines{translation line}
\pmdefines{dual translation plane}
\pmdefines{translation point}
\pmdefines{affine translation plane}

\endmetadata

\usepackage{amssymb,amscd}
\usepackage{amsmath}
\usepackage{amsfonts}
\usepackage{mathrsfs}

% used for TeXing text within eps files
%\usepackage{psfrag}
% need this for including graphics (\includegraphics)
%\usepackage{graphicx}
% for neatly defining theorems and propositions
\usepackage{amsthm}
% making logically defined graphics
%%\usepackage{xypic}
\usepackage{pst-plot}

% define commands here
\newcommand*{\abs}[1]{\left\lvert #1\right\rvert}
\newtheorem{prop}{Proposition}
\newtheorem{thm}{Theorem}
\newtheorem{ex}{Example}
\newcommand{\real}{\mathbb{R}}
\newcommand{\pdiff}[2]{\frac{\partial #1}{\partial #2}}
\newcommand{\mpdiff}[3]{\frac{\partial^#1 #2}{\partial #3^#1}}
\begin{document}
Let $\pi$ be a projective plane.  Recall that a central collineation on $\pi$ is a collineation $\rho$ with a center $C$ and an axis $\ell$.  It is well-known that $C$ and $\ell$ are uniquely determined.  We also call $\rho$ a $(C,\ell)$-collineation.

\textbf{Definition}.  Let $\pi$ be a projective plane.  We say that $\pi$ is \emph{$(C,\ell)$-transitive} if there is a point $C$ and a line $\ell$, such that for any points $P,Q$ where
\begin{itemize}
\item $P,Q$ and $C$ are collinear and pairwise distinct,
\item $P,Q\notin \ell$,
\end{itemize}
there is a $(C,\ell)$-collineation $\rho$ such that $\rho(P)=Q$.

It can be shown that $\pi$ if $(C,\ell)$-transitive iff it is $(C,\ell)$-Desarguesian; that is, if two triangles are perspective from point $C$, then they are perspective from line $\ell$.  From this, it is easy to see that $\pi$ is a Desarguesian plane iff it is $(C,\ell)$-transitive for any point $C$ and any line $\ell$, of $\pi$.

Now, suppose that $C$ lies on $\ell$.  Then one can show that $\pi$ is $(C,\ell)$-transitive iff it can be coordinatized by a linear ternary ring $R$ such that $R$ is a group with respect to the derived operation $+$ (addition).  When $\pi$ is so coordinatized, $\ell$ is the line at infinity, and $C$ is the point whose coordinate is $(\infty)$.

This group is not necessarily abelian.  So what condition(s) must be imposed on $\pi$ so that $(R,+)$ is an abelian group?  The answer lies in the next definition:

\textbf{Definition}.  Let $\pi$ be a projective plane.  $\pi$ is said to be $(m,\ell)$-transitive if there are lines $m,\ell$ such that $\pi$ is $(C,\ell)$-transitive for all $C\in m$.

\textbf{Definition}.  A projective plane $\pi$ is a \emph{translation plane} if there is a line $\ell$ such that $\pi$ is $(\ell,\ell)$-transitive.  We also say that $\pi$ is a translation plane with respect to $\ell$.  The line $\ell$ is called a \emph{translation line} of $\pi$.

It can be shown that $\pi$ is a translation plane with respect to $\ell$ iff it can be coordinatized by a Veblen-Wedderburn system (thus implying that $(R,+)$ is abelian).

When $\pi$ is a translation plane with respect to two distinct lines $\ell$ and $m$, then it is not hard to see that it is a translation plane with respect to every line passing through $\ell\cap m$.

When $\pi$ is a translation plane with respect to three non-concurrent lines, then it is a translation plane with respect to every line.  A projective plane which is a translation plane with respect to every line is called a Moufang plane.  An example of a translation plane that is not Moufang is the Hall plane, coordinatized by the Hall quasifield.  An example of a projective plane that is not a translation plane is the Hughes plane.

\textbf{Remark}.  There are also duals to the notions above: a projective plane $\pi$ is
\begin{enumerate}
\item
\emph{$(P,Q)$-transitive} if there are points $P,Q$ such that $\pi$ is $(P,m)$-transitive for any line $m$ passing through $Q$.  
\item
a \emph{dual translation plane} if there is a point $P$ such that $\pi$ is $(P,P)$-transitive.  We also say that $\pi$ is a dual translation plane with respect to $P$, and that $P$ is a \emph{translation point} of $\pi$.
\end{enumerate}

If $\pi$ is a projective plane, then the following are true:
\begin{itemize}
\item
$\pi$ is translation plane with respect to some line $\ell$ and a dual translation plane with respect to some $P\in \ell$ iff $\pi$ can be coordinatized by a semifield.  In this coordinatization, $\ell$ is the line at infinity and $P$ is the point with coordinate $(\infty)$.
\item
$\pi$ is translation plane with respect to some line $PQ$ and $(P,Q)$- and $(Q,P)$-transitive iff $\pi$ can be coordinatized by a nearfield.  In this coordinatization, $PQ$ is the line at infinity where $P$ and $Q$ have coordinates $(0)$ and $(\infty)$ (or vice versa).
\end{itemize}

\textbf{Remark}.  By removing the line at infinity from a translation plane, we obtain an \emph{affine translation plane}.  By the definition of a translation plane, an affine translation plane can be characterized as an affine plane where the minor affine Desarguesian property holds.

\begin{thebibliography}{7}
\bibitem{RC} R. Casse, {\it Projective Geometry, An Introduction}, Oxford University Press (2006)
\end{thebibliography}
%%%%%
%%%%%
\end{document}
