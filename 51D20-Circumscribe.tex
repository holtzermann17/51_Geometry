\documentclass[12pt]{article}
\usepackage{pmmeta}
\pmcanonicalname{Circumscribe}
\pmcreated{2013-03-22 16:24:32}
\pmmodified{2013-03-22 16:24:32}
\pmowner{PrimeFan}{13766}
\pmmodifier{PrimeFan}{13766}
\pmtitle{circumscribe}
\pmrecord{11}{38557}
\pmprivacy{1}
\pmauthor{PrimeFan}{13766}
\pmtype{Definition}
\pmcomment{trigger rebuild}
\pmclassification{msc}{51D20}
\pmsynonym{circumscribed}{Circumscribe}
\pmsynonym{circumscription}{Circumscribe}
\pmrelated{RegularPolygonAndCircles}
\pmrelated{Inscription}

\endmetadata

% this is the default PlanetMath preamble.  as your knowledge
% of TeX increases, you will probably want to edit this, but
% it should be fine as is for beginners.

% almost certainly you want these
\usepackage{amssymb}
\usepackage{amsmath}
\usepackage{amsfonts}

\usepackage{pstricks}

% used for TeXing text within eps files
%\usepackage{psfrag}
% need this for including graphics (\includegraphics)
%\usepackage{graphicx}
% for neatly defining theorems and propositions
%\usepackage{amsthm}
% making logically defined graphics
%%%\usepackage{xypic}

% there are many more packages, add them here as you need them

% define commands here

\begin{document}
To \emph{circumscribe} a regular polygon (such as a triangle, square, pentagon, etc.) or a cyclic quadrilateral, is to enclose it in a circle so that the vertices  of the polygon are on the circumference of the circle.

For example, a circumscribed square:

\begin{center}
\begin{pspicture}(-3,-3)(3,3)
\pscircle(0,0){2.13}
\pspolygon(1.5,1.5)(-1.5,1.5)(-1.5,-1.5)(1.5,-1.5)
\end{pspicture}
\end{center}

One more example, a circumscribed pentagon.

\begin{center}
\begin{pspicture}(-2,-2)(2,2)
\pscircle(0,0){1.702}
\pspolygon(0,1.702)(-1.619,0.526)(-1,-1.377)(1,-1.377)(1.619,0.526)
\end{pspicture}
\end{center}

%%%%%
%%%%%
\end{document}
