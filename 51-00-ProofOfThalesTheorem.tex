\documentclass[12pt]{article}
\usepackage{pmmeta}
\pmcanonicalname{ProofOfThalesTheorem}
\pmcreated{2013-03-22 12:45:27}
\pmmodified{2013-03-22 12:45:27}
\pmowner{mathwizard}{128}
\pmmodifier{mathwizard}{128}
\pmtitle{proof of Thales' theorem}
\pmrecord{6}{33064}
\pmprivacy{1}
\pmauthor{mathwizard}{128}
\pmtype{Proof}
\pmcomment{trigger rebuild}
\pmclassification{msc}{51-00}

% this is the default PlanetMath preamble.  as your knowledge
% of TeX increases, you will probably want to edit this, but
% it should be fine as is for beginners.

% almost certainly you want these
\usepackage{amssymb}
\usepackage{amsmath}
\usepackage{amsfonts}

% used for TeXing text within eps files
%\usepackage{psfrag}
% need this for including graphics (\includegraphics)
\usepackage{graphicx}
% for neatly defining theorems and propositions
%\usepackage{amsthm}
% making logically defined graphics
%%%\usepackage{xypic}

% there are many more packages, add them here as you need them

% define commands here
\begin{document}
Let $M$ be the center of the circle through $A$, $B$ and $C$.

\begin{center}
\includegraphics{thales.eps}
\end{center}

Then $AM=BM=CM$ and thus the triangles $AMC$ and $BMC$ are isosceles. If $\angle BMC=:\alpha$ then $\angle MCB=90^\circ-\frac{\alpha}{2}$ and $\angle CMA=180^\circ-\alpha$. Therefore $\angle ACM=\frac{\alpha}{2}$ and
$$\angle ACB=\angle MCB+\angle ACM=90^\circ.$$
QED.
%%%%%
%%%%%
\end{document}
