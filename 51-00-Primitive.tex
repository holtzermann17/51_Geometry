\documentclass[12pt]{article}
\usepackage{pmmeta}
\pmcanonicalname{Primitive}
\pmcreated{2013-03-22 17:11:49}
\pmmodified{2013-03-22 17:11:49}
\pmowner{Wkbj79}{1863}
\pmmodifier{Wkbj79}{1863}
\pmtitle{primitive}
\pmrecord{8}{39516}
\pmprivacy{1}
\pmauthor{Wkbj79}{1863}
\pmtype{Definition}
\pmcomment{trigger rebuild}
\pmclassification{msc}{51-00}

\usepackage{amssymb}
\usepackage{amsmath}
\usepackage{amsfonts}
\usepackage{pstricks}
\usepackage{psfrag}
\usepackage{graphicx}
\usepackage{amsthm}
%%\usepackage{xypic}

\begin{document}
A \emph{primitive} is a geometric figure that is left undefined.  Properties of primitives are assumed via axioms of the geometry.  From these axioms, one can gain some understanding of primitives, but they are technically not defined.

Primitives are typically the simplest figure considered in the geometry.  This occurs because it is easier to create more complicated figures from simpler ones; it seems backwards to leave complicated figures undefined and attempt to define simpler figures from these.

In Euclidean geometry, the primitives are typically taken to be point, line, and plane.  On PlanetMath, these all have ``definitions''.  Upon investigation, however, Euclid's definition for point, ``that which has no part,'' is a description rather than a definition, and the other definitions require knowledge of vector spaces, topology, or point-free geometry.  Similarly, the PlanetMath definitions of line and plane require knowledge of other \PMlinkescapetext{areas} of mathematics.

The \PMlinkescapetext{word} ``figure'' in the geometric sense is currently a primitive according to PlanetMath.  It is used in many different entries, including this one (in the first \PMlinkescapetext{sentence} for example), with the assumption that the reader knows what is meant by the \PMlinkescapetext{term}.
%%%%%
%%%%%
\end{document}
