\documentclass[12pt]{article}
\usepackage{pmmeta}
\pmcanonicalname{CyclicQuadrilateral}
\pmcreated{2013-03-22 11:44:16}
\pmmodified{2013-03-22 11:44:16}
\pmowner{drini}{3}
\pmmodifier{drini}{3}
\pmtitle{cyclic quadrilateral}
\pmrecord{12}{30150}
\pmprivacy{1}
\pmauthor{drini}{3}
\pmtype{Definition}
\pmcomment{trigger rebuild}
\pmclassification{msc}{51-00}
\pmclassification{msc}{81R50}
\pmclassification{msc}{81P05}
\pmclassification{msc}{81Q05}
\pmclassification{msc}{81-00}
\pmsynonym{cyclic}{CyclicQuadrilateral}
\pmrelated{OrthicTriangle}
\pmrelated{PtolemysTheorem}
\pmrelated{ProofOfPtolemysTheorem}
\pmrelated{Circumcircle}
\pmrelated{Quadrilateral}

\endmetadata

\usepackage{amssymb}
\usepackage{amsmath}
\usepackage{amsfonts}
\usepackage{graphicx}
%%%%%%%\usepackage{xypic}
\begin{document}
\textbf{Cyclic quadrilateral.}\\
A quadrilateral is cyclic when its four vertices lie on a circle.

\begin{center}
\includegraphics{quadcyclic}
\end{center}

A necessary and sufficient condition for a quadrilateral to be cyclic, is that the sum of a pair of opposite angles be equal to $180^\circ$.

One of the main results about these quadrilaterals is Ptolemy's theorem.

Also, from all the quadrilaterals with given sides $p,q,r,s$, the one that is cyclic has the greatest area. If the four sides of a cyclic quadrilateral are known, the area can be found using Brahmagupta's formula
%%%%%
%%%%%
%%%%%
%%%%%
%%%%%
%%%%%
%%%%%
\end{document}
