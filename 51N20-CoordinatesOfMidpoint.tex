\documentclass[12pt]{article}
\usepackage{pmmeta}
\pmcanonicalname{CoordinatesOfMidpoint}
\pmcreated{2013-03-22 17:31:07}
\pmmodified{2013-03-22 17:31:07}
\pmowner{pahio}{2872}
\pmmodifier{pahio}{2872}
\pmtitle{coordinates of midpoint}
\pmrecord{5}{39910}
\pmprivacy{1}
\pmauthor{pahio}{2872}
\pmtype{Result}
\pmcomment{trigger rebuild}
\pmclassification{msc}{51N20}
\pmclassification{msc}{51M15}
\pmclassification{msc}{51-00}
\pmrelated{ConjugateDiametersOfEllipse}
\pmrelated{CentreOfMassOfPolygon}
\pmrelated{Midpoint4}

\endmetadata

% this is the default PlanetMath preamble.  as your knowledge
% of TeX increases, you will probably want to edit this, but
% it should be fine as is for beginners.

% almost certainly you want these
\usepackage{amssymb}
\usepackage{amsmath}
\usepackage{amsfonts}

% used for TeXing text within eps files
%\usepackage{psfrag}
% need this for including graphics (\includegraphics)
%\usepackage{graphicx}
% for neatly defining theorems and propositions
 \usepackage{amsthm}
% making logically defined graphics
%%%\usepackage{xypic}

% there are many more packages, add them here as you need them

% define commands here

\theoremstyle{definition}
\newtheorem*{thmplain}{Theorem}

\begin{document}
The coordinates of the midpoint of a line segment are the arithmetic means of the coordinates of the endpoints of the segment.  Thus, if the endpoints are\, $(x_1,\,y_1)$\, and\, $(x_2,\,y_2)$,\, then the midpoint is
   $$\left(\frac{x_1\!+\!x_2}{2},\, \frac{y_1\!+\!y_2}{2}\right)\!.$$

For justifying the above coordinates of the midpoint, we know that its abscissa $x_0$ halves on the $x$-axis the line segment between $x_1$ and $x_2$.  Since the lengths of the half-segments are $x_0\!-\!x_1$ and $x_2\!-\!x_0$, if\, 
$x_1 < x_2$,\, and their opposite numbers, if\, $x_2 < x_1$,\, in any case we can write
$$x_0-x_1 = x_2-x_0.$$
Solving this equation for $x_0$ yields:\, $\displaystyle x_0 = \frac{x_1\!+\!x_2}{2}$.\, \PMlinkescapetext{Similar} result is gotten for the ordinate of the midpoint.
%%%%%
%%%%%
\end{document}
