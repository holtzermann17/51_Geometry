\documentclass[12pt]{article}
\usepackage{pmmeta}
\pmcanonicalname{Cevian}
\pmcreated{2013-03-22 12:10:57}
\pmmodified{2013-03-22 12:10:57}
\pmowner{CWoo}{3771}
\pmmodifier{CWoo}{3771}
\pmtitle{cevian}
\pmrecord{9}{31446}
\pmprivacy{1}
\pmauthor{CWoo}{3771}
\pmtype{Definition}
\pmcomment{trigger rebuild}
\pmclassification{msc}{51M99}
\pmrelated{Triangle}
\pmrelated{CevasTheorem}
\pmrelated{ProofOfApolloniusTheorem2}
\pmrelated{TrigonometricVersionOfCevasTheorem}
\pmrelated{Median}
\pmrelated{HeightOfATriangle}

\usepackage{graphicx}
%%%%\usepackage{xypic} 
\usepackage{bbm}
\newcommand{\Z}{\mathbbmss{Z}}
\newcommand{\C}{\mathbbmss{C}}
\newcommand{\R}{\mathbbmss{R}}
\newcommand{\Q}{\mathbbmss{Q}}
\newcommand{\mathbb}[1]{\mathbbmss{#1}}
\begin{document}
A \emph{cevian} of a triangle, is any line segment joining a vertex with a point of the opposite side.

\begin{center}\includegraphics{cevian}\end{center}
$AD$ is a cevian of $\triangle ABC$.

If $D$ is the midpoint of $BC$, then the cevian $AD$ is a \emph{median}.  If $AD$ is perpendicular to $BC$, then the cevian is a \emph{height}.
%%%%%
%%%%%
%%%%%
\end{document}
