\documentclass[12pt]{article}
\usepackage{pmmeta}
\pmcanonicalname{PartsOfABall}
\pmcreated{2013-03-22 18:18:36}
\pmmodified{2013-03-22 18:18:36}
\pmowner{pahio}{2872}
\pmmodifier{pahio}{2872}
\pmtitle{parts of a ball}
\pmrecord{10}{40932}
\pmprivacy{1}
\pmauthor{pahio}{2872}
\pmtype{Definition}
\pmcomment{trigger rebuild}
\pmclassification{msc}{51M05}
\pmsynonym{parts of ball}{PartsOfABall}
\pmsynonym{parts of sphere}{PartsOfABall}
\pmrelated{CircularSegment}
\pmdefines{spherical segment}
\pmdefines{spherical frustum}
\pmdefines{spherical cap}
\pmdefines{spherical calotte}
\pmdefines{spherical sector}

\endmetadata

% this is the default PlanetMath preamble.  as your knowledge
% of TeX increases, you will probably want to edit this, but
% it should be fine as is for beginners.

% almost certainly you want these
\usepackage{amssymb}
\usepackage{amsmath}
\usepackage{amsfonts}

% used for TeXing text within eps files
%\usepackage{psfrag}
% need this for including graphics (\includegraphics)
%\usepackage{graphicx}
% for neatly defining theorems and propositions
 \usepackage{amsthm}
% making logically defined graphics
%%%\usepackage{xypic}

% there are many more packages, add them here as you need them

% define commands here

\theoremstyle{definition}
\newtheorem*{thmplain}{Theorem}

\begin{document}
\PMlinkescapeword{height}

Let us consider in $\mathbb{R}^3$ a ball of radius $r$ and the sphere bounding the ball.
\begin{itemize}

\item Two parallel planes intersecting the ball separate between them from the ball a {\em spherical segment}, which can also be called a {\em spherical frustum} (see the frustum).\, The curved surface of the spherical segment is the {\em spherical zone}.

\item In the special case that one of the planes is a tangent plane of the sphere, the spherical segment is a {\em spherical cap} and the spherical zone is a {\em spherical calotte}.

\item The lateral surface of a circular cone with its apex in the \PMlinkname{centre}{Sphere} of the ball divides the ball into two {\em spherical sectors}.
\end{itemize}

The distance $h$ of the two planes intersecting the ball be is called the {\em height}.

The volume of the spherical cap is obtained from
$$V \,=\, \pi h^2\left(r\!-\!\frac{h}{3}\right)$$
and the area of the corresponding spherical calotte and also a spherical zone from
$$A \,=\, 2\pi rh.$$
The volume of a spherical segment can be got as the difference of the volumes of two spherical caps.

The volume of a spherical sector may be calculated from
$$V \,=\, \frac{2}{3}\pi r^2h,$$
where $h$ is the height of the spherical cap of the spherical sector.

%%%%%
%%%%%
\end{document}
