\documentclass[12pt]{article}
\usepackage{pmmeta}
\pmcanonicalname{CompassAndStraightedgeConstructionOfGeometricMean}
\pmcreated{2013-03-22 17:14:55}
\pmmodified{2013-03-22 17:14:55}
\pmowner{Wkbj79}{1863}
\pmmodifier{Wkbj79}{1863}
\pmtitle{compass and straightedge construction of geometric mean}
\pmrecord{10}{39581}
\pmprivacy{1}
\pmauthor{Wkbj79}{1863}
\pmtype{Algorithm}
\pmcomment{trigger rebuild}
\pmclassification{msc}{51M15}
\pmclassification{msc}{51-00}
\pmrelated{ConstructionOfCentralProportion}

\endmetadata

\usepackage{amssymb}
\usepackage{amsmath}
\usepackage{amsfonts}
\usepackage{pstricks}
\usepackage{psfrag}
\usepackage{graphicx}
\usepackage{amsthm}
%%\usepackage{xypic}

\begin{document}
\PMlinkescapeword{label}
\PMlinkescapeword{order}

Given line segments of lengths $a$ and $b$, one can construct a line segment of length $\sqrt{ab}$ using compass and straightedge as follows:

\begin{enumerate}
\item Draw a line segment of length $a$.  Label its endpoints $A$ and $C$.

\begin{center}
\begin{pspicture}(-3,-2)(3,3)
\rput[r](-0.4,0){.}
\psline[linecolor=blue](-2.5,0)(-0.5,0)
\psdots(-2.5,0)(-0.5,0)
\rput[a](-2.5,-0.3){$A$}
\rput[a](-0.8,0.2){$C$}
\end{pspicture}
\end{center}

\item Extend the line segment past $C$.

\begin{center}
\begin{pspicture}(-3,-2)(3,3)
\rput[r](3,0){.}
\psline(-2.5,0)(-0.5,0)
\psline[linecolor=blue]{->}(-0.5,0)(3,0)
\psdots(-2.5,0)(-0.5,0)
\rput[a](-2.5,-0.3){$A$}
\rput[a](-0.8,0.2){$C$}
\end{pspicture}
\end{center}

\item Mark off a line segment of length $b$ such that one of its endpoints is $C$.  Label its other endpoint as $B$.

\begin{center}
\begin{pspicture}(-3,-2)(3,3)
\rput[r](3,0){.}
\psline(-2.5,0)(-0.5,0)
\psline{->}(2.5,0)(3,0)
\psline[linecolor=blue](-0.5,0)(2.5,0)
\psdots(-2.5,0)(-0.5,0)(2.5,0)
\rput[a](-2.5,-0.3){$A$}
\rput[a](-0.8,0.2){$C$}
\rput[a](2.5,-0.3){$B$}
\end{pspicture}
\end{center}

\item Construct the perpendicular bisector of $\overline{AB}$ in order to find its midpoint $M$.

\begin{center}
\begin{pspicture}(-3,-3)(3,3)
\rput[r](3,0){.}
\rput[a](0,3){.}
\rput[b](0,-3){.}
\psline{->}(-2.5,0)(3,0)
\psarc[linecolor=blue](-2.5,0){3}{-45}{45}
\psarc[linecolor=blue](2.5,0){3}{135}{225}
\psline[linecolor=blue]{<->}(0,-3)(0,3)
\psdots(-2.5,0)(-0.5,0)(0,0)(2.5,0)
\rput[a](-2.5,-0.3){$A$}
\rput[a](-0.8,0.2){$C$}
\rput[a](0.3,-0.3){$M$}
\rput[a](2.5,-0.3){$B$}
\end{pspicture}
\end{center}

\item Construct a semicircle with center $M$ and radii $\overline{AM}$ and $\overline{BM}$.

\begin{center}
\begin{pspicture}(-3,-3)(3,3)
\rput[r](3,0){.}
\rput[a](0,3){.}
\rput[b](0,-3){.}
\psline{->}(-2.5,0)(3,0)
\psarc(-2.5,0){3}{-45}{45}
\psarc(2.5,0){3}{135}{225}
\psline{<->}(0,-3)(0,3)
\psarc[linecolor=blue](0,0){2.5}{0}{180}
\psdots(-2.5,0)(-0.5,0)(0,0)(2.5,0)
\rput[a](-2.5,-0.3){$A$}
\rput[a](-0.8,0.2){$C$}
\rput[a](0.3,-0.3){$M$}
\rput[a](2.5,-0.3){$B$}
\end{pspicture}
\end{center}

\item Erect the perpendicular to $\overline{AB}$ at $C$ to find the point $D$ where it intersects the semicircle.  The line segment $\overline{DC}$ is of the desired length.

\begin{center}
\begin{pspicture}(-3,-3)(3,3)
\rput[r](3,0){.}
\rput[a](0,3){.}
\rput[b](0,-3){.}
\psline{->}(-2.5,0)(3,0)
\psarc(-2.5,0){3}{-45}{45}
\psarc(2.5,0){3}{135}{225}
\psline{<->}(0,-3)(0,3)
\psarc(0,0){2.5}{0}{180}
\psarc[linecolor=blue](-0.5,0){0.5}{160}{380}
\psarc[linecolor=blue](-1,0){0.7}{-60}{60}
\psarc[linecolor=blue](0,0){0.7}{120}{240}
\psline[linecolor=blue]{<-}(-0.5,-1)(-0.5,2.45)
\psdots(-2.5,0)(-0.5,0)(0,0)(2.5,0)(-0.5,2.45)
\rput[a](-2.5,-0.3){$A$}
\rput[a](-0.8,0.2){$C$}
\rput[a](0.3,-0.3){$M$}
\rput[a](2.5,-0.3){$B$}
\rput[b](-0.5,2.65){$D$}
\end{pspicture}
\end{center}

This construction is justified because, if $\overline{AD}$ and $\overline{BD}$ were drawn, then the two smaller triangles would be similar, yielding

$$\frac{AC}{DC}=\frac{DC}{BC}.$$

Plugging in $AC=a$ and $BC=b$ gives that $DC=\sqrt{ab}$ as desired.

If you are interested in seeing the rules for compass and straightedge constructions, click on the \PMlinkescapetext{link} provided.

\end{enumerate}
%%%%%
%%%%%
\end{document}
