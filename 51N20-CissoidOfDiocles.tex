\documentclass[12pt]{article}
\usepackage{pmmeta}
\pmcanonicalname{CissoidOfDiocles}
\pmcreated{2013-03-22 17:17:22}
\pmmodified{2013-03-22 17:17:22}
\pmowner{pahio}{2872}
\pmmodifier{pahio}{2872}
\pmtitle{cissoid of Diocles}
\pmrecord{16}{39632}
\pmprivacy{1}
\pmauthor{pahio}{2872}
\pmtype{Definition}
\pmcomment{trigger rebuild}
\pmclassification{msc}{51N20}
\pmclassification{msc}{51-00}
\pmsynonym{cissoid}{CissoidOfDiocles}
\pmsynonym{ivy curve}{CissoidOfDiocles}
\pmrelated{Parameter}
\pmrelated{AnalyticGeometry}

\endmetadata

% this is the default PlanetMath preamble.  as your knowledge
% of TeX increases, you will probably want to edit this, but
% it should be fine as is for beginners.

% almost certainly you want these
\usepackage{amssymb}
\usepackage{amsmath}
\usepackage{amsfonts}

% used for TeXing text within eps files
%\usepackage{psfrag}
% need this for including graphics (\includegraphics)
%\usepackage{graphicx}
% for neatly defining theorems and propositions
 \usepackage{amsthm}
% making logically defined graphics
%%%\usepackage{xypic}
\usepackage{pstricks}
\usepackage{pst-plot}

% there are many more packages, add them here as you need them

% define commands here

\theoremstyle{definition}
\newtheorem*{thmplain}{Theorem}

\begin{document}
\PMlinkescapeword{type}
Let $c$ be a circle with diameter \,$OA = a$.  Set a tangent line $t$ of the circle at the point $A$.  For any point $C$ of $c$ let $P$ be the intersection point of the secant line $OC$ and the tangent line $t$.  Determine on the secant line between $O$ and $P$ the point $Q$ such that
$$PQ \;=\; OC.$$
Then the locus of the point $Q$ is the {\em cissoid of Diocles}.  The name is derived from Greek $\varkappa\iota\sigma\sigma\acute{o}\varsigma$ (kissos) `ivy', $\varepsilon\iota\delta{o}\varsigma$ (eidos) `form, kind, type'.

The cissoid is symmetric with regard to the line $OA$, having at $O$ a cusp.  The line $t$ is the asymptote of the curve.

For deriving the equation of the cissoid, chose the ray $OA$ for the positive $x$-axis.  Let $\varphi$ be the slope angle (polar angle) of any $C$ on $c$.  From the triangle $OAP$ we see that\, $\displaystyle OP = \frac{a}{\cos\varphi}$.  Since $\angle ACO$ is a right angle, we have\, $OC = a\cos\varphi$.  It follows that\, 
$\displaystyle OQ = r = \frac{a}{\cos\varphi}-a\cos\varphi$,\, that is,
\begin{align}              
r \;=\; a\sin\varphi\tan\varphi.
\end{align}
For obtaining the equation in rectangular coordinates, we may write (1) as\, $r^2 = (r\sin\varphi)\,a\tan\varphi$, i.e.\, $\displaystyle x^2\!+\!y^2 = ya\cdot\frac{y}{x}$,\, whence\, $(x^2\!+\!y^2)x = ay^2$,\, or
\begin{align}
y \;=\; \pm{x}\,\sqrt{\frac{x}{a\!-\!x}}.
\end{align}
Thus it's a question of an algebraic curve.\, The cissoid has also the parametric presentation
\begin{align}
x \;=\; \frac{au^2}{1\!+\!u^2}, \quad y \;=\; \frac{au^3}{1\!+\!u^2}.
\end{align}

\textbf{Note.}\, If we apply the inversion formulae
$$x \;\mapsto\; \frac{x}{x^2\!+\!y^2}, \quad y \;\mapsto\; \frac{y}{x^2\!+\!y^2}$$
(where\, $r = 1$) to the parabola \,$y = \pm\sqrt{x}$,\, we get as the image of the parabola the cissoid\, 
$\displaystyle{y} = \pm{x}\,\sqrt{\frac{x}{1-x}}$;\, correspondingly the image of this cissoid is that parabola.

\begin{center}
\begin{pspicture}(-5.5,-6.5)(5.5,5)
\psaxes[Dx=1,Dy=1]{->}(0,0)(-1.5,-5.5)(3.5,5.5)
\rput(3.6,-0.2){$x$}
\rput(0.2,5.45){$y$}
\rput(2.2,4.51){$t$}
\psline[linestyle=dashed](2,-5)(2,5)
\psplot[linecolor=blue]{0.0}{1.8}{x 1 mul 2 x sub div sqrt x mul}
\psplot[linecolor=blue]{0.0}{1.8}{x 1 mul 2 x sub div sqrt 0 x sub mul}
\rput(1,-6){$\mbox{The cissoid\,\,}y = \pm x\sqrt{\frac{x}{2-x}}$}
\end{pspicture}
\end{center}
The form of the cissoid of Diocles resembles the tractrix.

%%%%%
%%%%%
\end{document}
