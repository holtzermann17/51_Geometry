\documentclass[12pt]{article}
\usepackage{pmmeta}
\pmcanonicalname{FinitePlane}
\pmcreated{2013-03-22 13:05:34}
\pmmodified{2013-03-22 13:05:34}
\pmowner{marijke}{8873}
\pmmodifier{marijke}{8873}
\pmtitle{finite plane}
\pmrecord{18}{33510}
\pmprivacy{1}
\pmauthor{marijke}{8873}
\pmtype{Definition}
\pmcomment{trigger rebuild}
\pmclassification{msc}{51E20}
\pmclassification{msc}{05C65}
\pmclassification{msc}{51E15}
\pmclassification{msc}{05B25}
%\pmkeywords{finite geometry combinatorics}
\pmrelated{LinearSpace2}
\pmdefines{Fano plane}

\usepackage{graphicx}
%%%\usepackage{xypic} 
%\usepackage{bbm}
%\newcommand{\Z}{\mathbbmss{Z}}
%\newcommand{\C}{\mathbbmss{C}}
%\newcommand{\R}{\mathbbmss{R}}
%\newcommand{\Q}{\mathbbmss{Q}}
%\newcommand{\mathbb}[1]{\mathbbmss{#1}}
%\newcommand{\figura}[1]{\begin{center}\includegraphics{#1}\end{center}}
%\newcommand{\figuraex}[2]{\begin{center}\includegraphics[#2]{#1}\end{center}}
%\usepackage{amsmath}
%\usepackage{amsthm}
%\newtheorem*{thm}{}
\begin{document}
\PMlinkescapeword{order}

A {\bf finite plane} (synonym {\bf \PMlinkname{linear space}{LinearSpace2}}) is the finite (discrete) analogue of planes in more familiar geometries. It is an {\bf incidence structure} where any two {\bf points} are incident with exactly one {\bf line} (the line is said to ``pass through'' those points, the points ``lie on'' the line), and any two {\bf lines} are incident with {\em at most\/} one {\bf point} --- just like in ordinary planes, lines can be {\em parallel\/} i.e.\ not intersect in any point.

A finite plane without parallel lines is known as a {\bf projective plane}. Another kind of finite plane is an {\bf affine plane}, which can be obtained from a projective plane by removing one line (and all the points on it).

\subsection*{Example}

An example of a projective plane, that of order $2$, known as the \emph{Fano plane} (for projective planes, {\em order\/} $q$ means $q+1$ points on each line, $q+1$ lines through each point):

\begin{center}
\includegraphics{fano}
\end{center}

An edge here is represented by a straight line, and the inscribed circle is also an edge.  In other words, for a vertex set $\{1, 2, 3, 4, 5, 6, 7 \}$, the edges of the Fano plane are

$$
\{1, 2, 4 \}, 
\{2, 3, 5 \},
\{3, 4, 6 \},
\{4, 5, 7 \},
\{5, 6, 1 \},
\{6, 7, 2 \},
\{7, 1, 3 \} 
$$

Notice that the Fano plane is generated by the triple $\{1, 2, 4\}$
by repeatedly adding $1$ to each entry, modulo $7$.  The generating triple has the property that the differences of any two elements, in either order, are all
pairwise different modulo $7$.  In general, if we can find a set of $q+1$
of the integers (mod~$q^2 + q + 1$) with all pairwise differences distinct,
then this gives a cyclic representation of the finite plane of order $q$.
%%%%%
%%%%%
\end{document}
