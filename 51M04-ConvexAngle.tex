\documentclass[12pt]{article}
\usepackage{pmmeta}
\pmcanonicalname{ConvexAngle}
\pmcreated{2013-03-22 14:48:07}
\pmmodified{2013-03-22 14:48:07}
\pmowner{pahio}{2872}
\pmmodifier{pahio}{2872}
\pmtitle{convex angle}
\pmrecord{17}{36456}
\pmprivacy{1}
\pmauthor{pahio}{2872}
\pmtype{Definition}
\pmcomment{trigger rebuild}
\pmclassification{msc}{51M04}
%\pmkeywords{convex}
\pmrelated{Quadrant}
\pmrelated{SolidAngle}
\pmrelated{OrthogonalCurves}
\pmrelated{ComplementaryAngles}
\pmrelated{ContourLine}
\pmdefines{straight angle}
\pmdefines{obtuse angle}
\pmdefines{right angle}
\pmdefines{acute angle}
\pmdefines{straight}
\pmdefines{obtuse}
\pmdefines{right}
\pmdefines{acute}

% this is the default PlanetMath preamble.  as your knowledge
% of TeX increases, you will probably want to edit this, but
% it should be fine as is for beginners.

% almost certainly you want these
\usepackage{amssymb}
\usepackage{amsmath}
\usepackage{amsfonts}

% used for TeXing text within eps files
%\usepackage{psfrag}
% need this for including graphics (\includegraphics)
%\usepackage{graphicx}
% for neatly defining theorems and propositions
%\usepackage{amsthm}
% making logically defined graphics
%%%\usepackage{xypic}

% there are many more packages, add them here as you need them

% define commands here
\begin{document}
A positive angle is \PMlinkescapetext{{\em convex}}\, if it is at most 180 \PMlinkescapetext{degrees}, i.e. $\pi$ radians.\, Cf. the convex set.

A convex angle $\alpha$ is
\begin{itemize}
\item {\em straight}\, if\, $\alpha = 180^{\mathrm{o}}$,
\item {\em obtuse}\, if\, $90^{\mathrm{o}} < \alpha < 180^{\mathrm{o}}$,
\item {\em right}\, if\, $\alpha = 90^{\mathrm{o}}$,
\item {\em acute}\, if\, $0^{\mathrm{o}} < \alpha < 90^{\mathrm{o}}$,
\item {\em skew}\, if it is acute or obtuse.
\end{itemize}

\textbf{Note.}\, The \PMlinkname{angle between two curves}{ConformalMapping} which intersect each other in a point $P$ means the angle between the tangent lines of the curves in $P$; such an angle may always be chosen acute or right.
%%%%%
%%%%%
\end{document}
