\documentclass[12pt]{article}
\usepackage{pmmeta}
\pmcanonicalname{Concyclic}
\pmcreated{2013-03-22 16:07:58}
\pmmodified{2013-03-22 16:07:58}
\pmowner{CWoo}{3771}
\pmmodifier{CWoo}{3771}
\pmtitle{concyclic}
\pmrecord{7}{38204}
\pmprivacy{1}
\pmauthor{CWoo}{3771}
\pmtype{Definition}
\pmcomment{trigger rebuild}
\pmclassification{msc}{51-00}

\usepackage{amssymb,amscd}
\usepackage{amsmath}
\usepackage{amsfonts}

% used for TeXing text within eps files
%\usepackage{psfrag}
% need this for including graphics (\includegraphics)
%\usepackage{graphicx}
% for neatly defining theorems and propositions
%\usepackage{amsthm}
% making logically defined graphics
%%\usepackage{xypic}
\usepackage{pst-plot}
\usepackage{psfrag}

% define commands here

\begin{document}
In any geometry where a circle is defined, a collection of points are said to be \emph{concyclic} if there is a circle that is incident with all the points.

\textbf{Remarks}.
Suppose all points being considered below lie in a Euclidean plane.
\begin{itemize}
\item Any two points $P,Q$ are concyclic.  In fact, there are infinitely many circles that are incident to both $P$ and $Q$.  If $P\neq Q$, then the pencil $\mathfrak{P}$ of circles incident with $P$ and $Q$ share the property that their centers are collinear.  It is easy to see that any point on the perpendicular bisector of $\overline{PQ}$ serves as the center of a unique circle in $\mathfrak{P}$.
\item Any three non-collinear points $P,Q,R$ are concyclic to a unique circle $c$.  From the three points, take any two perpendicular bisectors, say of $\overline{PQ}$ and $\overline{PR}$.  Then their intersection $O$ is the center of $c$, whose radius is $|OP|$.
\item Four distinct points $A,B,C,D$ are concyclic iff $\angle CAD=\angle CBD$.
\end{itemize}
%%%%%
%%%%%
\end{document}
