\documentclass[12pt]{article}
\usepackage{pmmeta}
\pmcanonicalname{Radian}
\pmcreated{2013-03-22 14:48:58}
\pmmodified{2013-03-22 14:48:58}
\pmowner{PrimeFan}{13766}
\pmmodifier{PrimeFan}{13766}
\pmtitle{radian}
\pmrecord{7}{36476}
\pmprivacy{1}
\pmauthor{PrimeFan}{13766}
\pmtype{Definition}
\pmcomment{trigger rebuild}
\pmclassification{msc}{51M04}
\pmsynonym{absolute unit of angle}{Radian}
\pmrelated{SolidAngle}

% this is the default PlanetMath preamble.  as your knowledge
% of TeX increases, you will probably want to edit this, but
% it should be fine as is for beginners.

% almost certainly you want these
\usepackage{amssymb}
\usepackage{amsmath}
\usepackage{amsfonts}

% used for TeXing text within eps files
%\usepackage{psfrag}
% need this for including graphics (\includegraphics)
%\usepackage{graphicx}
% for neatly defining theorems and propositions
%\usepackage{amsthm}
% making logically defined graphics
%%%\usepackage{xypic}

% there are many more packages, add them here as you need them

% define commands here
\begin{document}
The {\em radian} is a \PMlinkescapetext{measure unit} of angle used in the higher mathematics. \,The magnitude of an angle $\alpha$ is one radian, if the arc corresponding the angle $\alpha$ as a central angle of a circle is equally long as the radius of the circle. \,Thus, a radian is equal to $\frac{180}{\pi}$ \PMlinkescapetext{degrees. \,It is in degrees}, minutes and seconds approximately $57^{\mathrm{o}}\,17'\,44.80625''.$

In degrees, a circle has 360 degrees, while in radians a circle has $2\pi$ radians. In fact, many angles of equilateral polygons are equal to a multiple of $\pi$ divided by some integer: for example, the interior angle of an equilateral triangle's vertex is $\frac{\pi}{3}$, while the interior angle of an equilateral pentagon's vertex is $\frac{3\pi}{5}$.
%%%%%
%%%%%
\end{document}
