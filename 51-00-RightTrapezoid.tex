\documentclass[12pt]{article}
\usepackage{pmmeta}
\pmcanonicalname{RightTrapezoid}
\pmcreated{2013-03-22 17:11:55}
\pmmodified{2013-03-22 17:11:55}
\pmowner{Wkbj79}{1863}
\pmmodifier{Wkbj79}{1863}
\pmtitle{right trapezoid}
\pmrecord{10}{39518}
\pmprivacy{1}
\pmauthor{Wkbj79}{1863}
\pmtype{Definition}
\pmcomment{trigger rebuild}
\pmclassification{msc}{51-00}
\pmsynonym{right trapezium}{RightTrapezoid}
\pmrelated{LambertQuadrilateral}
\pmrelated{SaccheriQuadrilateral}

\usepackage{amssymb}
\usepackage{amsmath}
\usepackage{amsfonts}
\usepackage{pstricks}
\usepackage{psfrag}
\usepackage{graphicx}
\usepackage{amsthm}
%%\usepackage{xypic}

\begin{document}
A \emph{right trapezoid} is a trapezoid that has at least two right angles.  Below is a picture of a right trapezoid.

\begin{center}
\begin{pspicture}(0,0)(4,2)
\pspolygon(0,0)(1,2)(4,2)(4,0)
\end{pspicture}
\end{center}

In some dialects of English (\PMlinkname{e.g.}{Eg} British English), this figure is referred to as a \emph{right trapezium}.  Because of the modifier ``right'', no confusion should arise with this usage.

All rectangles are right trapezoids (unless the \PMlinkescapetext{restricted} definition of trapezoid is used, see the entry on \PMlinkname{trapezoid}{Trapezoid} for more details).  Note also that, in Euclidean geometry, a trapezoid cannot have an odd number of right angles.

A \PMlinkescapetext{right isosceles trapezoid} is a trapezoid that is simultaneously a right trapezoid and an isosceles trapezoid.  In Euclidean geometry, such trapezoids are automatically rectangles.  In hyperbolic geometry, such trapezoids are automatically Saccheri quadrilaterals.  Thus, the phrase ``right isosceles trapezoid'' occurs rarely.

Right trapezoids are used in the trapezoidal rule and composite trapezoidal rule for estimating Riemann integrals.
%%%%%
%%%%%
\end{document}
