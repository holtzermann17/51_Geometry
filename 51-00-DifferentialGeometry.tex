\documentclass[12pt]{article}
\usepackage{pmmeta}
\pmcanonicalname{DifferentialGeometry}
\pmcreated{2013-03-22 17:13:50}
\pmmodified{2013-03-22 17:13:50}
\pmowner{rspuzio}{6075}
\pmmodifier{rspuzio}{6075}
\pmtitle{differential geometry}
\pmrecord{4}{39558}
\pmprivacy{1}
\pmauthor{rspuzio}{6075}
\pmtype{Topic}
\pmcomment{trigger rebuild}
\pmclassification{msc}{51-00}
\pmclassification{msc}{51-01}
\pmrelated{ClassicalDifferentialGeometry}
\pmrelated{DifferentialPropositionalCalculus}

\endmetadata

% this is the default PlanetMath preamble.  as your knowledge
% of TeX increases, you will probably want to edit this, but
% it should be fine as is for beginners.

% almost certainly you want these
\usepackage{amssymb}
\usepackage{amsmath}
\usepackage{amsfonts}

% used for TeXing text within eps files
%\usepackage{psfrag}
% need this for including graphics (\includegraphics)
%\usepackage{graphicx}
% for neatly defining theorems and propositions
%\usepackage{amsthm}
% making logically defined graphics
%%%\usepackage{xypic}

% there are many more packages, add them here as you need them

% define commands here

\begin{document}
\subsection{Classical differential geometry}

Differential geometry studies geometrical objects using techniques of
calculus.  In fact, its early history is indistiguishable from that of
calculus --- it is a matter of personal taste whether one chooses to
regard Fermat's method of drawing tangents and finding extrema as a
contribution to calculus or differential geometry; the pioneering work
of Barrow and Newton on calculus was presented in a geometrical
language; Halley's 1696 paper in which he announces his discovery that
$\displaystyle \int \frac{dx}{x} = \log x + C$ is entitled quadrature of the hyperbola.

It is only later on, when calculus became more algebraic in outlook
that one can begin to make a meaningful separation between the
subjects of calculus and differential geometry.  Early differential
geometers studied such properties of curves and surfaces such as:
computing their lengths and areas, finding tangents, constructing
evolute, involute, and pedal curves, studying curvature and osculating
circles, and finding envelopes and orthogonal curves to a given family
of curves.  Of the various objects they studied, the cycloid deserves
special mention.  Originally discovered by Galileo, it seems to have
been studied by just about every seventeenth century mathematician,
much more than any other curve.  Also, it is worth mentioning that the
deep connections between differential geometry and mechanics which
play a prominent role in contemporary theoretical physics also have
their origin at this time --- for instance, the fact that the problem
of geodesics on surfaces is related to the acceleration of particles
was already to known to the Bernoullis, even though the significance
of this similarity was not fully appreciated until Einstein.

\subsection{Intrinsic geometry}

A major turning point in differential geometry is marked by the
appearance of Gauss' memoir ``Disquisitiones generales circa
superficies curvas''.  In this memoir, Gauss first proposes the
intrinsic point of view in geometry.  Since his idea of has intrinsic
geometry has completely changed the outlook of geometry, it might not
be inappropriate to spend some time discussing it.

Ordinarily, when we think of a surface such as a cylinder or a sphere,
we concieve of a locus in space ($\mathbb{R}^3$).  However, imagine
that, like blind ants crawling on a sheet, we were confined to a
surface and had no direct knowledge of the space in which the surface
was situated.  What could we conclude about the surface on which we
live?  Obviously, such constructions as normals to the surface or
osculating spheres which are situated in the ambient space would be
meaningless to us.  However, we would be still able to speak of the
lengths of curves drawn on the surface, the angle between curves, and
areas of portions of our surface since these can be defined in terms
of measurements which are made on the surface without recourse to the
ambient space.

Considering differential geometry from this point of view, one comes
to several interesting conclusions.  One is that certain surfaces,
such as a portion of a plane and a portion of a cylinder are
indistinguishable from the intrinsic point of view.  This result can
be something of a surprise because one is usually accostomed to
thinking of planes and cylinders as rather different sorts of objects
--- planes are flat while cylinders are curved.

By itself this discovery is interesting, but perhaps not enough to
fire a revolution in geometry.  Hearing of it, one might come to the
conclusion that intrinsic measurements alone are not sufficient to
describe the geometry of a surface and, hence, the subject of
intrinsic geometry is uninteresting.  Further study shows that this is
not the case.  While one may not be able to distinguish a plane from a
cylinder on the basis of intrinsic measurements alone, it is possible
to distinguish a portion of plane from a portion of sphere solely on
the basis of intrinsic measurements.  Even more, it is possible to
distinguish portions of spheres of different radii intrinsically.
This, of course, is of interest not only because it shows that the
study of intrinsic geometry is non-trivial, but because a quantity
such as the radius of a sphere which is defined as the distance from a
point on the sphere to a point not on the surface (namely, the centre
of the sphere) can, in fact, be deduced solely from measurements on
the surface of the sphere.

To prove such results, Gauss used the concept of curvature.  Since the
idea of curvature, in one from or another, plays an important role in
differential geometry to this day, let us say a few words about this
key concept.  The concept of curvature was developed in the eighteenth
century as a measure how much a given curve or a surface deviates from
being a straight line or a plane.  In the case of a curve, the
curvature may be defined as the second derivative of the normal angle
with respect to arclength.  In the case of a surface, the situation is
a little more complicated --- to describe the direction of the normal
vector, one needs two angles instead of one and one can choose to
compute their directional derivatives along any direction tangent to
the surface.  Because of this, one obtains a $2 \times 2$ matrix of
partial derivatives instead of a single number.  If one changes the
coordinates used to describe the ambient space, then the components of
this matrix will undergo a linear transformation.  To obtain a
quantity which does not depend on an arbitrary choice of coordinates
and hence can be seen as describing geometric properties of the
surface, one should consider the eigenvalues of this matrix.  These
eigenvalues are known as the principal curvatures of the surface.

The remarkable theorem which Gauss proved was that, whilst the
principal curvatures cannot be determined from intrinsic measurements
alone, their product can.  This result, for instance, can be used to
explain the facts we mentioned about planes, cylinders, and spheres.
For a plane, the two principal curvatures equal zero.  For a cylinder,
one principal curvature is zero and the other is positive.  For a
sphere, both principal curvatures are equal and positive.  Since the
product equals zero for both the plane and the cylinder it is
plausible that they are indistinguishable intrinsically.  Since it is
not zero for the sphere, it follows from Gauss' theorem that the
intrinsic geometry of the sphere could not possibly be the same as
that of the plane and cylinder.

\subsection{The concept of a manifold}

The discovery of intrinsic geometry led thoughtful geometers such as
Riemann (who was a student of Gauss), Clifford, and Mach to the
conclusion that a ``right and natural'' approach to geometry should
regard surfaces as geometrical spaces in their own right on a par with
Euclidean and projective space.  In the terminology of the mediaeval
scholastic philosophers, one may say that they regarded the intrinsic
properties of surfaces as ``essential'' and extrinsic properties as
``accidental''.  To properly develop geometry from such a viewpoint,
they needed to start with a definition of surface which made no
reference to any sort of ambient space.  Their quest for such a
definition led to the concept of a manifold.

The genesis of this concept can be seen as a diasappearing act akin to
that of the Chesire cat --- just as the grin is all that remains of
the cat when the cat disappears, so too a manifold is what remains
when one starts with a parameterized surface and the space in which
the surface is situated disappears.

A parameterized surface may be decribed as a suitable subset $S \in
\mathbb{R}^3$ together with a smooth bijective map from an open subset
of $\mathbb{R}^2$ to $S$ (which is called a parameterization of the
surface).  Moreover, (pay careful attention because this will turn out
to be the key fact that makes the concept of a manifold possible) one
can describe the same surface using many different parameterizations.
Given two parameterizations $\phi \colon D_1 \subset \mathbb{R}^2 \to
S$ and $\psi \colon D_2 \subset \mathbb{R}^2 \to S$ of the same
portion of surface, the two will be related by reparametrization; that
is to say there exists a smooth bijection $g_{\phi \psi} \colon D_2
\to D_1$ such that $\psi = \phi \circ g_{\phi \psi}$.

By contrast, a manifold may be understood as a ``parameterized set''.
That is to say, a $n$-dimensional manifold $M$ is a set together with
a set of bijective maps from open sets of $\mathbb{R}^n$ to $M$ (which
are called coordiante maps).  As in the case of parameterized
surfaces, a coordinate map need not have the whole of $M$ as its
range.  Moreover, we require that if two coordinate maps $\phi \colon
D_1 \subset \mathbb{R}^2 \to M$ and $\psi \colon D_2 \subset
\mathbb{R}^2 \to M$ describe the same subset of the manifold, then
there exists a smooth bijective map $g_{\phi \psi} \colon D_2 \to D_1$
such that $\psi = \phi \circ g_{\phi \psi}$.  (Such a map is known as
a coordinate transformation map or a transition function.)  To make
this definition completely correct, we require a few more technical
assumptions but, in keeping with the spirit of this expsition, we
shall not discuss them here and instead refer the reader to the entry
notes on the classical definition of a manifold for a careful
definition which takes technicalities into account.

As is obvious from the definition, if we make the further assumption
that the objects of our set be points of Euclidean space and that the
correspondence between points and pairs of numbers be continuous, we
recover the definition of parameterized surface.  However, we choose
to refrain from making any asumption about the nature of the elements
of our set.  This freedom is exactly what allows us to regard two
surfaces with the same intrinsic geometry as the same manifold --- to
obtain one surface, we specify one mapping of our set into Euclidean
space one way and to obtain the other surface we specify a different
mapping.

Objects other than surfaces can be manifolds.  Most obviously, the
Euclidean plane itself is a manifold since its points can be described
by pairs of real numbers according to various coordiante systems
(Cartesian coordinates, polar coordinates, etc.)  Thus, the concept of
manifold fulfills the desire of its inventors that Euclidean space and
surfaces be of the same ontological status.

Less obviously and more curiously, the set of functions which satisfy
the differential equation 
 $$f''(t) + 2 f'(t) + 3 f(t) = 0$$
is also a two-dimensional manifold!  The reason is that we can specify
a solution of this equation uniquely by giving the values of $f$ and
$f'$ at a particular value of $t$ and, by the theorem on continuous
dependence on initial conditions, $f(t_1)$ and $f'(t_1)$ can be
expressed as a continuous functions of $f(t_2)$ and $f'(t_2)$ for any
two numbers $t_1, t_2 \in \mathbb{R}$.

Another example of a manifold is the group of affine transforms of the
line.  Recall (or look back at the section on affine geometry above!)
that an affine transformation is of the form $x \mapsto ax + b$.
Hence, such a transform is specified by giving the two real numbers
$a$ and $b$.  A group like this one which also happens to be a
manifold is called a Lie group.  The study of Lie groups forms an
important branch of group theory and is of relevance to other branches
of mathematics.

Because of examples like the two just exhibited, it has been possible
to apply the techniques of differential geometry in some rather
unlikely settings.  Once geometric notions like tangent spaces and
curvature have been defined for manifolds (we shall indicate how this
is done in the next section) then one can speak of such things as the
tangent space to the set of solutions of a differential equation or
the curvature of a group.  While this may sound like a parlor stunt to
demonstrate the generality of our definitions, it is more than that.
By applying the techniques of differential geometry in such unlikely
settings, mathematicians have been able to win insights and prove
results about differential equations, groups and other mathematical
objects which otherwise seemed intractable.

Before moving on to the next section, it might be worth pointing out
that, when speaking of manifolds, it is customary to refer to the
elements of the set as points and the real numbers which label them as
points.  The reader should keep in mind that this is merely customary
terminology which derives from thinking of manifolds as a
generalization of parameterized surfaces and should no way be
understood as suggesting that the elements of the set which comprise
our manifold resemble points of a surface.  As the examples show, they
may be functions, transformations, or other mathematical objects.

\subsection{Structures on manifolds}

In order to discuss such geometric notions as angles and lengths and
perform interesting geometric constructions, one needs to equip one's
manifold with suitable structures.  In classical differential
geometry, these structures were provided by the ambient space, but now
that the ambient space has disappeared, they must put in by hand.

In placing structures on a manifold, the notion of reparameterization
or change of coordinates which was built into the definition of
manifold plays a crucial role.  To explain how the process of imposing
structures proceeds, let us start with a simple example --- tangent
vectors on a manifold.  In classical differential geometry, a tangent
vector ${\bf v}$ to a surface $S$ at a point ${\bf p} \in S$ is simply
a vector in Euclidean space whose direction happens to be tangent to
the surface at the point ${\bf p}$.  We could represent it graphically
an arrow with its tail at ${\bf p}$ which points along a tangent to
the surface.

Of course, this description won't do if we don't have an ambient space
 in which to draw our arrow.  Therefore, we need to look for a
 different description of our tangent vector.  One possibility is to
 consider a description of the vector in terms of its components.
 However, to define the components of a vector, one needs a basis.  If
 our surface is specified parametrically, there is a natural choice of
 basis vectors, namely 
 \[\begin{pmatrix} \displaystyle \frac{\partial x}{\partial s}
 & \displaystyle \frac{\partial y}{\partial s} & \displaystyle \frac{\partial z}{\partial s}
 \end{pmatrix}\] 
and 
 \[\begin{pmatrix} \displaystyle \frac{\partial x}{\partial t} &
 \displaystyle \frac{\partial y}{\partial t} & \displaystyle \frac{\partial z}{\partial t}
 \end{pmatrix}.\]

Of course, if we choose a different parameterization, we obtain a
different pair of basis vectors.  However, we can express one pair in
terms of the other pair using the chain rule.  Thus, if a tangent
vector ${\bf v}$ has components $\begin{pmatrix} v_1 & v_2
\end{pmatrix}$ with respect to the basis corresponding to the
parameterization in terms of parameters $s$ and $t$, it will have
components $\begin{pmatrix} {v'}_1 & {v'}_2 \end{pmatrix}$ with
respect to the basis corresponding to the parameterization in terms
of parameters $s'$ and $t'$, where 
 \[\begin{pmatrix} {v'}_1 & {v'}_2 \end{pmatrix} = \begin{pmatrix} v_1
 & v_2 \end{pmatrix} \begin{pmatrix} \displaystyle \frac{\partial s}{\partial s'} &
 \displaystyle \frac{\partial s}{\partial t'} \\ & \\ \displaystyle \frac{\partial t}{\partial s'} &
 \displaystyle \frac{\partial t}{\partial t'} \end{pmatrix}.\]

These observations form the foundation of the definition of a tangent
vector to a manifold.  To define a tangent vector ${\bf v}$ to a
two-dimensional manifold $M$ at a point ${\bf p}$, we shall associate
a pair of numbers $\begin{pmatrix} v_1 & v_2 \end{pmatrix}$ to every
coordinate system which describes ${\bf p}$.  The only restriction we
impose is that, given two coordinate systems, the pairs of numbers to
be associated to these systems be related by the transform  
 \[\begin{pmatrix} {v'}_1 & {v'}_2 \end{pmatrix} = \begin{pmatrix} v_1
 & v_2 \end{pmatrix} \begin{pmatrix} \displaystyle \frac{\partial s}{\partial s'} &
 \displaystyle \frac{\partial s}{\partial t'} \\ & \\ \displaystyle \frac{\partial t}{\partial s'} &
 \displaystyle \frac{\partial t}{\partial t'} \end{pmatrix}.\] 

To define other structures on the manifold, we can follow a similar recipe:  
\begin{enumerate}
\item  Identify what sort of mathematical object will represent the
structure in a coordinate system.  In the case of a tangent vector,
this was an $n$-tuplet of real numbers.
\item  Identify how the this object is to transform under changes of
coordinate system.
\item  Define the structure as an assignment of mathematical objects
of the type  identified in item (1) to coordinate systems in such a
way that the objects assigned to two coordinate systems are related by
the transform identified in item (2).
\end{enumerate}

Using this procedure, one can define all sorts of structures on
manifolds, of which we shall consider only two more examples here.
One example is the vector field.  A vector field may be defined as the
assignment of a vector to every point in the manifold.  Given a
coordinate system, we may specify such an entity by assigning an
$n$-tuple of numbers to every point.  In other words, in a coordinate
system our vector field is represented by an $n$-tuple of functions.
Upon making a change of coordinates, these $n$-tuples change according
to the law presented earlier.

The second example is the metric field.  Recall that in our discussion
of intrinsic geometry, we considered measuring lengths and angles
along the surface.  Now, in Euclidean geometry, we may define the
notions of length and angle in terms of an inner product.  A metric
field may be described as the assignment of an inner product for
tangent vectors to every point of the manifold.  Given a basis, an
inner product can be described by a symmetric, positive-definite
matrix.  Hence, to define the notion of metric field, we will consider
the assignment of a symmetric, positive definite matrix of functions
to every coordinate system in such a way that the matrices assigned to
two coordinate systems are related according to the transformation law
for inner products under a change of basis.

Once we choose a metric field on a two-dimensional manifold, it
becomes possible to study its intrinsic geometry as defined in the
last section.

\subsection{Sheaves and bundles}

In order to understand the toatality of all possible tangents vectors,
metrics, etc. one typically collects them into larger structures
called sheaves and bundles.  To understand how such constructions
proceed, we will start by examining the fundamental example of the
tangent bundle.

Last section, we desribed how a tangent vector is described in an
intrinsic manner.  Suppose we now want to consider the totality of all
tangent vectors to a manifold.  We could start by picking a point of
the manifold and considering all tangent vectors based at that point.
Since it is meaningful to make linear combinations of tangent vectors
with the same basepoint, the set of all tangent vectors with a common
basepoint forms a vector space called the tangent space to the
manifold at that point.  As we saw earlier, vector spaces are
manifolds because one can impose coordinates on a vector space by
choosing a basis.

So our problem of describing the toality of all tangent vectors
reduces to the problem of describing the totality of all tangent
spaces.  We claim that they form a manifold.  Basically, the reason
for this is that, to specify a tangent vector, we could give the
coordinates of its basepoint with respect to a coordinate system on
the manifold and specify which vector by its coordnates with respect
to a basis for the tangent space at that point.  This manifold whose
points are tangent vectors to a certain manifold is known as the
tangent bundle of the orginal manifold.


\subsection{Back to Erlangen}

When written, this entry will show how differential geometry may be
understood as the study of invariants of structures on manifolds under
the group of diffeomorphisms.  In terms of this definition, we shall
introduce such geometries as Riemannian geometry, conformal geometry,
Kahler geometry, symplectic geoemtry, contact geometry, teleparallel
geometry, gauge geometry, etc. much as Euclidean, affine, and
projective geometry were introduced above.

\subsection{Differential invariants and local differential geometry}

When written, this section will give the reader the flavor of the nuts and bolts of constructing the invariants which are supposed to describe geometrical objects according to the Klein's principles.  We shall mention such topics as Christoffel's theorem and the Bianchi identitites and give some idea of the sort of topics which one considers in local differential geometry.

\subsection{Global differential geometry}

When written, this section will describe the Gauss-Bonnet theorem and some of its modern descendants such as characteristic classes and index theorems.

\subsection{The algebraic viewpoint}

When written, this entry outline how one may take the algebra of functions of a manifold as a starting point and see that differential geometric notions correspond to algebraic constructions.  In fact, one may reformulate differential geometry as the study of invariants of an algebra under the action of a group of automorphisms.  Using this even more general definition, one arrives at such novel topics as the differential geometry of finite-dimensional algebras and non-commutative geometry.

\subsection{Infinite-Dimensional Differential Geometry}

\section{Differential geometry on PlanetMath}

When written, this section will include links to entries on differential geometry on PlanetMath.
%%%%%
%%%%%
\end{document}
