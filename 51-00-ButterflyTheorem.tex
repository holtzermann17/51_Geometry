\documentclass[12pt]{article}
\usepackage{pmmeta}
\pmcanonicalname{ButterflyTheorem}
\pmcreated{2013-03-22 13:10:06}
\pmmodified{2013-03-22 13:10:06}
\pmowner{giri}{919}
\pmmodifier{giri}{919}
\pmtitle{butterfly theorem}
\pmrecord{8}{33612}
\pmprivacy{1}
\pmauthor{giri}{919}
\pmtype{Theorem}
\pmcomment{trigger rebuild}
\pmclassification{msc}{51-00}

% this is the default PlanetMath preamble.  as your knowledge
% of TeX increases, you will probably want to edit this, but
% it should be fine as is for beginners.

% almost certainly you want these
\usepackage{amssymb}
\usepackage{amsmath}
\usepackage{amsfonts}

% used for TeXing text within eps files
%\usepackage{psfrag}
% need this for including graphics (\includegraphics)
\usepackage{graphicx}
% for neatly defining theorems and propositions
%\usepackage{amsthm}
% making logically defined graphics
%%%\usepackage{xypic}

% there are many more packages, add them here as you need them

% define commands here
\begin{document}
\PMlinkescapeword{name}
\PMlinkescapeword{cuts}
Let $M$ be the midpoint of a chord $PQ$ of a circle, through which two other chords $AB$ and $CD$ are drawn. If $AD$ intersects $PQ$ at $X$ and
$CB$ intersects $PQ$ at  $Y$,then $M$ is also the midpoint of $XY.$

\begin{center}
\includegraphics{but.eps}
\end{center}

The theorem gets its name from the shape of the figure, which resembles a butterfly.
%%%%%
%%%%%
\end{document}
