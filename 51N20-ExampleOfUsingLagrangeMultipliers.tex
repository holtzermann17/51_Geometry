\documentclass[12pt]{article}
\usepackage{pmmeta}
\pmcanonicalname{ExampleOfUsingLagrangeMultipliers}
\pmcreated{2013-03-22 18:48:12}
\pmmodified{2013-03-22 18:48:12}
\pmowner{pahio}{2872}
\pmmodifier{pahio}{2872}
\pmtitle{example of using Lagrange multipliers}
\pmrecord{7}{41604}
\pmprivacy{1}
\pmauthor{pahio}{2872}
\pmtype{Example}
\pmcomment{trigger rebuild}
\pmclassification{msc}{51N20}
\pmclassification{msc}{26B10}
\pmsynonym{example of Lagrange multipliers}{ExampleOfUsingLagrangeMultipliers}
%\pmkeywords{Lagrange multiplier}
\pmrelated{ParallelismOfTwoPlanes}
\pmrelated{ExampleNeedingTwoLagrangeMultipliers}

% this is the default PlanetMath preamble.  as your knowledge
% of TeX increases, you will probably want to edit this, but
% it should be fine as is for beginners.

% almost certainly you want these
\usepackage{amssymb}
\usepackage{amsmath}
\usepackage{amsfonts}

% used for TeXing text within eps files
%\usepackage{psfrag}
% need this for including graphics (\includegraphics)
%\usepackage{graphicx}
% for neatly defining theorems and propositions
 \usepackage{amsthm}
% making logically defined graphics
%%%\usepackage{xypic}

% there are many more packages, add them here as you need them

% define commands here

\theoremstyle{definition}
\newtheorem*{thmplain}{Theorem}

\begin{document}
One \PMlinkescapetext{way} to determine the perpendicular distance of the parallel planes
$$Ax+By+Cz+D \;=\; 0 \quad \mbox{and} \quad Ax+By+Cz+E \;=\; 0$$
is to use the Lagrange multiplier method.\, In this case we may to minimise the Euclidean distance of a point 
\,$(x,\,y,\,z)$\, of the former plane to a (fixed) point \,$(x_0,\,y_0,\,z_0)$\, of the latter plane.\\

Thus we have the equation \,$Ax_0+By_0+Cz_0+E \,=\, 0$\, which we can subtract from the first plane equation, getting
\begin{align}
g \;:=\; A(x-x_0)+B(y-y_0)+C(z-x_0)+D-E \;=\; 0.
\end{align}
This is the (only) constraint equation for minimising the \PMlinkname{square}{SquareOfANumber}
\begin{align}
f \;:=\; (x-x_0)^2+(y-y_0)^2+(z-x_0)^2
\end{align}
of the distance of the points.

The polynomial functions $f$ and $g$ satisfy the differentiability requirements.\, Accordingly, we can find the minimising point\, $(x,\,y,\,z)$\, by considering the system of equations formed by (1) and
\begin{align}
\begin{cases}
\frac{\partial f}{\partial x}+\lambda\frac{\partial g}{\partial x} 
\;\equiv\; 2(x-x_0)+\lambda A \;=\; 0,\\
\frac{\partial f}{\partial y}+\lambda\frac{\partial g}{\partial y} 
\;\equiv\; 2(y-y_0)+\lambda B \;=\; 0,\\
\frac{\partial f}{\partial z}+\lambda\frac{\partial g}{\partial z} 
\;\equiv\; 2(z-z_0)+\lambda C \;=\; 0.
\end{cases}
\end{align}
We solve from (3) the differences
$$x-x_0 \;=\; -\frac{A\lambda}{2}, \quad y-y_0 \;=\; -\frac{B\lambda}{2}, \quad z-z_0 \;=\; -\frac{C\lambda}{2}$$
and set them into (1).\, It then yields the value 
$$\lambda \;=\; \frac{2(D-E)}{A^2+B^2+C^2}$$
of the Lagrange multiplier, which we substitute into the preceding three equations obtaining
$$x-x_0 \;=\; \frac{A(D-E)}{A^2+B^2+C^2}, \quad y-y_0 \;=\; \frac{B(D-E)}{A^2+B^2+C^2}, 
\quad z-z_0 \;=\; \frac{C(D-E)}{A^2+B^2+C^2}.$$
These values give the minimal distance when put into the expression of $\sqrt{f}$:
$$d \;=\; \sqrt{\frac{(D-E)^2(A^2+B^2+C^2)}{(A^2+B^2+C^2)^2}}.$$
Hence we have gotten the distance \PMlinkescapetext{formula}
$$d \;=\; \frac{|D-E|}{\sqrt{A^2+B^2+C^2}}.$$


%%%%%
%%%%%
\end{document}
