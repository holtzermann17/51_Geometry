\documentclass[12pt]{article}
\usepackage{pmmeta}
\pmcanonicalname{CenterNormalAndCenterNormalPlaneAsLoci}
\pmcreated{2013-03-22 18:48:51}
\pmmodified{2013-03-22 18:48:51}
\pmowner{pahio}{2872}
\pmmodifier{pahio}{2872}
\pmtitle{center normal and center normal plane as loci}
\pmrecord{12}{41616}
\pmprivacy{1}
\pmauthor{pahio}{2872}
\pmtype{Theorem}
\pmcomment{trigger rebuild}
\pmclassification{msc}{51M15}
\pmclassification{msc}{51N05}
\pmclassification{msc}{51N20}
\pmsynonym{center normal as locus}{CenterNormalAndCenterNormalPlaneAsLoci}
\pmsynonym{center normal plane as locus}{CenterNormalAndCenterNormalPlaneAsLoci}
\pmrelated{SAS}
\pmrelated{CircumCircle}
\pmrelated{AngleBisectorAsLocus}

% this is the default PlanetMath preamble.  as your knowledge
% of TeX increases, you will probably want to edit this, but
% it should be fine as is for beginners.

% almost certainly you want these
\usepackage{amssymb}
\usepackage{amsmath}
\usepackage{amsfonts}

% used for TeXing text within eps files
%\usepackage{psfrag}
% need this for including graphics (\includegraphics)
%\usepackage{graphicx}
% for neatly defining theorems and propositions
 \usepackage{amsthm}
% making logically defined graphics
%%%\usepackage{xypic}
\usepackage{pstricks}
\usepackage{pst-plot}

% there are many more packages, add them here as you need them

% define commands here

\theoremstyle{definition}
\newtheorem*{thmplain}{Theorem}

\begin{document}
\textbf{Theorem 1.}\, In the Euclidean plane, the center normal of a line segment is the locus of the points which are equidistant from the both end points of the segment.\\

{\em Proof.}\, Let $A$ and $B$ be arbitrary given distinct points.
\begin{center}
\begin{pspicture}(-3,-2)(3,3)
\pspolygon(-2,-2)(0,2)(2,-2)
\psline(-1.15,0.1)(-0.85,-0.1)
\psline(1.15,0.1)(0.85,-0.1)
\psline(-1.15,0)(-0.85,-0.2)
\psline(1.15,0)(0.85,-0.2)
\psline(0,-2)(0,2)
\rput[r](-2.2,-2){$A$}
\rput[l](2.2,-2){$B$}
\rput[b](0,2.2){$P$}
\rput[a](0,-2.3){$D$}
\psarc(0,2){0.35}{270}{300}
\psarc(0,2){0.42}{270}{300}
\psarc(0,2){0.47}{240}{270}
\psarc(0,2){0.54}{240}{270}
\psarc(0,-0.2){0.1}{270}{480}
\psarc(0,-0.4){0.1}{90}{300}
\end{pspicture}
\end{center}
 $1^\circ.$\; Let $P$ be a point equidistant from $A$ and $B$.\, If\, $P \in AB$,\, then $P$ is trivially on the center normal of $AB$.\, Thus suppose that\, $P \not\in AB$.\, In the triangle $PAB$, let the angle bisector of $\angle P$ intersect the \PMlinkname{side}{Triangle} $AB$ in the point $D$.\, Then we have
$$\Delta PDA \;\cong\; \Delta PDB \quad \mbox{(SAS)},$$
whence
$$\angle PDA \;=\; \angle PDB \;=\; 90^\circ, \quad DA \;=\; DB.$$
Consequently, the point $P$ is always on the center normal of $AB$.


\begin{center}
\begin{pspicture}(-3,-2)(3,3)
\pspolygon(-2,-2)(0,2)(2,-2)
\psline(-1,-1.85)(-1,-2.15)
\psline(+1,-1.85)(+1,-2.15)
\psline(-1.07,-1.85)(-1.07,-2.15)
\psline(+1.07,-1.85)(+1.07,-2.15)
\psline(0,-2)(0,2)

\psline(-0.22,-1.8)(0.22,-1.8)
\psline(-0.22,-2)(-0.22,-1.8)
\psline(+0.22,-2)(+0.22,-1.8)
\rput[r](-2.2,-2){$A$}
\rput[l](2.2,-2){$B$}
\rput[b](0,2.2){$Q$}
\rput[a](0,-2.3){$D$}
\psarc(0,-0.2){0.1}{270}{480}
\psarc(0,-0.4){0.1}{90}{300}
\end{pspicture}
\end{center}


$2^\circ.$\; Let $Q$ be any point on the center normal and $D$ the midpoint of the line segment $AB$.\, We can assume that\, $Q \neq D$.\, Then we have
$$\Delta QDA \;\cong\; \Delta QDB \quad \mbox{(SAS)},$$
implying that
$$QA \;=\; QB.$$
Thus $Q$ is equidistant from $A$ and $B$.\\


\textbf{Theorem 2.}\, In the Euclidean space, the center normal plane of a line segment is the locus of the points which are equidistant from the both end points of the segment.\\

{\em Proof.}\, Change ``center normal'' in the preceding proof to ``center normal plane''.
%%%%%
%%%%%
\end{document}
