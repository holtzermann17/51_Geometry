\documentclass[12pt]{article}
\usepackage{pmmeta}
\pmcanonicalname{ScaleneTriangle}
\pmcreated{2013-03-22 16:05:25}
\pmmodified{2013-03-22 16:05:25}
\pmowner{Wkbj79}{1863}
\pmmodifier{Wkbj79}{1863}
\pmtitle{scalene triangle}
\pmrecord{6}{38151}
\pmprivacy{1}
\pmauthor{Wkbj79}{1863}
\pmtype{Definition}
\pmcomment{trigger rebuild}
\pmclassification{msc}{51-00}
\pmsynonym{scalene}{ScaleneTriangle}
\pmrelated{Triangle}

\endmetadata

\usepackage{amssymb}
\usepackage{amsmath}
\usepackage{amsfonts}
\usepackage{graphicx}
%%\usepackage{xypic}

\begin{document}
A triangle in which no two sides have the same \PMlinkescapetext{measure} is called \emph{scalene}.

Below is a comparison of a scalene triangle with an equilateral triangles and some isosceles triangles:

\begin{center}
\includegraphics{trianglebyside}
\end{center}
%%%%%
%%%%%
\end{document}
