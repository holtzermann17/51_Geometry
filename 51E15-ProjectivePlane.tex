\documentclass[12pt]{article}
\usepackage{pmmeta}
\pmcanonicalname{ProjectivePlane}
\pmcreated{2013-03-22 15:11:05}
\pmmodified{2013-03-22 15:11:05}
\pmowner{yark}{2760}
\pmmodifier{yark}{2760}
\pmtitle{projective plane}
\pmrecord{13}{36940}
\pmprivacy{1}
\pmauthor{yark}{2760}
\pmtype{Topic}
\pmcomment{trigger rebuild}
\pmclassification{msc}{51E15}
\pmclassification{msc}{51N15}
\pmclassification{msc}{05B25}
%\pmkeywords{incidence}
%\pmkeywords{design}
%\pmkeywords{geometry}
\pmrelated{FiniteProjectivePlane4}
\pmrelated{IncidenceStructures}
\pmrelated{Geometry}
\pmrelated{LinearSpace2}
\pmrelated{AxiomaticProjectiveGeometry}
\pmrelated{LinearSpace3}

\usepackage{amssymb}

% portions from
% makra.sty 1989-2005 by Marijke van Gans

%%%% CHARS %%%%%%%%%%%%%%%%%%%%%%%%%%%%%%%%%%%%%%%%%%%%%%

                        %    code    char  frees  for

\let\Para\S             %    \Para     Ã?Ã?Ã?Ãç   \S \scriptstyle
\let\Pilcrow\P          %    \Pilcrow  Ã?Ã?Ã?Ãö   \P
\mathchardef\pilcrow="227B

\mathchardef\lt="313C   %    \lt       <   <     bra
\mathchardef\gt="313E   %    \gt       >   >     ket

\let\bs\backslash       %    \bs       \
\let\us\_               %    \us       _     \_  ...

%%%% DIACRITICS %%%%%%%%%%%%%%%%%%%%%%%%%%%%%%%%%%%%%%%%%

%let\udot\d             % under-dot (text mode), frees \d
\let\odot\.             % over-dot (text mode),  frees \.
%let\hacek\v            % hacek (text mode),     frees \v
%let\makron\=           % makron (text mode),    frees \=
%let\tilda\~            % tilde (text mode),     frees \~
\let\uml\"              % umlaut (text mode),    frees \"

%def\ij/{i{\kern-.07em}j}
\def\trema#1{\discretionary{-}{#1}{\uml #1}}

%%%% amssymb %%%%%%%%%%%%%%%%%%%%%%%%%%%%%%%%%%%%%%%%%%%%

\let\le\leqslant
\let\ge\geqslant
%let\prece\preceqslant
%let\succe\succeqslant

%%%% USEFUL MISC %%%%%%%%%%%%%%%%%%%%%%%%%%%%%%%%%%%%%%%%

%def\C++{C$^{_{++}}$}

%let\writelog\wlog
%def\wl@g/{{\sc wlog}}
%def\wlog{\@ifnextchar/{\wl@g}{\writelog}}

%def\org#1{\lower1.2pt\hbox{#1}} 
% chem struct formulae: \bs, --- /  \org{C} etc. 

%%%% USEFUL INTERNAL LaTeX STUFF %%%%%%%%%%%%%%%%%%%%%%%%

%let\Ifnextchar=\@ifnextchar
%let\Ifstar=\@ifstar
%def\currsize{\@currsize}

%%%% KERNING, SPACING, BREAKING %%%%%%%%%%%%%%%%%%%%%%%%%

\def\comma{,\,\allowbreak}

%def\qqquad{\hskip3em\relax}
%def\qqqquad{\hskip4em\relax}
%def\qqqqquad{\hskip5em\relax}
%def\qqqqqquad{\hskip6em\relax}
%def\qqqqqqquad{\hskip7em\relax}
%def\qqqqqqqquad{\hskip8em\relax}

%%%% LAYOUT %%%%%%%%%%%%%%%%%%%%%%%%%%%%%%%%%%%%%%%%%%%%%

%%%% COUNTERS %%%%%%%%%%%%%%%%%%%%%%%%%%%%%%%%%%%%%%%%%%%

%let\addtoreset\@addtoreset
%{A}{B} adds A to list of counters reset to 0
% when B is \refstepcounter'ed (see latex.tex)
%
%def\numbernext#1#2{\setcounter{#1}{#2}\addtocounter{#1}{\m@ne}}

%%%% MATH LAYOUT %%%%%%%%%%%%%%%%%%%%%%%%%%%%%%%%%%%%%%%%

\let\D\displaystyle
\let\T\textstyle
\let\S\scriptstyle
\let\SS\scriptscriptstyle

\def\Bset{\mathbb{B}}
\def\Nset{\mathbb{N}}
\def\Rset{\mathbb{R}}
\def\Hset{\mathbb{H}}
\def\Fset{\mathbb{F}}
\def\kset{\mathbb{k}}

% \frac overwrites LaTeX's one (use TeX \over instead)
%def\fraq#1#2{{}^{#1}\!/\!{}_{\,#2}}
\def\frac#1#2{\mathord{\mathchoice%
{\T{#1\over#2}}
{\T{#1\over#2}}
{\S{#1\over#2}}
{\SS{#1\over#2}}}}
%def\half{\frac12}

\mathcode`\<="4268         % < now is \langle, \lt is <
\mathcode`\>="5269         % > now is \rangle, \gt is >

%def\biggg#1{{\hbox{$\left#1\vbox %to20.5\p@{}\right.\n@space$}}}
%def\Biggg#1{{\hbox{$\left#1\vbox %to23.5\p@{}\right.\n@space$}}}

\let\epsi=\varepsilon
\def\omikron{o}

\def\Alpha{{\rm A}}
\def\Beta{{\rm B}}
\def\Epsilon{{\rm E}}
\def\Zeta{{\rm Z}}
\def\Eta{{\rm H}}
\def\Iota{{\rm I}}
\def\Kappa{{\rm K}}
\def\Mu{{\rm M}}
\def\Nu{{\rm N}}
\def\Omikron{{\rm O}}
\def\Rho{{\rm P}}
\def\Tau{{\rm T}}
\def\Ypsilon{{\rm Y}} % differs from \Upsilon
\def\Chi{{\rm X}}

%def\dg{^{\circ}}                   % degrees

%def\1{^{-1}}                       % inverse

\def\*#1{{\bf #1}}                  % boldface e.g. vector
%def\vi{\mathord{\hbox{\bf\i}}}     % boldface vector \i
%def\vj{\mathord{\,\hbox{\bf\j}}}   % boldface vector \j

%def\union{\mathbin\cup}
%def\isect{\mathbin\cap}

\let\so\Longrightarrow
\let\oso\Longleftrightarrow
\let\os\Longleftarrow

% := and :<=>
%def\isdef{\mathrel{\smash{\stackrel{\SS\rm def}{=}}}}
%def\iffdef{\mathrel{\smash{stackrel{\SS\rm def}{\oso}}}}

\def\isdef{\mathrel{\mathop{=}\limits^{\smash{\hbox{\tiny def}}}}}
%def\iffdef{\mathrel{\mathop{\oso}\limits^{\smash{\hbox{\tiny %def}}}}}

%def\tr{\mathop{\rm tr}}            % tr[ace]
%def\ter#1{\mathop{^#1\rm ter}}     % k-ter[minant]

%let\.=\cdot
%let\x=\times                % ><Ã? (direct product)

%def\qed{ ${\S\circ}\!{}^\circ\!{\S\circ}$}
%def\qed{\vrule height 6pt width 6pt depth 0pt}

%def\edots{\mathinner{\mkern1mu
%   \raise7pt\vbox{\kern7pt\hbox{.}}\mkern1mu   %  .shorter
%   \raise4pt\hbox{.}\mkern1mu                  %     .
%   \raise1pt\hbox{.}\mkern1mu}}                %        .
%def\fdots{\mathinner{\mkern1mu
%   \raise7pt\vbox{\kern7pt\hbox{.}}            %   . ~45Ãð
%   \raise4pt\hbox{.}                           %     .
%   \raise1pt\hbox{.}\mkern1mu}}                %       .

\def\mod#1{\allowbreak \mkern 10mu({\rm mod}\,\,#1)}
% redefines TeX's one using less space

%def\int{\intop\displaylimits}
%def\oint{\ointop\displaylimits}

%def\intoi{\int_0^1}
%def\intall{\int_{-\infty}^\infty}

%def\su#1{\mathop{\sum\raise0.7pt\hbox{$\S\!\!\!\!\!#1\,$}}}

%let\frakR\Re
%let\frakI\Im
%def\Re{\mathop{\rm Re}\nolimits}
%def\Im{\mathop{\rm Im}\nolimits}
%def\conj#1{\overline{#1\vphantom1}}
%def\cj#1{\overline{#1\vphantom+}}

%def\forAll{\mathop\forall\limits}
%def\Exists{\mathop\exists\limits}

%%%% PICTURES %%%%%%%%%%%%%%%%%%%%%%%%%%%%%%%%%%%%%%%%%%%

%def\cent{\makebox(0,0)}

%def\node{\circle*4}
%def\nOde{\circle4}

%%%% REFERENCES %%%%%%%%%%%%%%%%%%%%%%%%%%%%%%%%%%%%%%%%%

%def\opcit{[{\it op.\,cit.}]}
\def\bitem#1{\bibitem[#1]{#1}}
\def\name#1{{\sc #1}}
\def\book#1{{\sl #1\/}}
\def\paper#1{``#1''}
\def\mag#1{{\it #1\/}}
\def\vol#1{{\bf #1}}
\def\isbn#1{{\small\tt ISBN\,\,#1}}
\def\seq#1{{\small\tt #1}}
%def\url<{\verb>}
%def\@cite#1#2{[{#1\if@tempswa\ #2\fi}]}

%%%% AD HOC %%%%%%%%%%%%%%%%%%%%%%%%%%%%%%%%%%%%%%%%%%%%%

\def\p/{rojective plane}
\def\pt/{{\sc point}}
\def\ln/{{\sc line}}

\begin{document}
\PMlinkescapeword{class}
\PMlinkescapeword{difference}
\PMlinkescapeword{extension}
\PMlinkescapeword{grid}
\PMlinkescapeword{homogenous}
\PMlinkescapeword{incident}
\PMlinkescapeword{line}
\PMlinkescapeword{lines}
\PMlinkescapeword{opposite}
\PMlinkescapeword{parallel}
\PMlinkescapeword{point}
\PMlinkescapeword{points}
\PMlinkescapeword{represent}
\PMlinkescapeword{variety}
\PMlinkescapeword{word}
\PMlinkescapeword{words}

\subsection*{Projective planes}

A {\bf projective plane} is a plane (in various senses) where not only
%
\begin{itemize}

\item for any two distinct {\sc point}s,
      there is exactly one {\sc line} through both of them

\end{itemize}
%
(as usual, in things we call a ``plane''), but also
%
\begin{itemize}

\item for any two distinct {\sc line}s,
      there is exactly one {\sc point} on both of them

\end{itemize}
%
(in other words, no parallel {\sc line}s). This gives {\bf duality} between
{\sc point}s and {\sc line}s: for any statement there is a corresponding statement
that swaps the words {\sc point} and {\sc line}, and swaps the phrases {\em lies on\/}
and {\em passes through\/} (for which we can use the neutral
{\em is incident with\/}).

A third axiom is commonly added to avoid degenerate cases:
%
\begin{itemize}

\item there exist four points no three of which are collinear.

\end{itemize}
%

Both finite and infinite projective planes exist.

Here's an example, just to show that such things exist: let $S$ be the unit
sphere (the 2-dimensional surface $x^2+y^2+z^2=1$ embedded in $\Rset^3$). Call
every great circle (circle with radius 1 whose centre coincides with that of
the sphere) a {\sc line}, and call every {\em pair of opposite points\/} on
the sphere a {\sc point}. The notion of a {\sc point} lying on a {\sc line} is
well defined here, because each great circle (one of our {\sc line}s) passing
through some point on the sphere also passes through its opposite point
(which together form one of our {\sc point}s). Such a situation where different
flavors of ``point'' are discussed will arise again, and it is the reason
why the entities that act as {\sc point} and {\sc line} of a projective plane are typeset in their
own distinctive way in this entry.

And here's a finite example, the Fano plane: the seven {\sc point}s are labeled
with the residue classes (mod~7) and the seven {\sc line}s are numbered likewise.
A {\sc point} $p$ and a {\sc line} $q$ are incident if and only if the equation $x^2=q-p$ does not
have a solution. Line $q$ is incident with {\sc point}s $q+1$, $q+2$, $q+4$ and
{\sc point} $p$ is incident with {\sc line}s $p-1$, $p-2$, $p-4$. Another way to get the
same plane (but with a quite different numbering): let $\Fset_2$ be the finite
field of order~2 and $\Fset_2^3$ the 3-dimensional vector space over $\Fset$.
The seven non-null vectors label our {\sc point}s, and the {\sc line}s are numbered
likewise. A {\sc point} is incident with a {\sc line} if and only if their dot product is zero
(for example: {\sc point} (0,1,1) lies on {\sc line} (1,1,1) because $0\cdot1
+1\cdot1 + 1\cdot1 = 2 = 0$).

\clearpage
\subsection*{Homogeneous coordinates}

Historically, projective planes were formed by extending ordinary planes.

Let $\mit\Pi^*$ be a plane, a two-dimensional vector space over a field
$\Fset$, where a point can be represented by its Cartesian coordinates $(X,Y)$
with $X$ and $Y\in\Fset$ (when $\Fset=\Rset$, this is the usual
Euclidean plane). To specify a line $\ell$, we can give a linear equation
%%
\begin{equation}\label{px+qy+r}
  pX + qY + r = 0
\end{equation}
%%
satisfied by all the points on $\ell$. The slope of the line is given by
$-p/q$ (if $q\ne0$) which is a single field element. Writing it as a
{\bf ratio} $-p:q$ allows us to define slope for vertical lines as well,
where the ratio is $1:0$. This kind of extension of the field to include
some kind of infinity will be a recurrent theme.

There are many asymmetries between points and lines here. First of all, the
line is thought of as {\em being\/} a set of points (the points on the line)
while a point is not the same thing as the set of lines through that
point. This means we're really interpreting equation~(\ref{px+qy+r}) as
$$
  \ell \;\isdef\; \{ (X,Y) \mid pX+qY+r=0 \}
$$
where $X$ and $Y$ are variables (coordinates of any point $\in\ell$) and $p$,
$q$ and $r$ parameters specifying which line. We can redress this conceptual
imbalance by giving the line a kind of coordinates $[\,p,q,r\,]$. The linear
equation~(\ref{px+qy+r}) now takes on a different interpretation: it is the
statement {\em that\/} the point $(X,Y)$ and the line $[\,p,q,r\,]$ are
incident, with both sets of coordinates on an equal footing.

Secondly, there are three coordinates for the line but only two for the point.
This is caused by the fact that line coordinate triples are not unique; only
the {\em ratio\/} of the coordinates matters. We can define equivalence classes
$$
  [\,p : q : r\,] \;\isdef\; \{ [fp,fq,fr] \mid f\in\Fset,\; f\ne0 \}
$$
of coordinate triples that represent the same line. The $p$, $q$ and $r$ used
to label a $[\,p : q : r\,]$ are of course still just as non-unique (this
is in the same spirit as labeling a residue class or other coset by one of
its elements). One possible convention (for lines with $r\ne0$ at least)
is to choose the representative with $r=1$.

To find more symmetry between points and lines we could first define coordinate
triples for points with that same behaviour, so-called {\bf homogeneous coordinates}
$$
  (x\comma y\comma z) \;\isdef\; (x/z\comma y/z)
$$
which means $(X,Y,1)$ and more generally $(fX,fY,f)$ for $f\ne0$ are new
names for the point $(X,Y)$, and then define the equivalence classes
$$
  (x : y : z) \;\isdef\; \{ (fx,fy,fz) \mid f\in\Fset,\; f\ne0 \}
$$
to formally put all those different names for the same point back into a
single box. This exercise gives the statement
%%
\begin{equation}\label{px+qy+rz}
  px + qy + rz = 0
\end{equation}
%%
that $(x:y:z)$ and $[\,p:q:r\,]$ are incident a pleasing symmetry.

\clearpage
\subsection*{The line and points at infinity}

Thus far, we cannot have any point $(x:y:z)$ where $z=0$ (it does not
correspond to any point $(X,Y)$ in the plane). By contrast, $[\,p:q:r\,]$
can have $r=0$ (for a line through the arbitrary origin of the $XY$ coordinate
frame). For lines only the case $p=q=0$ is missing, whereas $x=y=0$ is
fine for a point. Thus, by trying to make points and lines as similar as
possible we have unearthed their essential difference algebraically.

The geometric difference is that lines can be parallel, and it is easy to
see this is the same difference. For any two points $(x,y,z)$ and
$(x',y',z')$ we can find a line $[\,yz'-zy'\comma zx'-xz'\comma xy'-yx'\,]$
that passes through both, and it is a valid line (first two coordinates not
both zero) if the points are valid ($z$ and $z'$ nonzero) and distinct,
but attempting the dual construction reveals pairs of valid lines that do
not intersect in a valid point.

We now extend the plane $\mit\Pi^*$ to a new kind of plane $\mit\Pi$,
inheriting all the points and lines of $\mit\Pi^*$ as {\sc point}s and {\sc line}s of
$\mit\Pi$ and co-opting additionally
%
\begin{itemize}

\item the new {\sc line} $[\,0 : 0 : 1\,]$ --- note only one new {\sc line} is
      needed as all triples $[\,0\comma 0\comma f]$ (with $f\ne0$)
      fall in this equivalence class, and

\item the new {\sc point}s $(1 : f : 0)$ (comprising all $(x,y,0)$ with nonzero $x$,
      using $f=y/x$), as well as $(0 : 1 : 0)$ (this class has those $(x,y,0)$
      where $x=0$).

\end{itemize}
%
The only ratio excluded for a {\sc point} is $(0:0:0)$ and for a {\sc line} $[\,0:0:0\,]$,
making the situation symmetric. Note all the new {\sc point}s ``lie on'' the new
{\sc line} and none of the old ones do, which can be seen by applying~(\ref{px+qy+rz}).
Any of the old {\sc line}s acquires one of the new {\sc point}s, {\sc line}s that were parallel
get the same new {\sc point} and {\sc line}s that weren't get different new {\sc point}s. This
prevents any pair of {\sc line}s already intersecting in an old {\sc point} intersecting in
a new one as well, and provides any pair that didn't yet intersect a ``place''
where to do so.

The new {\sc point}s are what is shared by parallel lines, so correspond to
{\em directions\/} (pairs of opposite directions, in fact). They are
called {\bf points at infinity} and the new {\sc line} comprising them the
{\bf line at infinity}.

\clearpage
\subsection*{The embedding in $\Fset^3$}

The extension of $\mit\Pi^*$ to $\mit\Pi$ in the previous section has an
immediate geometric interpretation. Interpret every $(x,y,z)$ as a distinct
point in $\Fset^3 \setminus (0,0,0)$. The equivalence classes $(x:y:z)$, or
strictly speaking $(x:y:z)\cup\{(0,0,0)\}$, are now lines through the origin:
%
\begin{itemize}

\item the {\sc point}s of $\mit\Pi$ can be regarded as 1-dimensional subspaces of
      $\Fset^3$.

\end{itemize}
%
We could equate the $[\,p:q:r\,]$ with lines through the origin as well, but it
will turn out to be more convenient to identify them with the planes through
the origin $\perp$ those lines. Those planes consist of precisely those vectors
$(x,y,z) \in \Fset^3$ that are $\perp$ the vector $(p,q,r) \in \Fset^3$.
In this way
%
\begin{itemize}

\item the {\sc line}s of $\mit\Pi$ can be regarded as 2-dimensional subspaces of
      $\Fset^3$.

\end{itemize}
%
The reason this is more convenient is that now, by construction, all the
points (vectors) in the 1-d subspace corresponding to a {\sc point} $(x,y,z)$ are
$\perp$ the vector $(p,q,r)$ if and only if equation~(\ref{px+qy+rz}) holds. In other
words, those points lie inside the 2-d subspace corresponding to the {\sc line}
$[\,p:q:r\,]$ if and only if the {\sc point} is incident with this {\sc line}:
%
\begin{itemize}

\item {\sc point} P is incident with {\sc line} $\ell$ if and only if the line corresponding
      to P $\subset$ the plane corresponding to $\ell$.

\end{itemize}
%
We can collapse back from 3 dimensions to 2 dimensions in various ways.

First, intersect the 1- and 2-dimensional subspaces representing {\sc point}s and
{\sc line}s with the plane in $\Fset^3$ with $z=1$; call this plane $\mit\Pi^*_1$.
%
\begin{itemize}

\item[$\circ$] The {\sc point}s (lines) $(x:y:z)$ with $z\ne 0$ each contain one
               point $(x/z\comma y/z\comma 1)$ in $\mit\Pi^*_1$. The {\sc point}s
               with $z=0$ (lines in $\Fset^3$ parallel to $\mit\Pi^*_1$)
               don't have any point in $\mit\Pi^*_1$.

\item[$\circ$] The {\sc line}s (planes) $[\,p:q:r\,]$ with $p$ and $q$ not both zero
               intersect $\mit\Pi^*_1$ in a line with equation~(\ref{px+qy+r}).
               The only missing {\sc line} (plane) is the one through the origin
               parallel to $\mit\Pi^*_1$, which contains all the missing {\sc point}s.

\end{itemize}
%
So the plane $\mit\Pi^*_1$ is exactly the plane $\mit\Pi^*$ we started off
with, with an extra third coordinate 1 tacked on at the end.

Alternatively, intersect the 1- and 2-dimensional subspaces representing
{\sc point}s and {\sc line}s with the sphere $x^2+y^2+z^2=1$.
%
\begin{itemize}

\item[$\circ$] The intersection of a {\sc point} (line through the origin) with this
               sphere is two opposite points.

\item[$\circ$] The intersection of a {\sc line} (plane through the origin) with the
               sphere is a great circle.

\end{itemize}
%
When $\Fset=\Rset$, this is exactly the first example in this article.

The second example there (the finite one, the Fano plane) can be seen as
the embedding in $\Fset^3$ when $\Fset=\Fset_2$. Its other representation
showed it also has a cyclic symmetry modulo 7 (many more than one, in fact
--- its automorphism group has order 168).

A health warning is in order: the coordinatisation here gives the
spurious idea there is some special relation between the {\sc point} $(a:b:c)$
and the {\sc line} $[\,a:b:c\,]$. In the sphere in $\Rset^3$ for example, such
a {\sc point} plays the r\^ole of poles relative to the {\sc line} as equator. The whole
point of projective geometry however, which we will not pursue further in
this entry, is that only incidence between {\sc point}s and {\sc line}s is considered
meaningful, and metric considerations (distances and angles) are ignored.
If we redo the $\Rset^3$ example with any arbitrary set of three independent
vectors as basis, we get projective planes all isomorphic with each other as far as
incidence is concerned, but the pairs $(a:b:c)$ and $[\,a:b:c\,]$ that
end up with the ``same'' coordinates are different in each version.

In the finite example too, if we choose a basis different from (1,0,0),
(0,1,0), (0,0,1) we can find the same plane in many different guises:
there are 7 ways to choose the first basis vector, 6 ways to find a
second distinct from the first, and 4 ways for a third not in the plane
of the first two, and $7\cdot6\cdot4=168$.

\clearpage
\subsection*{Classical finite planes}

The constructions above can be carried out for $\Fset^2=\Rset^2$ where they
extend the usual Euclidean plane to ``the'' projective plane, but equally
well for any other field, such as {\bf finite fields} (aka Galois fields).
Such fields have $q=p^d$ elements for $p$ any prime and $d$ any positive
integer.

The $\mit\Pi^*$ that takes the place of the Euclidean plane is now called a
(finite) {\bf affine plane}. For $d=1$ i.e.\ $q=p$ it is still fairly easy
to visualise. Now the field is arithmetic (mod~$p$) and the affine plane
is a grid of $q\times q$ points, with lines connecting them at all possible
integer ratio slopes (you need to wrap top to bottom and left to right,
in modulo fashion, if you attempt to draw it). There are $q$ slopes $f:1$
(in a field, every nonzero element has a multiplicative inverse so every $a:b$
is some $ab^{-1}:1$), and one slope $1:0$ for vertical lines. For each of the
$q+1$ slopes there are $q$ parallel lines, making $q^2+q$ lines in all.

The (finite) {\bf projective plane} $\mit\Pi$ adds $q+1$ {\sc point}s at infinity
(one for each slope, each direction shared by a bunch of parallel lines)
to the $q^2$ points of the grid, $q^2+q+1$ {\sc point}s in all. And it adds one
{\sc line} at infinity to the $q^2+q$ lines we had, making $q^2+q+1$ {\sc line}s as well.
So
%
\begin{itemize}

\item the plane has $q^2+q+1$ {\sc point}s and $q^2+q+1$ {\sc line}s.

\end{itemize}
%
This is also evident from the embedding in $\Fset^3 \setminus \{(0,0,0)\}$
where each time $q-1$ non-(0,0,0) points lie on the same {\sc point} (line through
the origin), and $(q^3-1)/(q-1) = q^2+q+1$. Each {\sc line} (plane through the
origin) is $\perp$ such a line through the origin, so the numbers are the
same. We also have that
%
\begin{itemize}

\item every {\sc line} is incident with $q+1$ {\sc point}s; every {\sc point} with $q+1$ {\sc line}s

\end{itemize}
%
because each {\sc line} (plane through the origin) contains $q^2-1$ non-(0,0,0)
points, and $(q^2-1)/(q-1) = q+1$. The simplest way to see the number the
other way round is also $q+1$ is calling {\sc point}s {\sc line}s and vice versa and
embedding the thing in another $\Fset^3$; the incidence relation
(\ref{px+qy+rz}) is unchanged under this swap.

The field-based planes constructed here are the {\bf classical} planes,
also called {\bf Desarguesian} because Desargues's theorem holds in them,
and {\bf Pappian} because Pappus's theorem holds.
$\Pi\acute\alpha\pi\pi\omikron\varsigma$ (Pappus) was a 4th century Alexandrine
mathematician and Desargues a 17th century French one.

It can be shown that Pappus's theorem holds in precisely those planes
constructed in this way with $\Fset$ a field (commutative division ring),
whereas Desargues's theorem holds whenever $\Fset$ is a skew field (division
ring). For finite planes both conditions are the same by Wedderburn's theorem,
so Desarguesian and Pappian are synonyms. For infinite planes you can have
Desarguesian non-Pappian ones, for instance if $\Fset$ is taken to be the
quaternions.

The question arises what other algebraic structures other than
(skew) fields can produce projective planes. See also the 
{\bf\PMlinkid{finite projective planes entry}{6943}}.

\clearpage
\subsection*{Projective spaces}

The same construction with homogeneous coordinates can be carried out in different
numbers of dimensions. This extends $\Fset^1$ to the {\bf projective line} with
for a finite field $q+1$ elements, labeled by $\Fset\cup\{\infty\}$ where
$\infty$ is any item not in $\Fset$. It likewise extends $\Fset^n$ for
$n\gt2$ to higher {\bf projective spaces} with $(q^{n+1}-1)/(q-1)$ elements.

For $n\ne2$ however these classical constructions give the only possible
projective spaces; there is nothing corresponding to the wild variety of
non-classical projective planes we find for $n=2$.
%%%%%
%%%%%
\end{document}
