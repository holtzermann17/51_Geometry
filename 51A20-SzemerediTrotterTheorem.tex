\documentclass[12pt]{article}
\usepackage{pmmeta}
\pmcanonicalname{SzemerediTrotterTheorem}
\pmcreated{2013-03-22 13:21:30}
\pmmodified{2013-03-22 13:21:30}
\pmowner{bbukh}{348}
\pmmodifier{bbukh}{348}
\pmtitle{Szemer\'edi-Trotter theorem}
\pmrecord{7}{33879}
\pmprivacy{1}
\pmauthor{bbukh}{348}
\pmtype{Theorem}
\pmcomment{trigger rebuild}
\pmclassification{msc}{51A20}
\pmclassification{msc}{05C10}

\usepackage{amssymb}
\usepackage{amsmath}
\usepackage{amsfonts}

\DeclareMathOperator{\crn}{cr}
\begin{document}
The number of incidences of a set of $n$ points and a set of $m$ lines in the real plane $\mathbb{R}^2$ is
\begin{equation*}
I=O(n+m+(nm)^{\frac{2}{3}}).
\end{equation*}

\textbf{Proof}. Let's consider the points as vertices of a graph, and connect two vertices by an edge if they are adjacent on some line. Then the number of edges is $e=I-m$. If $e<4n$ then we are done. If $e \geq 4n$ then by crossing lemma
\begin{equation*}
m^2 \geq \crn(G) \geq \frac{1}{64}\frac{(I-m)^3}{n^2},
\end{equation*}
and the theorem follows.

Recently, T\'oth\cite{cite:toth_szemtrotcomplex} extended the theorem to the complex plane $\mathbb{C}^2$. The proof is difficult.

\begin{thebibliography}{1}

\bibitem{cite:toth_szemtrotcomplex}
Csaba~D. T{\'o}th.
\newblock The {Szemer\'edi}-{Trotter} theorem in the complex plane.
\newblock \PMlinkexternal{arXiv:CO/0305283}{http://www.arxiv.org/abs/math.CO/0305283},
  May 2003.

\end{thebibliography}

%@UNPUBLISHED{cite:toth_szemtrotcomplex,
% author    = {Csaba D. T{\'o}th},
% title     = {The {Szemer\'edi}-{Trotter} Theorem in the Complex Plane},
% month     = may,
% year      = 2003,
% note      = {\href{arXiv:CO/0305283}{http://www.arxiv.org/abs/math.CO/0305283}}
%}
%%%%%
%%%%%
\end{document}
