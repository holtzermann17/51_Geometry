\documentclass[12pt]{article}
\usepackage{pmmeta}
\pmcanonicalname{Pentagram}
\pmcreated{2013-03-22 17:08:52}
\pmmodified{2013-03-22 17:08:52}
\pmowner{pahio}{2872}
\pmmodifier{pahio}{2872}
\pmtitle{pentagram}
\pmrecord{11}{39456}
\pmprivacy{1}
\pmauthor{pahio}{2872}
\pmtype{Definition}
\pmcomment{trigger rebuild}
\pmclassification{msc}{51F99}

% this is the default PlanetMath preamble.  as your knowledge
% of TeX increases, you will probably want to edit this, but
% it should be fine as is for beginners.

% almost certainly you want these
\usepackage{amssymb}
\usepackage{amsmath}
\usepackage{amsfonts}

% used for TeXing text within eps files
%\usepackage{psfrag}
% need this for including graphics (\includegraphics)
%\usepackage{graphicx}
% for neatly defining theorems and propositions
 \usepackage{amsthm}
% making logically defined graphics
%%%\usepackage{xypic}

% there are many more packages, add them here as you need them

\usepackage{pstricks}

% define commands here

\theoremstyle{definition}
\newtheorem*{thmplain}{Theorem}

\begin{document}
\PMlinkescapeword{regular}

A {\em pentagram} is the figure formed by the five diagonals of a \PMlinkname{regular}{RegularPolygon} pentagon.\, The name comes from the Greek $\pi\varepsilon\nu\tau\alpha\gamma\varrho\alpha\mu\mu o\nu$.  Its \PMlinkescapetext{roots} are $\pi\varepsilon\nu\tau\varepsilon$ `five' and $\gamma\varrho\alpha\mu\mu o\varsigma$ `stoke, \PMlinkescapetext{line}'.\,  These Greek \PMlinkescapetext{words} are transliterated as {\em pentagrammon}, {\em pente}, and {\em grammos}, respectively.

In the picture below, a regular pentagon is drawn dashed in black, the pentagram is drawn in blue, and the regular pentagon in the middle of the pentagram is shaded in cyan.

\begin{center}
\begin{pspicture}(-3,-3)(3,3)
\pspolygon[linestyle=dashed](0,3)(-2.853,0.927)(-1.763,-2.427)(1.763,-2.427)(2.853,0.927)
\pspolygon[linecolor=blue](0,3)(-1.763,-2.427)(2.853,0.927)(-2.853,0.927)(1.763,-2.427)
\psset{fillstyle=solid}
\pspolygon[linecolor=blue,fillcolor=cyan](0.67354,0.927)(-0.67354,0.927)(-1.09,-0.354)(0,-1.146)(1.09,-0.354)
\psdots(0,3)(-2.853,0.927)(-1.763,-2.427)(1.763,-2.427)(2.853,0.927)
\end{pspicture}
\end{center}

All pentagrams are similar.\, If the length of the pentagon diagonal is $d$, then each side of the small pentagon in the middle of the pentagram is of length\, $(\sqrt{5}-2)d$.
%%%%%
%%%%%
\end{document}
