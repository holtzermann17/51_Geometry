\documentclass[12pt]{article}
\usepackage{pmmeta}
\pmcanonicalname{CPCTC}
\pmcreated{2013-03-22 17:12:09}
\pmmodified{2013-03-22 17:12:09}
\pmowner{Wkbj79}{1863}
\pmmodifier{Wkbj79}{1863}
\pmtitle{CPCTC}
\pmrecord{7}{39522}
\pmprivacy{1}
\pmauthor{Wkbj79}{1863}
\pmtype{Definition}
\pmcomment{trigger rebuild}
\pmclassification{msc}{51-00}
\pmsynonym{corresponding parts of congruent triangles are congruent}{CPCTC}

\endmetadata

\usepackage{amssymb}
\usepackage{amsmath}
\usepackage{amsfonts}
\usepackage{pstricks}
\usepackage{psfrag}
\usepackage{graphicx}
\usepackage{amsthm}
%%\usepackage{xypic}

\begin{document}
``CPCTC'' is an acronym which stands for ``corresponding parts of \PMlinkname{congruent}{Congruent2} \PMlinkname{triangles}{Triangle} are congruent.''  In other \PMlinkescapetext{words}, if 
\begin{align}
\triangle ABC \cong \triangle DEF,
\end{align}
then all of the following are true:

\begin{itemize}
\item $\angle A \cong \angle D$;
\item $\overline{AB} \cong \overline{DE}$;
\item $\angle B \cong \angle E$;
\item $\overline{BC} \cong \overline{EF}$;
\item $\angle C \cong \angle F$;
\item $\overline{AC} \cong \overline{DF}$.
\end{itemize}

\textbf{Note.}\; The notation in (1) indicates that the isomorphism in question maps $A\mapsto D$,\, $B\mapsto E$,\, and\, $C\mapsto F$.
%%%%%
%%%%%
\end{document}
