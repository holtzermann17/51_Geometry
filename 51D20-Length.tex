\documentclass[12pt]{article}
\usepackage{pmmeta}
\pmcanonicalname{Length}
\pmcreated{2013-03-22 16:31:45}
\pmmodified{2013-03-22 16:31:45}
\pmowner{PrimeFan}{13766}
\pmmodifier{PrimeFan}{13766}
\pmtitle{length}
\pmrecord{17}{38709}
\pmprivacy{1}
\pmauthor{PrimeFan}{13766}
\pmtype{Definition}
\pmcomment{trigger rebuild}
\pmclassification{msc}{51D20}
\pmclassification{msc}{51-00}
\pmclassification{msc}{51M25}
\pmrelated{Area2}
\pmrelated{Volume2}
\pmrelated{BasicPolygon}
\pmdefines{length of a polygon}
\pmdefines{length of a set}

% this is the default PlanetMath preamble.  as your knowledge
% of TeX increases, you will probably want to edit this, but
% it should be fine as is for beginners.

% almost certainly you want these
\usepackage{amssymb}
\usepackage{amsmath}
\usepackage{amsfonts}

% used for TeXing text within eps files
%\usepackage{psfrag}
% need this for including graphics (\includegraphics)
%\usepackage{graphicx}
% for neatly defining theorems and propositions
%\usepackage{amsthm}
% making logically defined graphics
%%%\usepackage{xypic}

% there are many more packages, add them here as you need them

% define commands here
% as suggested by Wkbj79
\usepackage{pstricks}
\begin{document}
The \emph{length} of a line segment is the distance between its endpoints. Length may be measured in meters, yards, abstract units, etc. For example, in the following diagram

\begin{center}
\begin{pspicture}(-1,-0.3)(13,0.3)
\psline{<->}(-1,0)(13,0)
\psdots(0,0)(3,0)(9,0)(9.6,0)(12,0)
\rput[a](0,-0.3){0}
\rput[a](3,-0.3){1}
\rput[a](9,-0.3){3}
\rput[a](9.6,-0.3){3.2}
\rput[a](12,-0.3){4}
\rput[b](0,0.2){$A$}
\rput[b](3,0.2){$B$}
\rput[b](9,0.2){$C$}
\rput[b](9.6,0.2){$D$}
\rput[b](12,0.2){$E$}
\rput[l](-1,0){.}
\rput[r](13,0){.}
\end{pspicture}
\end{center}

$AB$ is one unit long, $BC$ is two units long, $CD$ is a fifth of a unit, $DE$ is four fifths of a unit, $AC$ is three units, etc.

The concept of length as defined above is a special case of a general concept called measure.

In two-dimensional space, length usually goes along the $x$ axis while height goes along the $y$ axis. The same holds in three-dimensional space.

For triangles, pentagons, and higher \PMlinkname{$n$-gons}{BasicPolygon}, it is customary to refer to the length of any of its sides as the \emph{length of the \PMlinkescapetext{polygon}}. The length of a circle's side is called its circumference.

In set theory, \emph{length} refers to the number of elements a set (or one-dimensional array) has. This is also known as cardinality. For example, $\{2, 5, 11, 23, 47\}$ has length $5$.
%%%%%
%%%%%
\end{document}
