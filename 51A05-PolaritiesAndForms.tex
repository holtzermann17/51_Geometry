\documentclass[12pt]{article}
\usepackage{pmmeta}
\pmcanonicalname{PolaritiesAndForms}
\pmcreated{2013-03-22 15:58:13}
\pmmodified{2013-03-22 15:58:13}
\pmowner{Algeboy}{12884}
\pmmodifier{Algeboy}{12884}
\pmtitle{polarities and forms}
\pmrecord{7}{37986}
\pmprivacy{1}
\pmauthor{Algeboy}{12884}
\pmtype{Topic}
\pmcomment{trigger rebuild}
\pmclassification{msc}{51A05}
\pmrelated{polarity}
\pmrelated{Projectivity}
\pmrelated{ProjectiveGeometry}
\pmrelated{Isometry2}
\pmrelated{ProjectiveGeometry3}
\pmrelated{ClassicalGroups}
\pmrelated{Polarity2}
\pmrelated{DualityWithRespectToANonDegenerateBilinearForm}

\usepackage{latexsym}
\usepackage{amssymb}
\usepackage{amsmath}
\usepackage{amsfonts}
\usepackage{amsthm}

%%\usepackage{xypic}

%-----------------------------------------------------

%       Standard theoremlike environments.

%       Stolen directly from AMSLaTeX sample

%-----------------------------------------------------

%% \theoremstyle{plain} %% This is the default

\newtheorem{thm}{Theorem}

\newtheorem{coro}[thm]{Corollary}

\newtheorem{lem}[thm]{Lemma}

\newtheorem{lemma}[thm]{Lemma}

\newtheorem{prop}[thm]{Proposition}

\newtheorem{conjecture}[thm]{Conjecture}

\newtheorem{conj}[thm]{Conjecture}

\newtheorem{defn}[thm]{Definition}

\newtheorem{remark}[thm]{Remark}

\newtheorem{ex}[thm]{Example}



%\countstyle[equation]{thm}



%--------------------------------------------------

%       Item references.

%--------------------------------------------------


\newcommand{\exref}[1]{Example-\ref{#1}}

\newcommand{\thmref}[1]{Theorem-\ref{#1}}

\newcommand{\defref}[1]{Definition-\ref{#1}}

\newcommand{\eqnref}[1]{(\ref{#1})}

\newcommand{\secref}[1]{Section-\ref{#1}}

\newcommand{\lemref}[1]{Lemma-\ref{#1}}

\newcommand{\propref}[1]{Prop\-o\-si\-tion-\ref{#1}}

\newcommand{\corref}[1]{Cor\-ol\-lary-\ref{#1}}

\newcommand{\figref}[1]{Fig\-ure-\ref{#1}}

\newcommand{\conjref}[1]{Conjecture-\ref{#1}}


% Normal subgroup or equal.

\providecommand{\normaleq}{\unlhd}

% Normal subgroup.

\providecommand{\normal}{\lhd}

\providecommand{\rnormal}{\rhd}
% Divides, does not divide.

\providecommand{\divides}{\mid}

\providecommand{\ndivides}{\nmid}


\providecommand{\union}{\cup}

\providecommand{\bigunion}{\bigcup}

\providecommand{\intersect}{\cap}

\providecommand{\bigintersect}{\bigcap}










\begin{document}
Through out this article we assume $\dim V\neq 2$.  This is not a true constraint as there are only trivial dualities for $\dim V\leq 2$.

\begin{prop}
Every duality gives rise to a non-degenerate sesquilinear form,
and visa-versa.
\end{prop}
\begin{proof}
To see this, let $d:PG(V)\rightarrow PG(V)$ be a duality.  We may express this as an order preserving map $d:PG(V)\rightarrow PG(V^*)$.  Then by the fundamental theorem of projective geometry it follows $d$
is induced by a bijective semi-linear transformation $\hat{d}:V\rightarrow V^*$. 

An semi-linear isomorphism of $V$ to $V^*$ is equivalent to specifying a non-degenerate sesquilinear form. In particular, define the form $b:V\times V\rightarrow k$ by $b(v,w)=(v)(w\hat{d})$ (notice $w\hat{d}\in V^*$ so $w\hat{d}:V\rightarrow k$).

Now, if $b:V\times V\rightarrow k$ is a non-degenerate sesquilinear form.  Then
define
\[\hat{b}:V\rightarrow V^*:v\mapsto b(-,v):V\rightarrow k\]
which is semi-linear, as $b$ is sesquilinear, and bijective, since $b$ is
non-degenerate.   Therefore $\hat{b}$ induces an order preserving bijection $PG(V)\rightarrow PG(V^*)$, that is, a duality.
\end{proof}

We write $W^\perp$ for the image of the induced duality of a non-degenerate
sesquilinear form $b$.  Notice that $W^\perp=\{w\in V:b(v,W)=0\}$.  (Although the form may not be reflexive, we still use the $\perp$ notation, but we now demonstrate that we can indeed specialize to the reflexive case.)
Notice then that
\[\dim W^\perp=\dim V-\dim W.\]

\begin{coro}
Every polarity gives rise to a reflexive non-degenerate sesquilinear form,
and visa-versa.
\end{coro}
\begin{proof}
Let $b$ be the sesquilinear form induced by the polarity $p$.  Then suppose we
have $v,w\in V$ such that $0=b(v,w)=(v)(w\hat{p})$.  So 
$\langle v\rangle \leq \langle w\rangle^\perp=\langle w\rangle p$.  But $p$ has order 2 so $\langle v\rangle^\perp=\langle v\rangle p\geq \langle w\rangle$.  But this implies $b(w,v)=0$ so $b$ is reflexive.

Likewise, given a reflexive non-degenerate sesquilinear form $b$ it gives rise
do a duality $d$ induced by $\hat{b}$.  By the reflexivity, $b(W,W^\perp)=0$ implies $b(W^\perp, W)=0$ also.  As $(W^\perp)^\perp=\{v\in V:b(v,(W^\perp)^\perp)=0\}$ it follows $W\leq (W^\perp)^\perp$.  But by 
dimension arguments: 
\[\dim (W^\perp)^\perp =\dim V-\dim W^\perp=\dim V-(\dim V-\dim W)=\dim W\]
we conclude $W=(W^\perp)^\perp$.  Thus $d$ is a polarity.
\end{proof}

From the fundamental theorem of projective geometry it follows if $\dim V\neq 2$ then every order preserving map is induced by a semi-linear transformation of $V$.  In similar fashion we have

\begin{prop}
$P\Gamma L^*(V)=P\Gamma L(V)\rtimes \mathbb{Z}_2$, meaning that every order
reversing map $f:PG(V)\rightarrow PG(V)$ can be decomposed as a $f=sr$ where
$s$ is induced from a semi-linear transformation and $r$ is a polarity.
\end{prop}
\begin{proof}
Let $d$ be any duality of $PG(V)$.  Then $d^2$ is order preserving.  Thus
$d^2$ is a projectivity so by the fundamental theorem of projective geometry
$d^2$ is induced by a semi-linear transformation $s$.  Therefore
$P\Gamma L(V)$ has index 2 in $P\Gamma L^*(V)$.  Finally it suffices to 
provide any polarity of $PG(V)$ to prove $P\Gamma L^*(V)=P\Gamma L(V)\rtimes \mathbb{Z}_2$.
For this use any reflexive non-degenerate sesquilinear form.
\end{proof}

\begin{remark}
The group $P\Gamma L^*(V)$ is the automorphism group of $PSL(V)$.
In particular, the polarities account for the graph automorphisms of the 
Dynkin diagram of $A_{d-1}$, $d=\dim V$.  When $\dim V=2$ there is no
graph automorphism, just as there are no dualities (points are hyperplanes 
when $\dim V=2$.)
\end{remark}


\bibliographystyle{amsplain}
\providecommand{\bysame}{\leavevmode\hbox to3em{\hrulefill}\thinspace}
\providecommand{\MR}{\relax\ifhmode\unskip\space\fi MR }
% \MRhref is called by the amsart/book/proc definition of \MR.
\providecommand{\MRhref}[2]{%
\href{http://www.ams.org/mathscinet-getitem?mr=#1}{#2}
}
\providecommand{\href}[2]{#2}
\begin{thebibliography}{10}


\bibitem{HP}
Gruenberg, K. W. and Weir, A.J.
\emph{Linear Geometry 2nd Ed.} (English)
[B] Graduate Texts in Mathematics. 49. New York - Heidelberg - Berlin: Springer-Verlag. X, 198 p. DM 29.10; \$ 12.80 (1977).

\bibitem{Ka}
Kantor, W. M.
\emph{Lectures notes on Classical Groups}.

\end{thebibliography}

%%%%%
%%%%%
\end{document}
