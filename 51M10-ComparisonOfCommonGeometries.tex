\documentclass[12pt]{article}
\usepackage{pmmeta}
\pmcanonicalname{ComparisonOfCommonGeometries}
\pmcreated{2013-03-22 17:13:06}
\pmmodified{2013-03-22 17:13:06}
\pmowner{Wkbj79}{1863}
\pmmodifier{Wkbj79}{1863}
\pmtitle{comparison of common geometries}
\pmrecord{17}{39543}
\pmprivacy{1}
\pmauthor{Wkbj79}{1863}
\pmtype{Topic}
\pmcomment{trigger rebuild}
\pmclassification{msc}{51M10}
\pmclassification{msc}{51M05}
\pmclassification{msc}{51-01}
\pmclassification{msc}{51-00}
\pmrelated{EuclideanGeometry}
\pmrelated{NonEuclideanGeometry}
\pmrelated{Geometry}

\endmetadata

\usepackage{amssymb}
\usepackage{amsmath}
\usepackage{amsfonts}
\usepackage{pstricks}
\usepackage{psfrag}
\usepackage{graphicx}
\usepackage{amsthm}
%%\usepackage{xypic}

\begin{document}
\PMlinkescapetext{This entry is best viewed in page images mode, but should look somewhat decent in html mode.}

In this entry, the most common models of the three most common two-dimensional geometries (\PMlinkname{Euclidean}{EuclideanGeometry}, \PMlinkname{hyperbolic}{HyperbolicGeometry}, and \PMlinkname{spherical}{SphericalGeometry}) will be considered.

The following abbreviations will be used in this entry:

\begin{itemize}
\item $E^2$ for the Euclidean plane (the most common model for two-dimensional Euclidean geometry);
\item $\mathbb{H}^2$ for two-dimensional hyperbolic geometry;
\item $BK$ for the Beltrami-Klein model of $\mathbb{H}^2$;
\item $PD$ for the Poincar\'e disc model of $\mathbb{H}^2$;
\item $UHP$ for the upper half plane model of $\mathbb{H}^2$;
\item $S^2$ for the unit sphere (the most common model for two-dimensional spherical geometry).
\end{itemize}

\section{Comparison of Properties of the Models}

\begin{center}
\begin{tabular}{|l||c|c|c|c|c|}
\hline
property & $E^2$ & $BK$ & $PD$ & $UHP$ & $S^2$ \\
\hline \hline
& & & & & \\
model has \PMlinkescapetext{finite} area when & no & yes & yes & no & yes \\
considered as a subset of a & & & & & \\
Euclidean space & & & & & \\
& & & & & \\
\hline
& & & & & \\
lines in model look like & lines & line segments & some line segments, & some vertical rays, & circles \\
& & & some arcs of circles & some semicircles & \\
& & & & & \\
\hline
& & & & & \\
lines have \PMlinkescapetext{finite} length when & no & yes & yes & yes for semicircles, & yes \\
considered as a subset of a & & & & no for vertical rays & \\
Euclidean space & & & & & \\
& & & & & \\
\hline
& & & & & \\
angles are preserved in & yes & no & yes & yes & yes \\
model & & & & & \\
& & & & & \\
\hline
\end{tabular}
\end{center}

\section{Comparison of Properties of the Geometries}

\begin{center}
\begin{tabular}{|l||c|c|c|}
\hline
property & $E^2$ & $\mathbb{H}^2$ & $S^2$ \\
\hline \hline
& & & \\
two distinct points determine a unique line & yes & yes & no \\
& & & (yes if points are not antipodal) \\
& & & \\
\hline
& & & \\
parallel lines exist & yes & yes & no \\
& & & \\
\hline
& & & \\
number of lines parallel to a given line and & 1 & $\infty$ & 0 \\
passing through a point not on the given line & & & \\
& & & \\
\hline
& & & \\
\PMlinkescapetext{entire} space has infinite area with respect & yes & yes & no \\
to its own geometry & & & \\
& & & \\
\hline
& & & \\
lines have infinite length & yes & yes & no \\
& & & \\
\hline
& & & \\
number of \PMlinkname{centers}{Center8} of a circle & 1 & 1 & 2 \\
& & & \\
\hline
& & & \\
angle sum $\Sigma$ of triangles (in radians) & $\Sigma=\pi$ & $0<\Sigma<\pi$ & $\pi<\Sigma<3\pi$ \\
& & & \\
\hline
& & & \\
ASA holds & yes & yes & yes \\
& & & \\
\hline
& & & \\
SAS holds & yes & yes & yes \\
& & & \\
\hline
& & & \\
SSS holds & yes & yes & yes \\
& & & \\
\hline
& & & \\
AAS holds & yes & yes & \PMlinkname{no}{AASIsNotValidInSphericalGeometry} \\
& & & \\
\hline
& & & \\
AAA holds & no & yes & yes \\
& & & \\
\hline
\end{tabular}
\end{center}
%%%%%
%%%%%
\end{document}
