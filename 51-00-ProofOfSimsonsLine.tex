\documentclass[12pt]{article}
\usepackage{pmmeta}
\pmcanonicalname{ProofOfSimsonsLine}
\pmcreated{2013-03-22 13:08:26}
\pmmodified{2013-03-22 13:08:26}
\pmowner{giri}{919}
\pmmodifier{giri}{919}
\pmtitle{proof of Simson's line}
\pmrecord{9}{33578}
\pmprivacy{1}
\pmauthor{giri}{919}
\pmtype{Proof}
\pmcomment{trigger rebuild}
\pmclassification{msc}{51-00}

\endmetadata

% this is the default PlanetMath preamble.  as your knowledge
% of TeX increases, you will probably want to edit this, but
% it should be fine as is for beginners.

% almost certainly you want these
\usepackage{amssymb}
\usepackage{amsmath}
\usepackage{amsfonts}

% used for TeXing text within eps files
\usepackage{psfrag}
% need this for including graphics (\includegraphics)
\usepackage{graphicx}
% for neatly defining theorems and propositions
%\usepackage{amsthm}
% making logically defined graphics
%%%\usepackage{xypic}

% there are many more packages, add them here as you need them

% define commands here
\begin{document}
Given a  $\triangle ABC$ with a point $P$ on its circumcircle (other than $A,B,C$),                          
we will prove that the feet of the perpendiculars drawn from P to the sides $AB,BC,CA$                
(or their prolongations) are collinear.

\begin{center}
\includegraphics{simson.eps}
\end{center}

Since $PW$ is perpendicular to $BW$ and $PU$ is perpendicular to $BU$
the point $P$ lies on the circumcircle of $\triangle BUW $.                              

By similar arguments, $P$ also lies on the circumcircle of  $\triangle AWV$ and $\triangle CUV$.

This implies that  $PUBW$ ,  $PUCV$ and $PVWA$ are all cyclic quadrilaterals.

Since  $PUBW$ is a cyclic quadrilateral,

$$\angle UPW = 180^\circ -\angle UBW$$

implies  $$\angle UPW = 180^\circ -\angle CBA$$

Also $CPAB$ is a cyclic quadrilateral, therefore

$$\angle CPA = 180^\circ -\angle CBA$$

(opposite angles in a cyclic quarilateral are supplementary).

From these two, we get

$$\angle UPW =\angle CPA$$
                   
Subracting $\angle CPW$, we have

$$\angle UPC =\angle WPA$$

Now, since $PVWA$ is a cyclic quadrilateral,
$$\angle WPA = \angle WVA$$

also, since  $UPVC$  is a cyclic quadrilateral,
$$\angle UPC = \angle UVC$$

Combining these two results with the previous one, we have

$$\angle WVA =\angle UVC$$

This implies that the points $U, V, W$ are collinear.
%%%%%
%%%%%
\end{document}
