\documentclass[12pt]{article}
\usepackage{pmmeta}
\pmcanonicalname{Centroid}
\pmcreated{2013-03-22 11:55:44}
\pmmodified{2013-03-22 11:55:44}
\pmowner{drini}{3}
\pmmodifier{drini}{3}
\pmtitle{centroid}
\pmrecord{11}{30647}
\pmprivacy{1}
\pmauthor{drini}{3}
\pmtype{Definition}
\pmcomment{trigger rebuild}
\pmclassification{msc}{51-00}
\pmsynonym{barycenter}{Centroid}
\pmsynonym{center of gravity}{Centroid}
\pmrelated{Median}
\pmrelated{Orthocenter}
\pmrelated{Triangle}
\pmrelated{EulerLine}
\pmrelated{CevasTheorem}
\pmrelated{CenterOfATriangle}
\pmrelated{LemoinePoint}
\pmrelated{GergonneTriangle}
\pmrelated{TrigonometricVersionOfCevasTheorem}

\endmetadata

\usepackage{amssymb}
\usepackage{amsmath}
\usepackage{amsfonts}
\usepackage{graphicx}
%%%%\usepackage{xypic}

\begin{document}
The \emph{centroid} of a triangle (also called \emph{center of gravity} of 
the triangle) is the point where the three medians intersect each other.

\begin{center}
\includegraphics{centroid}
\end{center}

In the figure, $AA', BB'$ and $CC'$ are medians and $G$ is the centroid of $ABC$.
The centroid $G$ has the property that divides the medians in the ratio $2:1$, that is 
$$AG=2GA'\quad BG=2GB'\quad CG=2GC'.$$
%%%%%
%%%%%
%%%%%
%%%%%
\end{document}
