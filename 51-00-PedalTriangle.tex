\documentclass[12pt]{article}
\usepackage{pmmeta}
\pmcanonicalname{PedalTriangle}
\pmcreated{2013-03-22 13:08:28}
\pmmodified{2013-03-22 13:08:28}
\pmowner{CWoo}{3771}
\pmmodifier{CWoo}{3771}
\pmtitle{pedal triangle}
\pmrecord{8}{33579}
\pmprivacy{1}
\pmauthor{CWoo}{3771}
\pmtype{Definition}
\pmcomment{trigger rebuild}
\pmclassification{msc}{51-00}

\endmetadata

% this is the default PlanetMath preamble.  as your knowledge
% of TeX increases, you will probably want to edit this, but
% it should be fine as is for beginners.

% almost certainly you want these
\usepackage{amssymb}
\usepackage{amsmath}
\usepackage{amsfonts}

% used for TeXing text within eps files
%\usepackage{psfrag}
% need this for including graphics (\includegraphics)
\usepackage{graphicx}
% for neatly defining theorems and propositions
%\usepackage{amsthm}
% making logically defined graphics
%%%\usepackage{xypic}

% there are many more packages, add them here as you need them
\usepackage{bbm}
% define commands here
\begin{document}
The \emph{pedal triangle} of any triangle $\triangle ABC$,
is the triangle whose vertices are the feet of perpendiculars from 
$A$, $B$ and $C$ to their opposite sides in $\triangle ABC$.

In this figure, the $\triangle DEF$ is the pedal triangle
of $\triangle ABC$.
\begin{center}
\includegraphics{triangle.1.eps}
\end{center}

In general, for any point $P$ inside a triangle, the \emph{pedal triangle} of $P$ is a triangle whose vertices are the 
feet of perpendiculars from $P$ to the sides of the triangle.

In the following figure, the $\triangle D'E'F'$ is the pedal
triangle of $P$ in $\triangle ABC$.

\begin{center}
\includegraphics{triangle.2.eps}
\end{center}

%%%%%
%%%%%
\end{document}
