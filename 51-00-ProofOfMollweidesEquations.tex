\documentclass[12pt]{article}
\usepackage{pmmeta}
\pmcanonicalname{ProofOfMollweidesEquations}
\pmcreated{2013-03-22 12:50:10}
\pmmodified{2013-03-22 12:50:10}
\pmowner{mathwizard}{128}
\pmmodifier{mathwizard}{128}
\pmtitle{proof of Mollweide's equations}
\pmrecord{5}{33161}
\pmprivacy{1}
\pmauthor{mathwizard}{128}
\pmtype{Proof}
\pmcomment{trigger rebuild}
\pmclassification{msc}{51-00}

\endmetadata

% this is the default PlanetMath preamble.  as your knowledge
% of TeX increases, you will probably want to edit this, but
% it should be fine as is for beginners.

% almost certainly you want these
\usepackage{amssymb}
\usepackage{amsmath}
\usepackage{amsfonts}

% used for TeXing text within eps files
%\usepackage{psfrag}
% need this for including graphics (\includegraphics)
%\usepackage{graphicx}
% for neatly defining theorems and propositions
%\usepackage{amsthm}
% making logically defined graphics
%%%\usepackage{xypic}

% there are many more packages, add them here as you need them

% define commands here
\begin{document}
We transform the equation
$$(a+b)\sin\frac{\gamma}{2}=c\cos\left(\frac{\alpha -\beta}{2}\right)$$
to
$$a\cos\left(\frac{\alpha}{2}+\frac{\beta}{2}\right) +b\cos\left(\frac{\alpha}{2}+\frac{\beta}{2}\right)= c\cos\frac{\alpha}{2}\cos\frac{\beta}{2} +c\sin\frac{\alpha}{2}\sin\frac{\beta}{2} ,$$
using the fact that $\gamma=\pi-\alpha-\beta$. The left hand side can be further expanded, so that we get:
$$a\left(\cos\frac{\alpha}{2}\cos\frac{\beta}{2}- \sin\frac{\alpha}{2}\sin\frac{\beta}{2}\right)+ b\left(\cos\frac{\alpha}{2}\cos\frac{\beta}{2}- \sin\frac{\alpha}{2}\sin\frac{\beta}{2}\right)= c\cos\frac{\alpha}{2}\cos\frac{\beta}{2} +c\sin\frac{\alpha}{2}\sin\frac{\beta}{2}.$$
Collecting terms we get:
$$(a+b-c)\cos\frac{\alpha}{2}\cos\frac{\beta}{2}- (a+b+c)\sin\frac{\alpha}{2}\sin\frac{\beta}{2}=0.$$
Using $s:=\frac{a+b+c}{2}$ and using the equations
\begin{eqnarray*}
\sin\frac{\alpha}{2}&=&\sqrt{\frac{(s-b)(s-c)}{bc}}\\
\cos\frac{\beta}{2}&=&\sqrt{\frac{s(s-a)}{bc}}
\end{eqnarray*}
we get:
$$2\frac{s(s-c)}{c}\sqrt{\frac{(s-a)(s-b)}{ab}}- 2\frac{s(s-c)}{c}\sqrt{\frac{(s-a)(s-b))}{ab}}=0,$$
which is obviously true. So we can prove the first equation by going backwards. The second equation can be proved in quite the same way.
%%%%%
%%%%%
\end{document}
