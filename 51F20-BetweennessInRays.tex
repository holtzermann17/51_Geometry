\documentclass[12pt]{article}
\usepackage{pmmeta}
\pmcanonicalname{BetweennessInRays}
\pmcreated{2013-03-22 15:33:05}
\pmmodified{2013-03-22 15:33:05}
\pmowner{CWoo}{3771}
\pmmodifier{CWoo}{3771}
\pmtitle{betweenness in rays}
\pmrecord{6}{37449}
\pmprivacy{1}
\pmauthor{CWoo}{3771}
\pmtype{Definition}
\pmcomment{trigger rebuild}
\pmclassification{msc}{51F20}
\pmclassification{msc}{51G05}
\pmrelated{Angle}
\pmrelated{Ray}
\pmrelated{Midpoint4}
\pmdefines{interior point}
\pmdefines{between rays}
\pmdefines{between two rays}
\pmdefines{crossbar theorem}

\usepackage{amssymb,amscd}
\usepackage{amsmath}
\usepackage{amsfonts}

% used for TeXing text within eps files
%\usepackage{psfrag}
% need this for including graphics (\includegraphics)
%\usepackage{graphicx}
% for neatly defining theorems and propositions
%\usepackage{amsthm}
% making logically defined graphics
%%%\usepackage{xypic}

% define commands here

\renewcommand{\line}[1]{\overleftrightarrow{#1}}
\newcommand{\ray}[1]{\overrightarrow{#1}}
\begin{document}
Let $S$ be a linear ordered geometry.
Fix a point $p$ and consider the pencil $\Pi(p)$ of all rays
emanating from it.  Let $\alpha\neq\beta \in\Pi(p)$.  A point $q$ is
said to be an \emph{interior point} of $\alpha$ and $\beta$ if there
are points $s\in\alpha$ and $t\in\beta$ such that
\begin{enumerate}
\item $q$ and $s$ are on the same side of line $\line{pt}$, and
\item $q$ and $t$ are on the same side of line $\line{ps}$.
\end{enumerate}
A point $q$ is said to be \emph{between} $\alpha$ and $\beta$ if
there are points $s\in\alpha$ and $t\in\beta$ such that $q$ is
between $s$ and $t$.  A point that is between two rays is an
interior point of these rays, but not vice versa in general.  A ray
$\rho\in\Pi(p)$ is said to be \emph{between} rays $\alpha$ and
$\beta$ if there is an interior point of $\alpha$ and $\beta$ lying
on $\rho$.
\\\\
\textbf{Properties}
\begin{enumerate}
\item Suppose $\alpha,\beta,\rho\in\Pi(p)$ and $\rho$ is between
$\alpha$ and $\beta$.  Then
\begin{enumerate}
\item any point on $\rho$ is an interior point of $\alpha$ and
$\beta$;
\item any point on the line containing $\rho$ that is an interior
point of $\alpha$ and $\beta$ must be a point on $\rho$;
\item there is a point $q$ on $\rho$ that is between $\alpha$ and
$\beta$.  This is known as the \textbf{Crossbar Theorem}, the two ``crossbars'' being $\rho$ and a line segment joining a point on $\alpha$ and a point on $\beta$;
\item if $q$ is defined as above, then any point between $p$ and
$q$ is between $\alpha$ and $\beta$.
\end{enumerate}
\item There are no rays between two opposite rays.
\item If $\rho$ is between $\alpha$ and $\beta$, then $\beta$ is not
between $\alpha$ and $\rho$.
\item If $\alpha,\beta\in\Pi(p)$ has a ray $\rho$ between them, then
$\alpha$ and $\beta$ must lie on the same half plane of some line.
\item The converse of the above statement is true too: if
$\alpha,\beta\in\Pi(p)$ are distinct rays that are not opposite of
one another, then there exist a ray $\rho\in\Pi(p)$ such that $\rho$
is between $\alpha$ and $\beta$.
\item Given $\alpha,\beta\in\Pi(p)$ with $\alpha\neq\beta$ and
$\alpha\neq-\beta$.  We can write $\Pi(p)$ as a disjoint union of
two subsets:
\begin{enumerate}
\item $A =\lbrace \rho\in\Pi(p)\mid
\rho\mbox{ is between }\alpha\mbox{ and }\beta\rbrace$,
\item $B=\Pi(p)-A$.
\end{enumerate}
Let $\rho,\sigma\in\Pi(p)$ be two rays distinct from both $\alpha$
and $\beta$.  Suppose $x\in\rho$ and $y\in\sigma$.  Then
$\rho,\sigma$ belong to the same subset if and only if
$\overline{xy}$ does not intersect either $\alpha$ or $\beta$.
\end{enumerate}

\begin{thebibliography}{6}
\bibitem{dh} D. Hilbert, {\it Foundations of Geometry}, Open Court Publishing Co. (1971)
\bibitem{bs} K. Borsuk and W. Szmielew, {\it Foundations of Geometry}, North-Holland Publishing Co. Amsterdam (1960)
\bibitem{mg} M. J. Greenberg, {\it Euclidean and Non-Euclidean Geometries, Development and History}, W. H. Freeman and Company, San Francisco (1974)
\end{thebibliography}
%%%%%
%%%%%
\end{document}
