\documentclass[12pt]{article}
\usepackage{pmmeta}
\pmcanonicalname{Quadrant}
\pmcreated{2014-12-16 22:42:51}
\pmmodified{2014-12-16 22:42:51}
\pmowner{Mathprof}{13753}
\pmmodifier{pahio}{2872}
\pmtitle{quadrant}
\pmrecord{14}{36293}
\pmprivacy{1}
\pmauthor{Mathprof}{2872}
\pmtype{Definition}
\pmcomment{trigger rebuild}
\pmclassification{msc}{51N20}
\pmsynonym{quarter-plane}{Quadrant}
\pmrelated{RightAngle}
\pmrelated{ConvexAngle}
\pmrelated{Octant}
\pmdefines{1st quadrant}
\pmdefines{2nd quadrant}
\pmdefines{3rd quadrant}
\pmdefines{4th quadrant}

% this is the default PlanetMath preamble.  as your knowledge
% of TeX increases, you will probably want to edit this, but
% it should be fine as is for beginners.

% almost certainly you want these
\usepackage{amssymb}
\usepackage{amsmath}
\usepackage{amsfonts}

% used for TeXing text within eps files
%\usepackage{psfrag}
% need this for including graphics (\includegraphics)
%\usepackage{graphicx}
% for neatly defining theorems and propositions
%\usepackage{amsthm}
% making logically defined graphics
%%%\usepackage{xypic}

% there are many more packages, add them here as you need them

% define commands here
\begin{document}
The $x$- and $y$-axes \PMlinkescapetext{divide} the $xy$-plane in four right-angle domains which are called the {\em quadrants} of the plane. \,They are numbered going round the origin in anticlockwise direction so  that
$$\{(x,\,y) \mid \;  x > 0,\; y > 0\}$$
is the {\em first quadrant}, 
$$\{(x,\,y) \mid \;  x < 0,\; y > 0\}$$
the {\em second quadrant}, and so on. 

Naturally, one can speak of the quadrants of the complex plane, too.\\

The lines $y = \pm x$ have as their slope angles $\pm 45^\circ$,  
thus halving the quadrant angles; they are called the 
{\it quadrant bisectors}.\, Cf. angle bisector as locus.




%%%%%
%%%%%
\end{document}
