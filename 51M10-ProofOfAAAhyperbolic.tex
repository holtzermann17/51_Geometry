\documentclass[12pt]{article}
\usepackage{pmmeta}
\pmcanonicalname{ProofOfAAAhyperbolic}
\pmcreated{2013-03-22 17:08:46}
\pmmodified{2013-03-22 17:08:46}
\pmowner{Wkbj79}{1863}
\pmmodifier{Wkbj79}{1863}
\pmtitle{proof of AAA (hyperbolic)}
\pmrecord{6}{39454}
\pmprivacy{1}
\pmauthor{Wkbj79}{1863}
\pmtype{Proof}
\pmcomment{trigger rebuild}
\pmclassification{msc}{51M10}

\endmetadata

\usepackage{amssymb}
\usepackage{amsmath}
\usepackage{amsfonts}
\usepackage{pstricks}
\usepackage{psfrag}
\usepackage{graphicx}
\usepackage{amsthm}
%%\usepackage{xypic}

\begin{document}
Following is a proof that AAA holds in hyperbolic geometry.

\begin{proof}
Suppose that we have two triangles $\triangle ABC$ and $\triangle DEF$ such that all three pairs of corresponding angles are congruent, but that the two triangles are not congruent.  Without loss of generality, let us further assume that $\ell(AB)<\ell(DE)$, where $\ell$ is used to denote length.  (Note that, if $\ell(AB)=\ell(DE)$, then the two triangles would be congruent by ASA.)  Then there are three cases:

\begin{enumerate}
\item $\ell(AC)>\ell(DF)$
\item $\ell(AC)=\ell(DF)$
\item $\ell(AC)<\ell(DF)$
\end{enumerate}

Before investigating the cases, $\triangle DEF$ will be placed on $\triangle ABC$ so that the following are true:

\begin{itemize}
\item $A$ and $D$ correspond
\item $A$, $B$, and $E$ are collinear
\item $A$, $C$, and $F$ are collinear
\end{itemize}

Now let us investigate each case.

Case 1:  Let $G$ denote the intersection of $\overline{BC}$ and $\overline{EF}$

\begin{center}
\begin{pspicture}(-2,-2)(2,2.5)
\psline(-1.5,-1.5)(0,2)
\psline(-1.5,-1.5)(1.17,-0.97)
\psline(-1.2,-0.8)(1.3,-1.3)
\psline(0,2)(1.3,-1.3)
\psdots(-1.5,-1.5)(-1.2,-0.8)(0,2)(1.17,-0.97)(1.3,-1.3)(0.407,-1.12)
\rput[b](0,2.1){$A=D$}
\rput[r](-1.25,-0.8){$B$}
\rput[r](-1.55,-1.5){$E$}
\rput[1](1.4,-0.8){$F$}
\rput[1](1.5,-1.5){$C$}
\rput[a](0.407,-1.35){$G$}
\end{pspicture}
\end{center}

Note that $\angle ABC$ and $\angle CBE$ are supplementary.  By hypothesis, $\angle ABC$ and $\angle DEF$ are congruent.  Thus, $\angle CBE$ and $\angle DEF$ are supplementary.  Therefore, $\triangle BEG$ contains two angles which are supplementary, a contradiction.

Case 2:

\begin{center}
\begin{pspicture}(-2,-2)(2,2.5)
\psline(-1.5,-1.5)(0,2)
\psline(-1.5,-1.5)(1.35,-1.2)
\psline(-1.2,-0.8)(1.35,-1.2)
\psline(0,2)(1.35,-1.2)
\psdots(-1.5,-1.5)(-1.2,-0.8)(0,2)(1.35,-1.2)
\rput[b](0,2.1){$A=D$}
\rput[r](-1.25,-0.8){$B$}
\rput[r](-1.55,-1.5){$E$}
\rput[l](1.4,-1.2){$C=F$}
\end{pspicture}
\end{center}

Note that $\angle ABC$ and $\angle CBE$ are supplementary.  By hypothesis, $\angle ABC$ and $\angle DEF$ are congruent.  Thus, $\angle CBE$ and $\angle DEF$ are supplementary.  Therefore, $\triangle BCE$ contains two angles which are supplementary, a contradiction.

Case 3:  This is the most interesting case, as it is the one that holds in Euclidean geometry.

\begin{center}
\begin{pspicture}(-2,-2)(2,2.5)
\psline(-1.5,-1.5)(0,2)
\psline(-1.5,-1.5)(1.3,-1.3)
\psline(-1.2,-0.8)(1.17,-0.97)
\psline(0,2)(1.3,-1.3)
\psdots(-1.5,-1.5)(-1.2,-0.8)(0,2)(1.17,-0.97)(1.3,-1.3)
\rput[b](0,2.1){$A=D$}
\rput[r](-1.25,-0.8){$B$}
\rput[r](-1.55,-1.5){$E$}
\rput[1](1.4,-0.8){$C$}
\rput[1](1.5,-1.5){$F$}
\end{pspicture}
\end{center}

Note that $\angle ABC$ and $\angle CBE$ are supplementary.  By hypothesis, $\angle ABC$ and $\angle DEF$ are congruent.  Thus, $\angle CBE$ and $\angle DEF$ are supplementary.  Similarly, $\angle BCF$ and $\angle DFE$ are supplementary.  Thus, $BCFE$ is a quadrilateral whose angle sum is exactly $2\pi$ radians, a contradiction.

Since none of the three cases is possible, it follows that $\triangle ABC$ and $\triangle DEF$ are congruent.

\end{proof}
%%%%%
%%%%%
\end{document}
