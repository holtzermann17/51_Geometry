\documentclass[12pt]{article}
\usepackage{pmmeta}
\pmcanonicalname{SupplementaryAngles}
\pmcreated{2013-03-22 17:34:59}
\pmmodified{2013-03-22 17:34:59}
\pmowner{pahio}{2872}
\pmmodifier{pahio}{2872}
\pmtitle{supplementary angles}
\pmrecord{8}{39995}
\pmprivacy{1}
\pmauthor{pahio}{2872}
\pmtype{Definition}
\pmcomment{trigger rebuild}
\pmclassification{msc}{51M04}
\pmclassification{msc}{51F20}
\pmsynonym{supplementary}{SupplementaryAngles}
\pmrelated{Supplement}
\pmrelated{Angle}
\pmrelated{ComplementaryAngles}
\pmrelated{GoniometricFormulae}

\endmetadata

% this is the default PlanetMath preamble.  as your knowledge
% of TeX increases, you will probably want to edit this, but
% it should be fine as is for beginners.

% almost certainly you want these
\usepackage{amssymb}
\usepackage{amsmath}
\usepackage{amsfonts}

% used for TeXing text within eps files
%\usepackage{psfrag}
% need this for including graphics (\includegraphics)
%\usepackage{graphicx}
% for neatly defining theorems and propositions
 \usepackage{amsthm}
% making logically defined graphics
%%%\usepackage{xypic}

% there are many more packages, add them here as you need them

% define commands here

\theoremstyle{definition}
\newtheorem*{thmplain}{Theorem}

\begin{document}
Two angles are called {\em supplementary angles}\, of each other if the sum of their \PMlinkname{measures}{AngleMeasure} is equal to the straight angle $\pi$, \PMlinkname{i.e.}{Ie} $180^\circ$.\\

For example, when two lines intersect each other, they \PMlinkescapetext{divide} the plane into four disjoint \PMlinkname{domains}{Domain2} corresponding to four convex angles; then any of these angles has a supplementary angle on either side of it (see linear pair).  However, two angles that are supplementary to each other do not need to have a common side --- see \PMlinkname{e.g.}{Eg} an entry regarding \PMlinkname{opposing angles in a cyclic quadrilateral}{OpposingAnglesInACyclicQuadrilateralAreSupplementary}.\\

Supplementary angles have always equal sines, but the cosines are opposite numbers:
$$\sin(\pi\!-\!\alpha) \;=\; \sin\alpha, \qquad \cos(\pi\!-\!\alpha) \;=\; -\cos\alpha$$
These formulae may be proved by using the subtraction formulas of sine and cosine.


%%%%%
%%%%%
\end{document}
