\documentclass[12pt]{article}
\usepackage{pmmeta}
\pmcanonicalname{PerspectiveDrawing}
\pmcreated{2013-03-22 15:41:10}
\pmmodified{2013-03-22 15:41:10}
\pmowner{stevecheng}{10074}
\pmmodifier{stevecheng}{10074}
\pmtitle{perspective drawing}
\pmrecord{12}{37628}
\pmprivacy{1}
\pmauthor{stevecheng}{10074}
\pmtype{Topic}
\pmcomment{trigger rebuild}
\pmclassification{msc}{51N20}
\pmclassification{msc}{15A90}
%\pmkeywords{orthographic projection}
%\pmkeywords{perspective}

\endmetadata

\usepackage{amssymb}
\usepackage{amsmath}
\usepackage{amsfonts}
%\usepackage{amsthm}
%\usepackage{enumerate}

% used for TeXing text within eps files
%\usepackage{psfrag}
% need this for including graphics (\includegraphics)
\usepackage{graphicx}
% making logically defined graphics
%%%\usepackage{xypic}

% define commands here
\newcommand{\real}{\mathbb{R}}

\providecommand{\abs}[1]{\lvert#1\rvert}
\providecommand{\absW}[1]{\left\lvert#1\right\rvert}
\providecommand{\absB}[1]{\Bigl\lvert#1\Bigr\rvert}
\providecommand{\norm}[1]{\lVert#1\rVert}
\providecommand{\normW}[1]{\left\lVert#1\right\rVert}
\providecommand{\normB}[1]{\Bigl\lVert#1\Bigr\rVert}
\providecommand{\defnterm}[1]{\emph{#1}}

\providecommand{\pe}{\mathbf{e}}
\providecommand{\vg}{\mathbf{g}}
\providecommand{\vu}{\mathbf{u}}
\providecommand{\vv}{\mathbf{v}}
\providecommand{\po}{\mathbf{o}}
\providecommand{\pp}{\mathbf{p}}
\providecommand{\pq}{\mathbf{q}}
\providecommand{\transpose}[1]{#1^{\mathrm{t}}}
\begin{document}
\tableofcontents

\section{Introduction}

Perspective drawing refers to a particular technique 
of projecting a three-dimensional scene ---
mathemetically, some subset of $\real^3$ --- onto a subset of $\real^2$.
The aim is to make the scene look ``natural''.

The most common model for perspective drawing consists of an ideal eye 
or focus at a point $\pe \in \real^3$,
located behind a planar screen that is perpendicular 
to the unit vector $\vg \in \real^3$,
representing the direction of gaze.


To project a point $\pp \in \real^3$ in front of the screen
onto a point $\pq \in \real^3$ on 
the screen, we find the intersection, with the screen,
of the light ray emitted from the point $\pp$ towards the eye $\pe$:

\begin{center}
\includegraphics{model.1.eps}
\end{center}

\section{Basic equation for the projection}
Let $\po \in \real^3$ be a point on the screen.
Without loss of generality, assume that $\pe - \po$ is anti-parallel to
$\vg$, as in the previous picture.  
(We can always move the point $\po$ on the screen.)
This means that $\po$ will be the origin of the perspective drawing.

To calculate the projected point $\pq$ explicitly, 
we can solve these equations for $t \in [0,1]$:
\begin{align*}
\pq &= (1-t) \pp + t\pe & \text{($\pq$ is on the ray from $\pp$ to $\pe$)} \\
0 &= \langle \pq - \po , \vg \rangle & 
\text{($\pq$ is on plane centered at $\po$ perpendicular to $\vg$)}
\end{align*}
(The notation $\langle, \rangle$ denotes the dot product in $\real^3$.)
The solution is readily found to be:
\[
t = \frac{\langle \pp - \po, \vg \rangle}{\langle \pp - \pe, \vg \rangle}
= \frac{\text{horiz. distance of $\pp$ to the screen}}{\text{horiz. distance of $\pp$ to the eye}}\,.
\]
This $t$ satisfies $0 \leq t < 1$, because the 
horizontal distance of the point $\pp$ to the screen is 
less than its horizontal distance to the eye.

From the expression for $t$, we can determine $\pq$:
\begin{equation}
\label{eq:projection}
\pq = \frac{
\langle \pp - \po, \vg \rangle \pe
-\langle \pe - \po, \vg \rangle \pp  
}
{\langle \pp - \pe, \vg \rangle}\,.
\end{equation}

\section{Isomorphism of the planar screen to $\real^2$}

Of course, we are interested in displaying or drawing the projected point $\pq$
on a flat screen, so we need to assign coordinates to the screen.
In mathematical terms, we seek an (affine) isomorphism to $\real^2$,
of the plane centered at $\po$ and perpendicular to $\vg$.

The isomorphism is, obviously, not unique in general.
But it is uniquely determined if we impose these reasonable conditions:
\begin{itemize}
\item
Firstly, Euclidean distances on the screen should be preserved
when it is transformed into $\real^2$.
\item
And secondly, if we assign an ``up direction'' $\vv \in \real^3$,
then this direction should correspond with the 
unit vector $(0, 1)$ on $\real^2$.
\item
Similarly, if $\vu \in \real^3$ is a ``right''-pointing vector,
satisfying
$\vu \times \vv = -\vg$,
then $\vu$ should map to $(1, 0)$ on $\real^2$.
(The minus sign is there to preserve the right-hand rule:
the vector $\vu \times \vv$ should point towards the viewer,
which is exactly the opposite of the viewer's gaze direction.)
\end{itemize}

Let $\gamma$ be the right-handed orthonormal basis for $\real^3$
consisting of
the vectors $\vu$, $\vv$ and $-\vg$,
where $\vg$ is the gaze direction as before and $\vv$ is a given 
``up'' vector.  The vector $\vu$ can be determined by
\[
\vu = \vv \times -\vg = \vg \times \vv\,.
\]
\begin{center}
\includegraphics{basis.1.eps}
\end{center}

Then any point $\pq$ on the planar screen is to be mapped to the 
point $\pq' \in \real^2$, where
\begin{equation}
\label{eq:isomorphism}
(\pq', 0) = [\pq - \po]_\gamma\,,
\end{equation}
and $[\pq - \po]_\gamma$ is the representation of $(\pq - \po)$ 
in the basis $\gamma$.

\section{The perspective transform in eye coordinates}

The perspective projection 
\eqref{eq:projection}
takes a simple form in eye coordinates (coordinates 
with respect to the basis $\gamma$).
We derive it now.

Let $\pp_o = \pp - \po$, and $\pe_o = \pe - \po$.
Then equation \eqref{eq:projection} can be rewritten:
\begin{align*}
\pq &= 
\frac{
-\langle \pp_o, -\vg \rangle \pe
+\langle \pe_o, -\vg \rangle \pp  
}
{\langle \pe_o - \pp_o, -\vg \rangle}\,, \\
\pq - \po &=  
\frac{
-\langle \pp_o, -\vg \rangle \pe_o
+\langle \pe_o, -\vg \rangle \pp_o  
}
{\langle \pe_o - \pp_o, -\vg \rangle}\,.
\end{align*}

We can write out the last vector equation in $\gamma$ coordinates.
Let $[\pp_o]_\gamma = (x, y, z)$ and $[\pe_o]_\gamma = (0, 0, a)$ for some $a > 0$ (the distance of the eye from the screen).
Then
\begin{align*}
[\pq - \po]_\gamma &= \frac{-z(0, 0, a) + a(x,y,z)}{a - z} =
\frac{(ax, ay, 0)}{a -z} 
\\ &= 
\left( \frac{x}{1-z/a}, \, \frac{y}{1-z/a}, \, 0 \right)\,.
\end{align*}

Thus, according to equation \eqref{eq:isomorphism},
the perspective transform of $\pp$ onto $\real^2$
is given by
\begin{equation}
\label{eq:perspective-gamma}
\pq' = \left( \frac{x}{1-z/a}, \, \frac{y}{1-z/a}\right)\,.
\end{equation}

\section{Computing the perspective transform in arbitrary coordinates}

Finally, we want explicit expressions for the perspective transform
in terms of coordinates of an arbitrarily given orthonormal basis
$\beta$ for $\real^3$.  (For example, $\beta$ might be the standard basis.)

The matrix that changes
from $\gamma$ coordinates to $\beta$ coordinates
this matrix composed of three column vectors:
\[
[I]^\beta_\gamma = \begin{pmatrix}
[\vu]_\beta &
[\vv]_\beta &
[-\vg]_\beta
\end{pmatrix}\,.
\]
The matrix $[I]^\gamma_\beta$ that changes 
from $\beta$ coordinates to $\gamma$ coordinates
is the inverse to $[I]_\gamma^\beta$;
but both matrices are orthogonal,
so the inverse simplifies to the transpose:
\[
[I]_\beta^\gamma = 
\Bigl( [I]^\beta_\gamma \Bigr)^{-1} =
\transpose{\Bigl( [I]^\beta_\gamma \Bigr)}
= 
\begin{pmatrix}
\transpose{[\vu]_\beta} \\
\transpose{[\vv]_\beta} \\
\transpose{[-\vg]_\beta}
\end{pmatrix}\,.
\]

Therefore, to obtain the perspective transform of a given point $\pp$
given in $\beta$ coordinates, we are to compute the matrix product
\begin{equation}
\label{eq:perspective-beta}
\begin{pmatrix}
x \\
y \\
z 
\end{pmatrix} = [I]^\gamma_\beta  \cdot [\pp - \po]_\beta\,,
\end{equation}
and apply formula \eqref{eq:perspective-gamma} right afterwards.

\section{Example}

This figure was created using the MetaPost programming language.
MetaPost by itself has no facilities to produce three-dimensional graphics,
so the \PMlinktofile{source code for this drawing}{cube.mp} implements formulae \eqref{eq:perspective-gamma} and \eqref{eq:perspective-beta} directly.

\begin{center}
\includegraphics[scale=0.8]{cube.1.eps}
\end{center}

\section{The perspective transform in homogeneous coordinates}

The operation described by formula \eqref{eq:perspective-gamma} 
is not linear in $\pp$
or $\pp - \po$, so it cannot be represented by a $3\times 3$ matrix.
But using homogeneous coordinates, the perspective transform
can be represented by the following $4 \times 4$ matrix
(with respect to eye coordinates):
\begin{align*}
\begin{pmatrix}
1 & 0 & 0 & 0 \\
0 & 1 & 0 & 0 \\
0 & 0 & 0 & 0 \\
0 & 0 & -a^{-1} & 1
\end{pmatrix}
\text{ or }
\begin{pmatrix}
a & 0 & 0 & 0 \\
0 & a & 0 & 0 \\
0 & 0 & 0 & 0 \\
0 & 0 & -1 & a
\end{pmatrix}\,.
\end{align*}

\section{Properties of the perspective transform}

(To be written. Talk about the fact that lines are mapped to lines
by the perspective transform, and vanishing points.)

\section{Other models for three-dimensional drawing}

(To be written. Talk about orthographic projection (and relate it to
the special case where $a \to \infty$).  Could also mention 
the pin-hole camera model, or even more complicated models with non-infinitesimal lens.)
%%%%%
%%%%%
\end{document}
