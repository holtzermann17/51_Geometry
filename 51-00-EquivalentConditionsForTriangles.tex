\documentclass[12pt]{article}
\usepackage{pmmeta}
\pmcanonicalname{EquivalentConditionsForTriangles}
\pmcreated{2013-03-22 17:12:46}
\pmmodified{2013-03-22 17:12:46}
\pmowner{Wkbj79}{1863}
\pmmodifier{Wkbj79}{1863}
\pmtitle{equivalent conditions for triangles}
\pmrecord{10}{39536}
\pmprivacy{1}
\pmauthor{Wkbj79}{1863}
\pmtype{Theorem}
\pmcomment{trigger rebuild}
\pmclassification{msc}{51-00}
\pmrelated{Triangle}
\pmrelated{IsoscelesTriangle}
\pmrelated{EquilateralTriangle}
\pmrelated{EquiangularTriangle}
\pmrelated{RegularTriangle}

\endmetadata

\usepackage{amssymb}
\usepackage{amsmath}
\usepackage{amsfonts}
\usepackage{pstricks}
\usepackage{psfrag}
\usepackage{graphicx}
\usepackage{amsthm}
%%\usepackage{xypic}
\newtheorem{thm*}{Theorem}

\begin{document}
\PMlinkescapeword{equilateral}
\PMlinkescapeword{equiangular}

The following theorem holds in Euclidean geometry, hyperbolic geometry, and spherical geometry:

\begin{thm*}
Let $\triangle ABC$ be a triangle.  Then the following are equivalent:

\begin{itemize}
\item $\triangle ABC$ is \PMlinkname{equilateral}{EquilateralTriangle};
\item $\triangle ABC$ is \PMlinkname{equiangular}{EquiangularTriangle};
\item $\triangle ABC$ is \PMlinkname{regular}{RegularTriangle}.
\end{itemize}

\end{thm*}

Note that this statement does not generalize to any polygon with more than three sides in any of the indicated geometries.

\begin{proof}
It suffices to show that $\triangle ABC$ is equilateral if and only if it is equiangular.

Sufficiency:  Assume that $\triangle ABC$ is equilateral.

\begin{center}
\begin{pspicture}(-0.2,-0.2)(5.2,5.2)
\pspolygon(0,0)(5,0)(2.5,4.33)
\rput[b](2.5,4.5){$A$}
\rput[a](0,-0.2){$B$}
\rput[a](5,-0.2){$C$}
\psline(2.5,-0.2)(2.5,0.2)
\psline(1.15,2.2)(1.35,2.1)
\psline(3.65,2.1)(3.85,2.2)
\end{pspicture}
\end{center}

Since $\overline{AB} \cong \overline{AC} \cong \overline{BC}$, SSS yields that $\triangle ABC \cong \triangle BCA$.  By CPCTC, $\angle A \cong \angle B \cong \angle C$.  Hence, $\triangle ABC$ is equiangular.

Necessity:  Assume that $\triangle ABC$ is equiangular.

\begin{center}
\begin{pspicture}(-0.2,-0.2)(5.2,5.2)
\pspolygon(0,0)(5,0)(2.5,4.33)
\rput[b](2.5,4.5){$A$}
\rput[a](0,-0.2){$B$}
\rput[a](5,-0.2){$C$}
\psarc(0,0){0.5}{0}{60}
\psarc(5,0){0.5}{120}{180}
\psarc(2.5,4.33){0.5}{240}{300}
\end{pspicture}
\end{center}

By the theorem on determining from angles that a triangle is isosceles, we conclude that $\triangle ABC$ is isosceles with legs $\overline{AB} \cong \overline{AC}$ and that $\triangle BCA$ is isosceles with legs $\overline{AC} \cong \overline{BC}$.  Thus, $\overline{AB} \cong \overline{AC} \cong \overline{BC}$.  Hence, $\triangle ABC$ is equilateral.
\end{proof}
%%%%%
%%%%%
\end{document}
