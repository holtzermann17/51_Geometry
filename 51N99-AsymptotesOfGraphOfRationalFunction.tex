\documentclass[12pt]{article}
\usepackage{pmmeta}
\pmcanonicalname{AsymptotesOfGraphOfRationalFunction}
\pmcreated{2013-03-22 15:09:34}
\pmmodified{2013-03-22 15:09:34}
\pmowner{eshyvari}{13396}
\pmmodifier{eshyvari}{13396}
\pmtitle{asymptotes of graph of rational function}
\pmrecord{10}{36908}
\pmprivacy{1}
\pmauthor{eshyvari}{13396}
\pmtype{Result}
\pmcomment{trigger rebuild}
\pmclassification{msc}{51N99}
\pmclassification{msc}{26C15}
\pmclassification{msc}{26A09}
\pmrelated{Polytrope}

% this is the default PlanetMath preamble.  as your knowledge
% of TeX increases, you will probably want to edit this, but
% it should be fine as is for beginners.

% almost certainly you want these
\usepackage{amssymb}
\usepackage{amsmath}
\usepackage{amsfonts}

% used for TeXing text within eps files
%\usepackage{psfrag}
% need this for including graphics (\includegraphics)
%\usepackage{graphicx}
% for neatly defining theorems and propositions
 \usepackage{amsthm}
% making logically defined graphics
%%%\usepackage{xypic}

% there are many more packages, add them here as you need them

% define commands here

\theoremstyle{definition}
\newtheorem*{thmplain}{Theorem}
\begin{document}
Let \,$f(x) \,=\, \frac{P(x)}{Q(x)}$\, be a fractional expression where $P(x)$ and $Q(x)$ are polynomials with real coefficients such that their quotient can not be \PMlinkname{reduced}{Division} to a polynomial. \,We suppose that $P(x)$ and $Q(x)$ have no common zeros.

If the division of the polynomials is performed, then a result of the form
               $$f(x) \;=\; H(x)+\frac{R(x)}{Q(x)}$$
is gotten, where $H(x)$ and $R(x)$ are polynomials such that 
                 $$\deg{R(x)} < \deg{Q(x)}$$

The graph of the rational function $f$ may have asymptotes:
\begin{enumerate}
\item Every zero $a$ of the denominator $Q(x)$ gives a vertical asymptote 
 \,$x = a$.
\item If \,$\deg{H(x)} < 1$\, (i.e. $0$\, or\, $-\infty$) then the graph has the horizontal asymptote 
 \,$y = H(x)$.
\item If \,$\deg{H(x)} = 1$\, then the graph has the skew asymptote \,$y = H(x)$.
\end{enumerate}

{\em Proof of 2 and 3.} \,We have
 \,$\displaystyle f(x)\!-\!H(x) = \frac{R(x)}{Q(x)}\,\to 0$\;\, as \;\,$|x|\to\infty$.

\textbf{Remark.} \,Here we use the convention that the degree of the zero polynomial is \,$-\infty$.
%%%%%
%%%%%
\end{document}
