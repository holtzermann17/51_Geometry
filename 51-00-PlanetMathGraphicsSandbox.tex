\documentclass[12pt]{article}
\usepackage{pmmeta}
\pmcanonicalname{PlanetMathGraphicsSandbox}
\pmcreated{2013-03-22 16:39:06}
\pmmodified{2013-03-22 16:39:06}
\pmowner{PrimeFan}{13766}
\pmmodifier{PrimeFan}{13766}
\pmtitle{PlanetMath graphics sandbox}
\pmrecord{27}{38856}
\pmprivacy{1}
\pmauthor{PrimeFan}{13766}
\pmtype{Data Structure}
\pmcomment{trigger rebuild}
\pmclassification{msc}{51-00}

% this is the default PlanetMath preamble.  as your knowledge
% of TeX increases, you will probably want to edit this, but
% it should be fine as is for beginners.

% almost certainly you want these
\usepackage{amssymb}
\usepackage{amsmath}
\usepackage{amsfonts}

% used for TeXing text within eps files
%\usepackage{psfrag}
% need this for including graphics (\includegraphics)

\usepackage{graphicx}

% for neatly defining theorems and propositions
%\usepackage{amsthm}
% making logically defined graphics
%%\usepackage{xypic}

% there are many more packages, add them here as you need them

% define commands here
% define commands here
\newcommand{\trefoil}{\xygraph{
 !{0;/r2.0pc/:}
 !P3"a"{~>{}}
 !P9"b"{~:{(1.3288,0):}~>{}}
 !P3"c"{~:{(2.5,0):}~>{}}
 !{\vunder~{"b2"}{"b1"}{"a1"}{"a3"}}
 !{\vcap~{"c1"}{"c1"}{"b4"}{"b2"}}
 !{\vunder~{"b5"}{"b4"}{"a2"}{"a1"}}
 !{\vcap~{"c2"}{"c2"}{"b7"}{"b5"}}
 !{\vunder~{"b8"}{"b7"}{"a3"}{"a2"}}
 !{\vcap~{"c3"}{"c3"}{"b1"}{"b8"}}
}}

\newcommand{\quatrefoil}{\xygraph{
 !{0;/r2.0pc/:}
 !P3"a"{~>{}}
 !P9"b"{~:{(1.3288,0):}~>{}}
 !P3"c"{~:{(2.5,0):}~>{}} [rrrr]
 !P3"d"{~>{}}
 !P9"e"{~:{(1.3288,0):}~>{}}
 !P3"f"{~:{(2.5,0):}~>{}} [rr]
 !{\vunder~{"b2"}{"b1"}{"a1"}{"a3"}}
 !{\vcap~{"c1"}{"c1"}{"b4"}{"b2"}}
 !{\vunder~{"b5"}{"b4"}{"a2"}{"a1"}}
 !{\vcap~{"c2"}{"c2"}{"b7"}{"b5"}}
 !{\vunder~{"b8"}{"b7"}{"a3"}{"a2"}}
 !{\huntwist~{"b8"}{"e7"}{"b1"}{"e5"}}
 !{\vover~{"e7"}{"e8"}{"d2"}{"d3"}}
 !{\vcap~{"f3"}{"f3"}{"e8"}{"e1"}}
 !{\vover~{"e1"}{"e2"}{"d3"}{"d1"}}
 !{\vcap~{"f1"}{"f1"}{"e2"}{"e4"}}
 !{\vover~{"e4"}{"e5"}{"d1"}{"d2"}}
}}


% suggested by user mps
\usepackage{pstricks}
\newcommand{\smallbox}{\psframe*(0,0)(0.9,0.9)}
\begin{document}
$$
\xymatrix{
1 & 2 \ar@{-}[d] & 3 \ar@{-}[r] \ar@{-}[d] \ar@{-}[dr] & 4 & 5 \ar@{-}[r] & 6 \ar@{-}[d] \\
& 7 & 8 & 9 & 10 \ar@{-}[r] & 12
}
$$

Some text, yada, yada, yada. $10\spadesuit~A\clubsuit~J\heartsuit$
% Then $x = y + 1$ holds.

Silver rectangle with a horizontal line plopped down somewhere.

\begin{pspicture*}(-1,-1)(15.142135623730950488,11)
\psset{unit=0.5cm}
\pspolygon(0,0)(14.142135623730950488,0)(14.142135623730950488,10)(0,10)
\psline[linestyle=dotted](10,0)(10,10)
\psline[linestyle=dotted](0,5)(15,5)
\end{pspicture*}

A Collatz tree of height 12.
\[\xymatrix{
96\ar[d] & 17\ar[dr] & 104\ar[d] & 106\ar[d] & 640\ar[d] & 672\ar[d] & 113\ar[dr] & 680\ar[d] & 682\ar[d] & 4096\ar[d] \\
48\ar[d] & & 52\ar[d] & 53\ar[dr] & 320\ar[d] & 336\ar[d] & & 340\ar[d] & 341\ar[dr] & 2048\ar[d] \\
24\ar[d] & & 26\ar[d] & & 160\ar[d] & 168\ar[d] & & 170\ar[d] & & 1024\ar[d] \\
12\ar[d] & & 13\ar[drr] & & 80\ar[d] & 84\ar[d] & & 85\ar[drr] & & 512\ar[d] \\
6\ar[d] & & & & 40\ar[d] & 42\ar[d] & & & & 256\ar[d] \\
3\ar[drrrr] & & & & 20\ar[d] & 21\ar[drrrr] & & & & 128\ar[d] \\
& & & & 10\ar[d] & & & & & 64\ar[d] \\
& & & & 5\ar[drrrrr] & & & & & 32\ar[d] \\
& & & & & & & & & 16\ar[d] \\
& & & & & & & & & 8\ar[d] \\
& & & & & & & & & 4\ar[d] \\
& & & & & & & & & 2\ar[d] \\
& & & & & & & & & 1
}\]


Suppose $K$ and $J$ are both the trefoil knot.

By one choice of segment deletion and reattachment, $K\#J$ is the quatrefoil knot.
%\quatrefoil source in quatrefoil.tex (also defined in header of this file)
\begin{center}
\begin{figure}[here]
\includegraphics{quatrefoil.epsi}
\caption{$K\#J$ is the quatrefoil knot}
\end{figure}
\end{center}
%%%%%
%%%%%
\end{document}
