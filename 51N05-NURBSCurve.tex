\documentclass[12pt]{article}
\usepackage{pmmeta}
\pmcanonicalname{NURBSCurve}
\pmcreated{2013-03-22 17:10:59}
\pmmodified{2013-03-22 17:10:59}
\pmowner{stitch}{17269}
\pmmodifier{stitch}{17269}
\pmtitle{NURBS curve}
\pmrecord{12}{39499}
\pmprivacy{1}
\pmauthor{stitch}{17269}
\pmtype{Definition}
\pmcomment{trigger rebuild}
\pmclassification{msc}{51N05}
\pmsynonym{nonuniform rational B-spline curve}{NURBSCurve}
\pmrelated{BezierCurve}
\pmrelated{BSpline}
\pmrelated{NURBSSurface}

% this is the default PlanetMath preamble.  as your knowledge
% of TeX increases, you will probably want to edit this, but
% it should be fine as is for beginners.

% almost certainly you want these
\usepackage{amssymb}
\usepackage{amsmath}
\usepackage{amsfonts}

% used for TeXing text within eps files
%\usepackage{psfrag}
% need this for including graphics (\includegraphics)
%\usepackage{graphicx}
% for neatly defining theorems and propositions
%\usepackage{amsthm}
% making logically defined graphics
%%%\usepackage{xypic}

% there are many more packages, add them here as you need them

% define commands here

\begin{document}
\section{Introduction}
A \emph{NURBS curve}, which is an acronym for \emph{Non-Uniform Rational B-Spline curve}, is a generalization of both \PMlinkname{B\'ezier}{BezierCurve} and \PMlinkname{BSpline}{B-splines} curves. NURBS are commonly used in computer graphics, computer-aided design (CAD), engineering (CAE), and manufacturing (CAM).

\section{Definition}
A NURBS curve is a parametric curve defined by its \PMlinkescapetext{degree}, a set of weighted control points, and a knot vector. It is defined as

\[
c(u) = \frac { \sum_{i=0}^{n} N_{i,p}(u) w_i P_i} {\sum_{i=0}^{n} N_{i,p}(u) w_i} \quad\quad 0 \leq u \leq 1
\]

where $u$ is the parameter, $p$ is the \PMlinkescapetext{degree}, $N_{i,p}$ are the B-spline basis functions, $P_i$ are the control points and $w_i$ are the weights.
%%%%%
%%%%%
\end{document}
