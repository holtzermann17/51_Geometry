\documentclass[12pt]{article}
\usepackage{pmmeta}
\pmcanonicalname{LengthsOfAngleBisectors}
\pmcreated{2013-03-22 18:26:50}
\pmmodified{2013-03-22 18:26:50}
\pmowner{pahio}{2872}
\pmmodifier{pahio}{2872}
\pmtitle{lengths of angle bisectors}
\pmrecord{9}{41108}
\pmprivacy{1}
\pmauthor{pahio}{2872}
\pmtype{Corollary}
\pmcomment{trigger rebuild}
\pmclassification{msc}{51M05}
\pmrelated{Incenter}
\pmrelated{AngleBisectorAsLocus}
\pmrelated{LengthsOfMedians}

% this is the default PlanetMath preamble.  as your knowledge
% of TeX increases, you will probably want to edit this, but
% it should be fine as is for beginners.

% almost certainly you want these
\usepackage{amssymb}
\usepackage{amsmath}
\usepackage{amsfonts}

% used for TeXing text within eps files
%\usepackage{psfrag}
% need this for including graphics (\includegraphics)
%\usepackage{graphicx}
% for neatly defining theorems and propositions
 \usepackage{amsthm}
% making logically defined graphics
%%%\usepackage{xypic}

% there are many more packages, add them here as you need them

% define commands here

\theoremstyle{definition}
\newtheorem*{thmplain}{Theorem}

\begin{document}
In any triangle, the \PMlinkescapetext{lengths} $w_a$, $w_b$, $w_c$ of the angle bisectors opposing the sides $a$, $b$, $c$, respectively, are
\begin{align}
w_a = \frac{\sqrt{bc\,[(b\!+\!c)^2\!-\!a^2]\,}}{b\!+\!c},
\end{align}
\begin{align}
w_b = \frac{\sqrt{ca\,[(c\!+\!a)^2\!-\!b^2]\,}}{c\!+\!a},
\end{align}
\begin{align}
w_c = \frac{\sqrt{ab\,[(a\!+\!b)^2\!-\!c^2]\,}}{a\!+\!b}.
\end{align}

{\em Proof.}\; By the symmetry, it suffices to prove only (1).

According the angle bisector theorem, the bisector $w_a$ divides the side $a$ into the portions 
$$\frac{b}{b\!+\!c}\cdot a \;=\; \frac{ab}{b\!+\!c}, \qquad \frac{c}{b\!+\!c}\cdot a \;=\; \frac{ca}{b\!+\!c}.$$
If the angle opposite to $a$ is $\alpha$, we apply the law of cosines to the half-triangles \PMlinkescapetext{separated} by $w_a$:
\begin{align}
\begin{cases}
2w_ab\cos\frac{\alpha}{2} \;=\; w_a^2\!+\!b^2\!-\!\left(\frac{ab}{b+c}\right)^2\\
2w_ac\cos\frac{\alpha}{2} \;=\; w_a^2\!+\!c^2\!-\!\left(\frac{ca}{b+c}\right)^2
\end{cases}
\end{align}
For eliminating the angle $\alpha$, the equations (4) are divided sidewise, when one gets
$$\frac{b}{c} \;=\; \frac{w_a^2\!+\!b^2\!-\!\left(\frac{ab}{b+c}\right)^2}{w_a^2\!+\!c^2\!-\!\left(\frac{ca}{b+c}\right)^2},$$
from which one can after some routine manipulations solve $w_a$, and this can be simplified to the form (1).


%%%%%
%%%%%
\end{document}
