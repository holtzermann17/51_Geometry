\documentclass[12pt]{article}
\usepackage{pmmeta}
\pmcanonicalname{Adjacent1}
\pmcreated{2013-03-22 17:10:44}
\pmmodified{2013-03-22 17:10:44}
\pmowner{Wkbj79}{1863}
\pmmodifier{Wkbj79}{1863}
\pmtitle{adjacent}
\pmrecord{6}{39493}
\pmprivacy{1}
\pmauthor{Wkbj79}{1863}
\pmtype{Definition}
\pmcomment{trigger rebuild}
\pmclassification{msc}{51-00}
\pmsynonym{adjacent side}{Adjacent1}

\usepackage{amssymb}
\usepackage{amsmath}
\usepackage{amsfonts}
\usepackage{pstricks}
\usepackage{psfrag}
\usepackage{graphicx}
\usepackage{amsthm}
%%\usepackage{xypic}

\begin{document}
Let $\theta$ be an angle of a triangle.  A side of the triangle is \emph{adjacent} to $\theta$ if it is one of the \PMlinkname{sides}{Angle} of $\theta$.

\begin{center}
\begin{pspicture}(0,-1)(5,5)
\pspolygon(0,0)(5,0)(4,4)
\rput[b](2.5,0){adjacent}
\rput[r](1.8,2){adjacent}
\psarc(0,0){0.3}{0}{45}
\rput[b](0.5,0.15){$\theta$}
\rput[l](0,0){.}
\rput[r](5,0){.}
\rput[a](4,4){.}
\end{pspicture}
\end{center}

When a phrase such as ``adjacent of an angle'' is used, one must determine from context whether it refers to this definition of adjacent or the other definition of \PMlinkname{adjacent}{Adjacent2}.  Note that the latter is specifically for right triangles.
%%%%%
%%%%%
\end{document}
