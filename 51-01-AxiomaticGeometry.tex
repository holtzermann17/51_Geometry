\documentclass[12pt]{article}
\usepackage{pmmeta}
\pmcanonicalname{AxiomaticGeometry}
\pmcreated{2013-03-22 18:00:27}
\pmmodified{2013-03-22 18:00:27}
\pmowner{CWoo}{3771}
\pmmodifier{CWoo}{3771}
\pmtitle{axiomatic geometry}
\pmrecord{6}{40521}
\pmprivacy{1}
\pmauthor{CWoo}{3771}
\pmtype{Topic}
\pmcomment{trigger rebuild}
\pmclassification{msc}{51-01}
\pmclassification{msc}{51-00}
\pmrelated{EuclideanAxiomByHilbert}

\usepackage{amssymb,amscd}
\usepackage{amsmath}
\usepackage{amsfonts}
\usepackage{mathrsfs}

% used for TeXing text within eps files
%\usepackage{psfrag}
% need this for including graphics (\includegraphics)
%\usepackage{graphicx}
% for neatly defining theorems and propositions
\usepackage{amsthm}
% making logically defined graphics
%%\usepackage{xypic}
\usepackage{pst-plot}

% define commands here
\newcommand*{\abs}[1]{\left\lvert #1\right\rvert}
\newtheorem{prop}{Proposition}
\newtheorem{thm}{Theorem}
\newtheorem{ex}{Example}
\newcommand{\real}{\mathbb{R}}
\newcommand{\pdiff}[2]{\frac{\partial #1}{\partial #2}}
\newcommand{\mpdiff}[3]{\frac{\partial^#1 #2}{\partial #3^#1}}
\begin{document}
Axiomatic geometry can be traced back to the time of Euclid.  In his book
\emph{Elements}, written back in the 300's B.C., Euclid gave five rules, or postulates, describing how
points, lines, line segments, etc behave as they are ordinarily
perceived.  Based on these postulates, he set out to prove hundreds of 
properties.  Today, these properties are under the field of study known as 
\emph{plane Euclidean geometry}, more popularly known as high school geometry.  
The systematic and axiomatic approach to proving geometric facts is what makes his \emph{Elements} one of the most important contributions to mathematics.

One key feature of Euclid's axioms is the abundance of what are known today as 
the \emph{undefined terms}.  Geometric notions such as points, lines, and 
circles are mentioned in his axioms but never clearly defined.  For example, Euclid 
called a ``point'' as ``that which has no part''.  But what the meaning of ``part'' was 
never clarified.  It is because of this abundance of undefined terms, Euclid's postulates by 
today's mathematical standards lack rigor.  While some undefined terms give no serious 
problems, others create holes in proofs which are unacceptable.  In the late 19th century, 
David Hilbert published his classic \emph{Foundations of Geometry}, putting Euclid's postulates 
on a more solid ground.  In the book, he broke down Euclid's postulates into five
groups of axioms:
\begin{enumerate}
\item incidence axioms
\item order axioms
\item congruence axioms
\item continuity axiom
\item \PMlinkname{parallel axiom}{ParallelPostulate}
\end{enumerate}
These axioms have been shown to be independent of each other, in the sense that no one axiom can be 
proved from the rest, and consistent, in the sense that no contradictions can be derived from them.  These 
axioms today serve as the foundation of plane Euclidean geometry.

Since Hilbert's work, the natural next step is to look at geometries that are not Euclidean.  In other words, 
geometric models that lack one or several of the ``Euclidean axioms'' above.  The result has been the many 
exotic geometries that, surprisingly, have found applications in other places.
%%%%%
%%%%%
\end{document}
