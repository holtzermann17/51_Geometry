\documentclass[12pt]{article}
\usepackage{pmmeta}
\pmcanonicalname{TangentOfCircle}
\pmcreated{2013-03-22 17:35:58}
\pmmodified{2013-03-22 17:35:58}
\pmowner{pahio}{2872}
\pmmodifier{pahio}{2872}
\pmtitle{tangent of circle}
\pmrecord{7}{40013}
\pmprivacy{1}
\pmauthor{pahio}{2872}
\pmtype{Definition}
\pmcomment{trigger rebuild}
\pmclassification{msc}{51-00}
\pmclassification{msc}{51M15}
%\pmkeywords{tangent line}
\pmrelated{TangentOfConicSection}
\pmrelated{ConjugateDiametersOfEllipse}

% this is the default PlanetMath preamble.  as your knowledge
% of TeX increases, you will probably want to edit this, but
% it should be fine as is for beginners.

% almost certainly you want these
\usepackage{amssymb}
\usepackage{amsmath}
\usepackage{amsfonts}

% used for TeXing text within eps files
%\usepackage{psfrag}
% need this for including graphics (\includegraphics)
%\usepackage{graphicx}
% for neatly defining theorems and propositions
 \usepackage{amsthm}
% making logically defined graphics
%%%\usepackage{xypic}
\usepackage{pstricks}
\usepackage{pst-plot}

% there are many more packages, add them here as you need them

% define commands here

\theoremstyle{definition}
\newtheorem*{thmplain}{Theorem}

\begin{document}
\PMlinkescapeword{tangent}
The tangent of \PMlinkescapetext{circle} may be defined by using whatever of the below three characterisations.\\

The \PMlinkescapetext{{\em tangent of a circle}} is a line
\begin{itemize}
\item whose distance from the centre of the circle is equal to the radius of the circle;
\item that is perpendicular to a radius of the circle and passes through the end point of the radius lying on the circumference;
\item that has only one common point with the curve.
\end{itemize}

\textbf{Note.}  The tangent of ellipse may also be defined by using the third characterisation.

\begin{center}
\begin{pspicture}(-5.5,-4.5)(5.5,4)
\psdot(0,0)
\pscircle(0,0){2}
\psline[linecolor=blue](-0.5,2.875)(4,-0.5)
\psline(0,0)(1.2,1.6)
\psline(1.05,1.45)(1.25,1.31)
\psline(1.25,1.31)(1.35,1.45)
\psdot[linecolor=blue](1.2,1.6)
\end{pspicture}
\end{center}

%%%%%
%%%%%
\end{document}
