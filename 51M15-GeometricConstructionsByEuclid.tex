\documentclass[12pt]{article}
\usepackage{pmmeta}
\pmcanonicalname{GeometricConstructionsByEuclid}
\pmcreated{2013-03-22 17:12:32}
\pmmodified{2013-03-22 17:12:32}
\pmowner{pahio}{2872}
\pmmodifier{pahio}{2872}
\pmtitle{geometric constructions by Euclid}
\pmrecord{20}{39531}
\pmprivacy{1}
\pmauthor{pahio}{2872}
\pmtype{Topic}
\pmcomment{trigger rebuild}
\pmclassification{msc}{51M15}
\pmclassification{msc}{51-00}
\pmsynonym{geometric construction}{GeometricConstructionsByEuclid}
\pmdefines{QEF}
\pmdefines{Q.E.F.}

\endmetadata

% this is the default PlanetMath preamble.  as your knowledge
% of TeX increases, you will probably want to edit this, but
% it should be fine as is for beginners.

% almost certainly you want these
\usepackage{amssymb}
\usepackage{amsmath}
\usepackage{amsfonts}

% used for TeXing text within eps files
%\usepackage{psfrag}
% need this for including graphics (\includegraphics)
%\usepackage{graphicx}
% for neatly defining theorems and propositions
 \usepackage{amsthm}
% making logically defined graphics
%%%\usepackage{xypic}

% there are many more packages, add them here as you need them

% define commands here

\theoremstyle{definition}
\newtheorem*{thmplain}{Theorem}

\begin{document}
The \emph{geometric constructions} using compass and straightedge consist of three \PMlinkescapetext{simple} fundamental tasks as given in  Euclid's {\em The Elements} (in ancient Greek $\Sigma\tau{o}\iota\chi\varepsilon\acute\iota\alpha$, transliterated {\em Stoikheia}). These fundamental tasks are as follows:

\begin{enumerate}
\item Drawing a line through two given points.
\item Drawing a circle having a given point as its center and passing through another given point.
\item Setting a plane passing through three given non-collinear points, where one performs tasks based on the two preceding tasks.
\end{enumerate}


\textbf{Example.}\, The usual task of drawing a circle with a given point as its center and with a given line segment as its radius (a fundamental task in many textbooks) can be \PMlinkescapetext{reduced} to Euclid's fundamental tasks (one needs five circles!).\\

\textbf{Remark.}\, It can be proven that all geometric constructions with compass and straightedge are possible using \emph{only} the compass.  (See \PMlinkname{e.g.}{Eg} compass and straightedge construction of parallel line.)\\

In the text of Euclid, the constructions are not listed separately, but are combined with the theorems as
propositions.  A way to tell whether a proposition is a theorem or a construction is to go to the end of
the proof and see if it ends with QED, in which case it is a theorem, or with QEF, in which case it 
is a construction.  Note that QEF is an abbreviation for the Latin phrase \emph{quod erat faciendum}, meaning `which was to be done'.\\

Here is a list of the geometric constructions to be found in \emph{The Elements}:

\begin{itemize}
\item I 1 Given a line segment, construct an equilateral triangle having that segment as a side.
\item I 2 Given a point and a line segment, construct a line segment having the given point as
an endpoint and equal in length to the given line segment.
\item I 3 Given two line segments, produce a line segment whose length is the difference of
the lengths of the two given line segments.
\item I 9 Bisect a given angle.
\item I 10 Bisect a given line segment.
\item I 11 Given a line and a point on this line, construct a line orthogonal to the given
line passing through the given point.
\item I 12 Given a line and a point not on this line, construct a line orthogonal to the given
line passing through the given point.  (i.e. Find the projection of a point on a line.)
\item III 1 Construct the center of a given circle.
\end{itemize}

If you are interested in seeing the rules for compass and straightedge constructions, click on the \PMlinkescapetext{link} provided.

\textbf{References}

\PMlinkexternal{Online edition}{http://www.physics.ntua.gr/Faculty/mourmouras/euclid/} of Euclid's \emph{The Elements} 
in Greek prepared by D. E. Mourmouras.
%%%%%
%%%%%
\end{document}
