\documentclass[12pt]{article}
\usepackage{pmmeta}
\pmcanonicalname{ProofOfSteinersTheorem}
\pmcreated{2013-03-22 13:48:06}
\pmmodified{2013-03-22 13:48:06}
\pmowner{mathcam}{2727}
\pmmodifier{mathcam}{2727}
\pmtitle{proof of Steiner's theorem}
\pmrecord{5}{34520}
\pmprivacy{1}
\pmauthor{mathcam}{2727}
\pmtype{Proof}
\pmcomment{trigger rebuild}
\pmclassification{msc}{51N20}

\endmetadata

\usepackage{amssymb}
\usepackage{amsmath}
\usepackage{amsfonts}
\usepackage{graphicx}
\begin{document}
\includegraphics{steiner}

Using $\alpha,\beta,\gamma,\delta$ to denote angles as in the diagram
at left, the sines law yields
\begin{eqnarray}
\frac{AB}{\sin(\gamma)}&=&\frac{AC}{\sin(\beta)} \\
\frac{NB}{\sin(\alpha+\delta)}&=&\frac{NA}{\sin(\beta)} \\
\frac{MC}{\sin(\alpha+\delta)}&=&\frac{MA}{\sin(\gamma)} \\
\frac{MB}{\sin(\alpha)}&=&\frac{MA}{\sin(\beta)} \\
\frac{NC}{\sin(\alpha)}&=&\frac{NA}{\sin(\gamma)}
\end{eqnarray}
Dividing (2) and (3), and (4) by (5):
$$\frac{MA}{NA}\frac{NB}{MC}=\frac{\sin(\gamma)}{\sin(\beta)}=\frac{NA}{MA}\frac{MB}{NC}$$
and therefore
$$\frac{NB\cdot MB}{MC\cdot NC}=\frac{\sin^2(\gamma)}{\sin^2(\beta)}
=\frac{AB^2}{AC^2}
$$
by (1).
%%%%%
%%%%%
\end{document}
