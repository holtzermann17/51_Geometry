\documentclass[12pt]{article}
\usepackage{pmmeta}
\pmcanonicalname{StraightLineIsShortestCurveBetweenTwoPoints}
\pmcreated{2013-03-22 15:39:43}
\pmmodified{2013-03-22 15:39:43}
\pmowner{stevecheng}{10074}
\pmmodifier{stevecheng}{10074}
\pmtitle{straight line is shortest curve between two points}
\pmrecord{11}{37595}
\pmprivacy{1}
\pmauthor{stevecheng}{10074}
\pmtype{Result}
\pmcomment{trigger rebuild}
\pmclassification{msc}{51N05}
\pmrelated{ArcLength}
\pmrelated{Rectifiable}

\usepackage{amssymb}
\usepackage{amsmath}
\usepackage{amsfonts}
\usepackage{amsthm}

% used for TeXing text within eps files
%\usepackage{psfrag}
% need this for including graphics (\includegraphics)
\usepackage{graphicx}
% making logically defined graphics
%%%\usepackage{xypic}

% define commands here
\newcommand{\real}{\mathbb{R}}

\providecommand{\abs}[1]{\lvert#1\rvert}
\providecommand{\absW}[1]{\left\lvert#1\right\rvert}
\providecommand{\absB}[1]{\Bigl\lvert#1\Bigr\rvert}
\providecommand{\norm}[1]{\lVert#1\rVert}
\providecommand{\normW}[1]{\left\lVert#1\right\rVert}
\providecommand{\normB}[1]{\Bigl\lVert#1\Bigr\rVert}
\begin{document}
\PMlinkescapeword{length}

Suppose $p$ and $q$ are two distinct points in $\real^n$, 
and $\gamma$ is a rectifiable curve from $p$ to $q$.
Then every curve other than the straight line segment from $p$ to $q$
has a length greater than the Euclidean distance $\norm{p -q}$.

\begin{proof}
Let $\gamma\colon [0,1] \to  \real^n$
be the curve with length $L$.
If it is not straight\footnote{If $\gamma$ is a straight line segment but is not injective, that is, it moves \PMlinkescapetext{back and forth between} $p$ and $q$, then it is obvious that $L > \norm{p-q}$.}, then there exists a point $x = \gamma(t)$ that does not lie on the line segment from $p$ to $q$.
We have
\[
L \geq \norm{q - x} + \norm{x - p} > \norm{p -q }\,.
\]
The first inequality comes from the definition of $L$ as the least upper bound of the length of \emph{any} broken-line approximation to the curve $\gamma$.
The second inequality is the usual triangle inequality,
but it is a strict inequality since $x$ lies outside the line segment between $p$ and $q$,
as shown in the following diagram.
\end{proof}

\begin{figure}
\includegraphics{curve-triangle.eps}
\end{figure}

%%%%%
%%%%%
\end{document}
