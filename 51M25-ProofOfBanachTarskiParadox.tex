\documentclass[12pt]{article}
\usepackage{pmmeta}
\pmcanonicalname{ProofOfBanachTarskiParadox}
\pmcreated{2013-03-22 15:19:03}
\pmmodified{2013-03-22 15:19:03}
\pmowner{GrafZahl}{9234}
\pmmodifier{GrafZahl}{9234}
\pmtitle{proof of Banach-Tarski paradox}
\pmrecord{7}{37123}
\pmprivacy{1}
\pmauthor{GrafZahl}{9234}
\pmtype{Proof}
\pmcomment{trigger rebuild}
\pmclassification{msc}{51M25}
\pmclassification{msc}{03E25}
%\pmkeywords{Banach}
%\pmkeywords{Tarski}
\pmrelated{HausdorffParadox}

\endmetadata

% this is the default PlanetMath preamble.  as your knowledge
% of TeX increases, you will probably want to edit this, but
% it should be fine as is for beginners.

% almost certainly you want these
\usepackage{amssymb}
\usepackage{amsmath}
\usepackage{amsfonts}
\usepackage[latin1]{inputenc}

% used for TeXing text within eps files
%\usepackage{psfrag}
% need this for including graphics (\includegraphics)
%\usepackage{graphicx}
% for neatly defining theorems and propositions
\usepackage{amsthm}
% making logically defined graphics
%%%\usepackage{xypic}

% there are many more packages, add them here as you need them

% define commands here
\newcommand{\Bigcup}{\bigcup\limits}
\newcommand{\Prod}{\prod\limits}
\newcommand{\Sum}{\sum\limits}
\newcommand{\mbb}{\mathbb}
\newcommand{\mbf}{\mathbf}
\newcommand{\mc}{\mathcal}
\newcommand{\mmm}[9]{\left(\begin{array}{rrr}#1&#2&#3\\#4&#5&#6\\#7&#8&#9\end{array}\right)}
\newcommand{\mf}{\mathfrak}
\newcommand{\ol}{\overline}

% Math Operators/functions
\DeclareMathOperator{\Aut}{Aut}
\DeclareMathOperator{\Frob}{Frob}
\DeclareMathOperator{\cwe}{cwe}
\DeclareMathOperator{\id}{id}
\DeclareMathOperator{\we}{we}
\DeclareMathOperator{\wt}{wt}
\begin{document}
\PMlinkescapeword{decomposition}
\PMlinkescapeword{decompositions}
\newtheorem{thm}{Theorem}
We deal with some technicalities first, mainly concerning the
properties of equi-decomposability. We can then prove the paradox in a
clear and unencumbered line of argument: we show that, given two unit
balls $U$ and $U'$ with arbitrary origin, $U$ and $U\cup U'$ are
equi-decomposable, regardless whether $U$ and $U'$ are disjoint or
not. The original proof can be found in~\cite{BT}.

\subsection*{Technicalities}
\begin{thm}
\label{theo3}
Equi-decomposability gives rise to an equivalence relation on the
subsets of Euclidean space.
\end{thm}
\smallskip

\begin{proof}
The only non-obvious part is transitivity. So let $A$, $B$, $C$ be
sets such that $A$ and $B$ as well as $B$ and $C$ are
equi-decomposable. Then there exist disjoint decompositions
$A_1,\ldots A_n$ of $A$ and $B_1,\ldots B_n$ of $B$ such that $A_k$
and $B_k$ are congruent for $1\leq k\leq n$. Furthermore, there exist
disjoint decompositions $B_1',\ldots,B_m'$ of $B$ and $C_1,\ldots C_m$
of $C$ such that $B_l'$ and $C_l$ are congruent for $1\leq l\leq
m$. Define
\begin{equation*}
B_{k,l}:=B_k\cap B_l'\text{ for }1\leq k\leq n, 1\leq l\leq m.
\end{equation*}
Now if $A_k$ and $B_k$ are congruent via some isometry $\theta\colon
A_k\to B_k$, we obtain a disjoint decomposition of $A_k$ by setting
$A_{k,l}:=\theta^{-1}(B_{k,l})$. Likewise, if $B_l'$ and $C_l$ are
congruent via some isometry $\theta'\colon B_l'\to C_l$, we obtain a
disjoint decomposition of $C_l$ by setting
$C_{k,l}:=\theta'(B_{k,l})$. Clearly, the $A_{k,l}$ and $C_{k,l}$
define a disjoint decomposition of $A$ and $C$, respectively, into
$nm$ parts. By transitivity of congruence, $A_{k,l}$ and $C_{k,l}$ are
congruent for $1\leq k\leq n$ and $1\leq l\leq m$. Therefore, $A$ and
$C$ are equi-decomposable.
\end{proof}
\medskip

\begin{thm}
\label{theo4}
Given disjoint sets $A_1,,\ldots A_n$, $B_1,\ldots B_n$ such that
$A_k$ and $B_k$ are equi-decomposable for $1\leq k\leq n$, their
unions $A=\Bigcup_{k=1}^nA_k$ and $B=\Bigcup_{k=1}^nB_k$ are
equi-decomposable as well.
\end{thm}
\smallskip

\begin{proof}
By definition, there exists for every $k$, $1\leq k\leq n$ an integer
$l_k$ such that there are disjoint decompositions
\begin{equation*}
A_k=\Bigcup_{i=1}^{l_k}A_{k,i},\qquad B_k=\Bigcup_{i=1}^{l_k}B_{k,i}
\end{equation*}
such that $A_{k,i}$ and $B_{k,i}$ are congruent for $1\leq i\leq
l_k$. Rewriting $A$ and $B$ in the form
\begin{equation*}
A=\Bigcup_{k=1}^n\Bigcup_{i=1}^{l_k}A_{k,i},\qquad B=\Bigcup_{k=1}^n\Bigcup_{i=1}^{l_k}B_{k,i}
\end{equation*}
gives the result.
\end{proof}
\medskip

\begin{thm}
\label{cor7}
Let $A$, $B$, $C$ be sets such that $A$ and $B$ are equi-decomposable
and $C\subsetneq A$, then there exists $D\subsetneq B$ such that $C$
and $D$ are equi-decomposable.
\end{thm}
\smallskip

\begin{proof}
Let $A_1,\ldots,A_n$ and $B_1,\ldots B_n$ be disjoint decompositions of
$A$ and $B$, respectively, such that $A_k$ and $B_k$ are congruent via
an isometry $\theta_k\colon A_k\to B_k$ for all $1\leq k\leq n$. Let
$\theta\colon A\to B$ a map such that $\theta(x)=\theta_k(x)$ if $x\in
A_k$. Since the $A_k$ are disjoint, $\theta$ is well-defined
everywhere. Furthermore, $\theta$ is obviously bijective. Now set
$C_k:=C\cap A_k$ and define $D_k:=\theta_k(C_k)$, so that $C_k$ and
$D_k$ are congruent for $1\leq k\leq n$, so the disjoint union
$D:=D_1\cup\cdots\cup D_n$ and $C$ are equi-decomposable. By
construction, $\theta(C)=D$. Since $C$ is a proper subset of $A$ and
$\theta$ is bijective, $D$ is a proper subset of $B$.
\end{proof}
\medskip

\begin{thm}
\label{cor9}
Let $A$, $B$ and $C$ be sets such that $A$ and $C$ are
equi-decomposable and $A\subseteq B\subseteq C$. Then $B$ and $C$ are
equi-decomposable.
\end{thm}
\smallskip

\begin{proof}
Let $A_1,\ldots,A_n$ and $C_1,\ldots C_n$ disjoint decompositions of
$A$ and $C$, respectively, such that $A_k$ and $C_k$ are congruent via
an isometry $\theta_k$ for $1\leq k\leq n$. Like in the proof of
theorem~\ref{cor7}, let $\theta\colon A\to C$ be the well-defined,
bijective map such that $\theta(x)=\theta_k(x)$ if $x\in A_k$. Now,
for every $b\in B$, let $\mc{C}(b)$ be the intersection of all sets
$X\subseteq B$ satisfying
\begin{itemize}
\item$b\in X$,
\item for all $x\in X$, the preimage $\theta^{-1}(x)$ lies in $X$,
\item for all $x\in X\cap A$, the image $\theta(x)$ lies in $X$.
\end{itemize}
Let $b_1,b_2\in B$ such that $\mc{C}(b_1)$ and $\mc{C}(b_2)$ are not
disjoint. Then there is a $b\in\mc{C}(b_1)\cap\mc{C}(b_2)$ such that
$b=\theta^r(b_1)=\theta^s(b_2)$ for suitable integers $s$ and
$r$. Given $b'\in\mc{C}(b_1)$, we have
$b'=\theta^t(b_1)=\theta^{t+s-r}(b_2)$ for a suitable integer $t$,
that is $b'\in\mc{C}(b_2)$, so that
$\mc{C}(b_1)\subseteq\mc{C}(b_2)$. The reverse inclusion follows
likewise, and we see that for arbitrary $b_1,b_2\in B$ either
$\mc{C}(b_1)=\mc{C}(b_2)$ or $\mc{C}(b_1)$ and $\mc{C}(b_2)$ are
disjoint. Now set
\begin{equation*}
D:=\{b\in B\mid\mc{C}(b)\subseteq A\},
\end{equation*}
then obviously $D\subseteq A$. If for $b\in B$, $\mc{C}(b)$ consists
of the sequence of elements
$\ldots,\theta^{-1}(b),b,\theta(b),\ldots$ which is infinite in both
directions, then $b\in D$. If the sequence is infinite in only one
direction, but the final element lies in $A$, then $b\in D$ as well.
Let $E:=\theta(D)$ and $F:=B\setminus D$, then clearly $E\cup F\subseteq
C$.

Now let $c\in C$. If $c\notin E$, then $\theta^{-1}(c)\notin D$, so
$\mc{C}(\theta^{-1}(c))$ consists of a sequence
$\ldots,\theta^{-2}(c),\theta^{-1}(c),\ldots$ which is infinite in
only one direction and the final element does not lie in $A$. Now
$\theta^{-1}(c)\in\mc{C}(\theta^{-1}(c))$, but since $\theta^{-1}(c)$
\emph{does} lie in $A$, it is not the final element. Therefore the
subsequent element $c$ lies in $\mc{C}(\theta^{-1}(c))$, in particular
$c\in B$ and $\mc{C}(c)=\mc{C}(\theta^{-1}(c))\not\subseteq A$, so
$c\in B\setminus D=F$. It follows that $C=E\cup F$, and furthermore $E$
and $F$ are disjoint.

It now follows similarly as in the preceding proofs that $D$ and
$E=\theta(D)$ are equi-decomposable. By theorem~\ref{theo4}, $B=D\cup
F$ and $C=E\cup F$ are equi-decomposable.
\end{proof}
\medskip

\subsection*{The proof}
We may assume that the unit ball $U$ is centered at the origin, that
is $U=\mbb{B}_3$, while the other unit ball $U'$ has an arbitrary
origin.
Let $S$ be the surface of $U$, that is, the unit sphere. By the
Hausdorff paradox, there exists a disjoint decomposition
\begin{equation*}
S=B'\cup C'\cup D'\cup E'
\end{equation*}
such that $B'$, $C'$, $D'$ and $C'\cup D'$ are congruent, and $E'$ is
countable. For $r>0$ and $A\subseteq\mbb{R}^3$, let $rA$ be the set of
all vectors of $A$ multiplied by $r$. Set
\begin{equation*}
B:=\Bigcup_{0<r<1}rB',\quad C:=\Bigcup_{0<r<1}rC',\quad D:=\Bigcup_{0<r<1}rD',\quad E:=\Bigcup_{0<r<1}rE'.
\end{equation*}
These sets give a disjoint decomposition of the unit ball with the
origin deleted, and obviously $B$, $C$, $D$ and $C\cup D$
are congruent (but $E$ is of course \emph{not} countable). Set
\begin{equation*}
A_1:=B\cup E\cup\{0\}
\end{equation*}
where $0$ is the origin. $B$ and $C\cup D$ are trivially
equi-decomposable. Since $C$ and $B$ as well as $D$ and $C$ are
congruent, $C\cup D$ and $B\cup C$ are equi-decomposable. Finally, $B$
and $C\cup D$ as well as $C$ and $B$ are congruent, so $B\cup C$ and
$B\cup C\cup D$ are equi-decomposable. In total, $B$ and $B\cup C\cup
D$ are equi-decomposable by theorem~\ref{theo3}, so $A_1$ and $U$ are
equi-decomposable by theorem~\ref{theo4}. Similarly, we conclude that
$C$, $D$ and $B\cup C\cup D$ are equi-decomposable.

Since $E'$ is only countable but there are uncountably many rotations
of $S$, there exists a rotation $\theta$ such that
$\theta(E')\subsetneq B'\cup C'\cup D'$, so $F:=\theta(E)$ is a proper
subset of $B\cup C\cup D$. Since $C$ and $B\cup C\cup D$ are
equi-decomposable, there exists by theorem~\ref{cor7} a proper subset
$G\subsetneq C$ such that $G$ and $F$ (and thus $E$) are
equi-decomposable. Finally, let $p\in C\setminus G$ an arbitrary point
and set
\begin{equation*}
A_2:=D\cup G\cup\{p\},
\end{equation*}
a disjoint union by construction. Since $D$ and $B\cup C\cup D$, $G$
and $E$ as well as $\{0\}$ and $\{p\}$ are equi-decomposable, $A_2$
and $U$ are equi-decomposable by theorem~\ref{theo4}. $A_1$ and $A_2$
are disjoint (but $A_1\cup A_2\neq U$!).

Now $U$ and $U'$ are congruent, so $U'$ and $A_2$ are
equi-decomposable by theorem~\ref{theo3}. If $U$ and $U'$ are
disjoint, set $H:=A_2$. Otherwise we may use theorem~\ref{cor7} and
choose $H\subsetneq A_2$ such that $H$ and $U'\setminus U$ are
equi-decomposable. By theorem~\ref{theo4}, $A_1\cup H$ and $U\cup
(U'\setminus U)=U\cup U'$ are equi-decomposable. Now we have
\begin{equation*}
A_1\cup H\subseteq U\subseteq U\cup U',
\end{equation*}
so by theorem~\ref{cor9}, $U$ and $U\cup U'$ are equi-decomposable.

\begin{thebibliography}{BT}

\bibitem[BT]{BT} \textsc{St.~Banach, A.~Tarski}, Sur la d\'{e}composition
  des ensembles de points en parties respectivement congruentes,
  \emph{Fund.\ math.}\ 6, 244--277, (1924).

\end{thebibliography}
%%%%%
%%%%%
\end{document}
