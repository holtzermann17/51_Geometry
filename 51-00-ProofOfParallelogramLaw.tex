\documentclass[12pt]{article}
\usepackage{pmmeta}
\pmcanonicalname{ProofOfParallelogramLaw}
\pmcreated{2013-03-22 12:41:29}
\pmmodified{2013-03-22 12:41:29}
\pmowner{Mathprof}{13753}
\pmmodifier{Mathprof}{13753}
\pmtitle{proof of parallelogram law}
\pmrecord{5}{32971}
\pmprivacy{1}
\pmauthor{Mathprof}{13753}
\pmtype{Proof}
\pmcomment{trigger rebuild}
\pmclassification{msc}{51-00}
\pmrelated{ApolloniusTheorem}
\pmrelated{Median}
\pmrelated{ProofOfParallelogramLaw2}

\endmetadata

\usepackage{graphicx}
%%%\usepackage{xypic} 
\usepackage{bbm}
\newcommand{\Z}{\mathbbmss{Z}}
\newcommand{\C}{\mathbbmss{C}}
\newcommand{\R}{\mathbbmss{R}}
\newcommand{\Q}{\mathbbmss{Q}}
\newcommand{\mathbb}[1]{\mathbbmss{#1}}
\newcommand{\figura}[1]{\begin{center}\includegraphics{#1}\end{center}}
\newcommand{\figuraex}[2]{\begin{center}\includegraphics[#2]{#1}\end{center}}
\begin{document}
The proof follows directly from Apollonius theorem noticing that each diagonal 
is a median for the triangles in which parallelogram is split by the other diagonal.
 Also, the diagonals bisect each other.
\figura{parallelogramlaw}

Therefore, Apollonius theorem implies
$$2\left(\frac{d_1}{2}\right)^2 +\left(\frac{d_2}{2}\right)^2=u^2+v^2.$$
Multiplying both sides by $2$ and simplification leads to the desired expression.
%%%%%
%%%%%
\end{document}
