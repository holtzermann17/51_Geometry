\documentclass[12pt]{article}
\usepackage{pmmeta}
\pmcanonicalname{FermatTorricelliTheorem}
\pmcreated{2013-03-22 19:36:39}
\pmmodified{2013-03-22 19:36:39}
\pmowner{pahio}{2872}
\pmmodifier{pahio}{2872}
\pmtitle{Fermat$-$Torricelli theorem}
\pmrecord{16}{42604}
\pmprivacy{1}
\pmauthor{pahio}{2872}
\pmtype{Theorem}
\pmcomment{trigger rebuild}
\pmclassification{msc}{51M04}
\pmclassification{msc}{51F20}
\pmrelated{CenterOfATriangle}
\pmdefines{Fermat point}

\endmetadata

% this is the default PlanetMath preamble.  as your knowledge
% of TeX increases, you will probably want to edit this, but
% it should be fine as is for beginners.

% almost certainly you want these
\usepackage{amssymb}
\usepackage{amsmath}
\usepackage{amsfonts}

\usepackage{pstricks}
% used for TeXing text within eps files
%\usepackage{psfrag}
% need this for including graphics (\includegraphics)
%\usepackage{graphicx}
% for neatly defining theorems and propositions
 \usepackage{amsthm}
% making logically defined graphics
%%%\usepackage{xypic}

% there are many more packages, add them here as you need them

% define commands here

\theoremstyle{definition}
\newtheorem*{thmplain}{Theorem}

\begin{document}
\textbf{Theorem (Fermat$-$Torricelli).}\, Let all angles of a triangle $ABC$ be at most $120^\circ$.\, Then the inner point $F$ of the triangle which makes the sum $AF\!+\!BF\!+\!CF$ as little as possible, is the point from which the angle of view of every side is $120^\circ$.

{\it Proof.}\, Let's perform the rotation of $60^\circ$ about the point $A$.\, When $P$ is the image of the point $C$, the triangle $ACP$ is equilateral and its angles are $60^\circ$.\, Let $F$ be any inner point of the triangle $ABC$ and $Q$ its image in the rotation.\, We infer that if the sides of the triangle $ABC$ are all seen from $F$ in the angle $120^\circ$, then the points $B$, $F$, $Q$, $P$ lie on the same line.
\begin{center}
\begin{pspicture}(-1,-0.5)(4,3.5)
\pspolygon[linecolor=blue](0,0)(0,3)(3,1)
\rput(0,-0.25){$A$}
\rput(3,0.75){$B$}
\rput(0,3.25){$C$}
\psline(0,0)(-2.5,1.35)(0,3)
\psline(-2.5,1.35)(3,1)
\rput(-2.8,1.3){$P$}
\rput(0.75,0.85){$F$}
\psline(0,3)(0.7,1.15)(0,0)
\psdot(-0.7,1.23)
\rput(-0.55,1.5){$Q$}
\psline(0,0)(-0.7,1.2)
\psdot[linecolor=red](0.7,1.15)
\end{pspicture}
\end{center}
Generally, the triangles $APQ$ and $ACF$ are congruent, whence\, $CF = QP$.\, From the equilateral triangles we obtain:
$$AF\!+\!BF\!+\!CF \;=\; FQ\!+\!BF\!+\!QP \;=\; BFQP$$
Here, the right hand side is minimal when the points $B$, $F$, $Q$, $P$ are collinear, in which case 
\begin{align*}
&\angle CFA \;=\; \angle PQA \;=\; 180^\circ\!-\!\angle AQF \;=\; 120^\circ,\\
&\angle AFB \;=\; 180^\circ\!-\!\angle QFA \;=\; 120^\circ,\\
&\angle BFC \;=\; 360^\circ\!-\!240^\circ \;=\; 120^\circ.
\end{align*}
\begin{center}
\begin{pspicture}(-1,-0.5)(4,3.5)
\pspolygon[linecolor=blue](0,0)(0,3)(3,1)
\rput(0,-0.25){$A$}
\rput(3,0.75){$B$}
\rput(0,3.25){$C$}
\psline(0,0)(-2.5,1.35)(0,3)
\rput(-2.8,1.3){$P$}
\rput(1.05,1.65){$F$}
\psline(0,3)(0.8,1.45)(0,0)
\psline(-2.5,1.35)(-0.7,1.45)(0.8,1.45)(3,1)
\psdot(-0.7,1.45)
\rput(-0.5,1.75){$Q$}
\psline(0,0)(-0.7,1.45)
\psdot[linecolor=red](0.8,1.45)
\end{pspicture}
\end{center}

\textbf{Remark.}\, The point $F$ is called the {\it Fermat point} of the triangle $ABC$.

\begin{thebibliography}{8}
\bibitem{th}{\sc Tero Harju}: {\em Geometria. Lyhyt kurssi}.\, Matematiikan laitos. Turun yliopisto, Turku (2007).
\end{thebibliography}

%%%%%
%%%%%
\end{document}
