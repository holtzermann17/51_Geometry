\documentclass[12pt]{article}
\usepackage{pmmeta}
\pmcanonicalname{ProofOfHadwigerFinslerInequality}
\pmcreated{2013-03-22 12:45:21}
\pmmodified{2013-03-22 12:45:21}
\pmowner{mathwizard}{128}
\pmmodifier{mathwizard}{128}
\pmtitle{proof of Hadwiger-Finsler inequality}
\pmrecord{5}{33062}
\pmprivacy{1}
\pmauthor{mathwizard}{128}
\pmtype{Proof}
\pmcomment{trigger rebuild}
\pmclassification{msc}{51M16}

\endmetadata

% this is the default PlanetMath preamble.  as your knowledge
% of TeX increases, you will probably want to edit this, but
% it should be fine as is for beginners.

% almost certainly you want these
\usepackage{amssymb}
\usepackage{amsmath}
\usepackage{amsfonts}

% used for TeXing text within eps files
%\usepackage{psfrag}
% need this for including graphics (\includegraphics)
%\usepackage{graphicx}
% for neatly defining theorems and propositions
%\usepackage{amsthm}
% making logically defined graphics
%%%\usepackage{xypic}

% there are many more packages, add them here as you need them

% define commands here
\begin{document}
From the cosines law we get:
$$a^2=b^2+c^2-2bc\cos\alpha,$$
$\alpha$ being the angle between $b$ and $c$. This can be transformed into:
$$a^2=(b-c)^2+2bc(1-\cos\alpha).$$
Since $A=\frac{1}{2}bc\sin\alpha$ we have:
$$a^2=(b-c)^2+4A\frac{1-\cos\alpha}{\sin\alpha}.$$
Now remember that
$$1-\cos\alpha=2\sin^2\frac{\alpha}{2}$$
and
$$\sin\alpha=2\sin\frac{\alpha}{2}\cos\frac{\alpha}{2}.$$
Using this we get:
$$a^2=(b-c)^2+4A\tan\frac{\alpha}{2}.$$
Doing this for all sides of the triangle and adding up we get:
$$a^2+b^2+c^2=(a-b)^2+(b-c)^2+(c-a)^2+4A\left(\tan\frac{\alpha}{2} +\tan\frac{\beta}{2}+\tan\frac{\gamma}{2}\right).$$
$\beta$ and $\gamma$ being the other angles of the triangle. Now since the halves of the triangle's angles are less than $\frac{\pi}{2}$ the function $\tan$ is convex we have:
$$\tan\frac{\alpha}{2}+\tan\frac{\beta}{2}+\tan\frac{\gamma}{2} \geq 3\tan\frac{\alpha+\beta+\gamma}{6} =3\tan\frac{\pi}{6}=\sqrt{3}.$$
Using this we get:
$$a^2+b^2+c^2\geq (a-b)^2+(b-c)^2+(c-a)^2+4A\sqrt{3}.$$
This is the Hadwiger-Finsler inequality. $\Box$
%%%%%
%%%%%
\end{document}
