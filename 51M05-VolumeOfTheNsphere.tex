\documentclass[12pt]{article}
\usepackage{pmmeta}
\pmcanonicalname{VolumeOfTheNsphere}
\pmcreated{2013-03-22 13:47:09}
\pmmodified{2013-03-22 13:47:09}
\pmowner{CWoo}{3771}
\pmmodifier{CWoo}{3771}
\pmtitle{volume of the $n$-sphere}
\pmrecord{10}{34496}
\pmprivacy{1}
\pmauthor{CWoo}{3771}
\pmtype{Derivation}
\pmcomment{trigger rebuild}
\pmclassification{msc}{51M05}
\pmrelated{AreaOfTheNSphere}

% this is the default PlanetMath preamble.  as your knowledge
% of TeX increases, you will probably want to edit this, but
% it should be fine as is for beginners.

% almost certainly you want these
\usepackage{amssymb}
\usepackage{amsmath}
\usepackage{amsfonts}

% used for TeXing text within eps files
%\usepackage{psfrag}
% need this for including graphics (\includegraphics)
%\usepackage{graphicx}
% for neatly defining theorems and propositions
%\usepackage{amsthm}
% making logically defined graphics
%%%\usepackage{xypic}

% there are many more packages, add them here as you need them

% define commands here
\def\sse{\subseteq}
\def\bigtimes{\mathop{\mbox{\Huge $\times$}}}
\def\impl{\Rightarrow}
\def\R{\mathbb{R}}
\begin{document}
The volume contained inside $S^n$, the $n$-sphere (or hypersphere), is
given by the integral
\[
  V(n) = \int_{\sum_{i=1}^{n+1}x_i^2\le1} d^{n+1} x.
\]
Going to polar coordinates ($r^2=\sum_{i=1}^{n+1}x_i^2$) this becomes
\[
  V(n) = \int_{S^n} d\Omega \int_0^1 r^{n}\, dr.
\]
The first integral is the integral over all solid angles subtended by the
sphere and is equal to its area
$A(n)=\frac{2\pi^{\frac{n+1}{2}}}{\Gamma\left(\frac{n+1}{2}\right)}$,
where $\Gamma(x)$ is the gamma function.
The second integral is elementary and evaluates to
$\int_0^1 r^{n}\, dr = 1/(n+1)$.

Finally, the volume is
\[
  V(n) = \frac{\pi^{\frac{n+1}{2}}}{\frac{n+1}{2}\Gamma\left(\frac{n+1}{2}\right)}
    = \frac{\pi^{\frac{n+1}{2}}}{\Gamma\left(\frac{n+3}{2}\right)}.
\]
If the sphere has radius $R$ instead of $1$, then the correct volume is
$V(n)R^{n+1}$.

Note that this formula works for $n\ge0$. The first few cases are
\begin{itemize}
  \item[$n=0$] $\Gamma(3/2)=\sqrt{\pi}/2$, hence $V(0)=2$ (this is the
    length of the interval $[-1,1]$ in $\R$);
  \item[$n=1$] $\Gamma(2)=1$, hence $V(1) = \pi$ (this is the familiar
    result for the area of the unit circle);
  \item[$n=2$] $\Gamma(5/2)=3\sqrt{\pi}/4$, hence $V(2) = 4\pi/3$
    (this is the familiar result for the volume of the \PMlinkescapetext{unit} sphere);
  \item[$n=3$] $\Gamma(3)=2$, hence $V(3) = \pi^2/2$.
\end{itemize}
%%%%%
%%%%%
\end{document}
