\documentclass[12pt]{article}
\usepackage{pmmeta}
\pmcanonicalname{TerminalRay}
\pmcreated{2013-03-22 16:06:11}
\pmmodified{2013-03-22 16:06:11}
\pmowner{Wkbj79}{1863}
\pmmodifier{Wkbj79}{1863}
\pmtitle{terminal ray}
\pmrecord{12}{38167}
\pmprivacy{1}
\pmauthor{Wkbj79}{1863}
\pmtype{Definition}
\pmcomment{trigger rebuild}
\pmclassification{msc}{51-01}
\pmrelated{Trigonometry}
\pmrelated{CyclometricFunctions}

\endmetadata

\usepackage{amssymb}
\usepackage{amsmath}
\usepackage{amsfonts}

\usepackage{amsthm}
\usepackage{pstricks}

\begin{document}
Let an angle whose \PMlinkescapetext{measure} in radians is $\theta$ be placed \PMlinkescapetext{onto} the Cartesian plane such that one of its rays $R_1$ corresponds to the nonnegative $x$ axis and one can go from the point $(1,0)$ to the point that is the intersection of the other ray $R_2$ of the angle with the circle $x^2+y^2=1$ by traveling exactly $\theta$ units on the circle.  (If $\theta$ is positive, the distance should be traveled counterclockwise; otherwise, the distance $|\theta|$ should be traveled clockwise.  Also, note that ``other ray'' is used quite loosely, as it may also correspond to the nonnegative $x$ axis also.)  Then $R_2$ is the \emph{terminal ray} of the angle.

The picture below shows the terminal ray $R_2$ of the angle $\displaystyle \theta=\frac{2\pi}{3}$.

\begin{center}
\begin{pspicture}(-2,-3)(5,2)
\psset{unit=0.8cm}
\psline{<->}(-2,0)(2,0)
\rput[b](2,-0.3){$x$}
\psline{<->}(0,-2)(0,2)
\rput[r](0,2){$y$}
\psline[linewidth=1.5pt]{->}(0,0)(2,0)
\rput[b](2,0.2){$R_1$}
\psline[linewidth=1.5pt]{->}(0,0)(-1.1,1.505)
\rput[b](-1,1.705){$R_2$}
\psarc(0,0){0.3}{0}{120}
\rput[b](0.3,0.3){$\theta$}
\pscircle(0,0){1.3}
\rput[b](-1.35,-1.9){$x^2+y^2=1$}
\end{pspicture}
\end{center}
%%%%%
%%%%%
\end{document}
