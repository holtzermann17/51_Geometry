\documentclass[12pt]{article}
\usepackage{pmmeta}
\pmcanonicalname{CrossRatio}
\pmcreated{2013-03-22 15:23:31}
\pmmodified{2013-03-22 15:23:31}
\pmowner{rspuzio}{6075}
\pmmodifier{rspuzio}{6075}
\pmtitle{cross ratio}
\pmrecord{8}{37223}
\pmprivacy{1}
\pmauthor{rspuzio}{6075}
\pmtype{Definition}
\pmcomment{trigger rebuild}
\pmclassification{msc}{51N25}
\pmclassification{msc}{30C20}
\pmclassification{msc}{30F40}
\pmsynonym{cross-ratio}{CrossRatio}
\pmrelated{MobiusTransformationCrossRatioPreservationTheorem}

\endmetadata

% this is the default PlanetMath preamble.  as your knowledge
% of TeX increases, you will probably want to edit this, but
% it should be fine as is for beginners.

% almost certainly you want these
\usepackage{amssymb}
\usepackage{amsmath}
\usepackage{amsfonts}

% used for TeXing text within eps files
%\usepackage{psfrag}
% need this for including graphics (\includegraphics)
%\usepackage{graphicx}
% for neatly defining theorems and propositions
\usepackage{amsthm}
% making logically defined graphics
%%%\usepackage{xypic}

% there are many more packages, add them here as you need them

% define commands here
\theoremstyle{definition}
\newtheorem{example}{Example}
\begin{document}
The \emph{cross ratio} of the points $a$, $b$, $c$, and $d$ in $\mathbb{C}\cup\{\infty\}$ is denoted by $[a, b, c, d\,]$ and is defined by
\[
  [a, b, c, d\,] = \frac{a-c}{a-d}\cdot\frac{b-d}{b-c}.
\]
Some authors denote the cross ratio by $(a, b, c, d)$.

\section*{Examples}

\begin{example}
The cross ratio of $1$, $i$, $-1$, and $-i$ is
\[
\frac{1-(-1)}{1-(-i)}\cdot\frac{i-(-i)}{i-(-1)}
=\frac{4i}{(1+i)^2}=2.
\]
\end{example}

\begin{example}
The cross ratio of $1$, $2i$, $3$, and $4i$ is
\[
\frac{1-3}{1-4i}\cdot\frac{2i-4i}{2i-3}
=\frac{4i}{5+14i}
=\frac{56+20i}{221}.
\]
\end{example}

\section*{Properties}

\begin{enumerate}
\item
The cross ratio is invariant under M\"obius transformations and projective transformations.  This fact can be used to determine distances between objects in a photograph when the distance between certain reference points is known.

\item
The cross ratio $[a, b, c, d\,]$ is real if and only if $a$, $b$, $c$, and $d$ lie on a single circle on the Riemann sphere.

\item
The function $T:\mathbb{C}\cup\lbrace \infty \rbrace \to\mathbb{C}\cup\lbrace \infty\rbrace$ defined by
\[
T(z) = [z, b, c, d\,]
\]
is the unique M\"obius transformation which sends $b$ to $1$, $c$ to $0$, and $d$ to $\infty$.
\end{enumerate}

\begin{thebibliography}{1}
\bibitem{cite:A}
Ahlfors, L., \emph{Complex Analysis}.  McGraw-Hill, 1966. 
\bibitem{cite:B}
Beardon, A. F., \emph{The Geometry of Discrete Groups}.  Springer-Verlag, 1983.
\bibitem{cite:H}
Henle, M., \emph{Modern Geometries: Non-Euclidean, Projective, and Discrete}.  Prentice-Hall, 1997 [2001].
\end{thebibliography}
%%%%%
%%%%%
\end{document}
