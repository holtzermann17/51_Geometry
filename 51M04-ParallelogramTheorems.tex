\documentclass[12pt]{article}
\usepackage{pmmeta}
\pmcanonicalname{ParallelogramTheorems}
\pmcreated{2013-03-22 17:15:37}
\pmmodified{2013-03-22 17:15:37}
\pmowner{pahio}{2872}
\pmmodifier{pahio}{2872}
\pmtitle{parallelogram theorems}
\pmrecord{11}{39598}
\pmprivacy{1}
\pmauthor{pahio}{2872}
\pmtype{Theorem}
\pmcomment{trigger rebuild}
\pmclassification{msc}{51M04}
\pmclassification{msc}{51-01}
\pmsynonym{properties of parallelograms}{ParallelogramTheorems}
%\pmkeywords{alternate interior angles}
\pmrelated{Parallelogram}
\pmrelated{TriangleMidSegmentTheorem}

\endmetadata

% this is the default PlanetMath preamble.  as your knowledge
% of TeX increases, you will probably want to edit this, but
% it should be fine as is for beginners.

% almost certainly you want these
\usepackage{amssymb}
\usepackage{amsmath}
\usepackage{amsfonts}
\usepackage{pstricks}

% used for TeXing text within eps files
%\usepackage{psfrag}
% need this for including graphics (\includegraphics)
%\usepackage{graphicx}
% for neatly defining theorems and propositions
 \usepackage{amsthm}
% making logically defined graphics
%%\usepackage{xypic}

% there are many more packages, add them here as you need them

% define commands here

\theoremstyle{definition}
\newtheorem*{thmplain}{Theorem}

\begin{document}
\PMlinkescapeword{cuts}

\textbf{Theorem 1.}\, The opposite sides of a parallelogram are congruent.

\begin{proof}
\begin{center}
\begin{pspicture}(-2,-2)(4,2)
\pspolygon(-2,-1)(3,-1)(4,2)(-1,2)
\psline(-1,2)(3,-1)
\rput[r](-0.7,1.6){$\alpha$}
\rput[r](3,-0.55){$\beta$}
\rput[r](-0.3,1.8){$\gamma$}
\rput[r](2.45,-0.75){$\delta$}
\rput[l](-2.4,-1){$A$}
\rput[l](3.15,-1){$B$}
\rput[l](4.15,2){$C$}
\rput[r](-1.2,2){$D$}
\end{pspicture}
\end{center}

In the parallelogram $ABCD$, the line $BD$ as a transversal cuts the parallel lines $AD$ and $BC$, whence by the theorem of the \PMlinkname{parent entry}{CorrespondingAnglesInTransversalCutting} the alternate interior angles $\alpha$ and $\beta$ are congruent.  And since the line $BD$ also cuts the parallel lines $AB$ and $DC$, the alternate interior angles $\gamma$ and $\delta$ are congruent.  Moreover, the triangles $ABD$ and $CDB$ have a common side $BD$.  Thus, these triangles are congruent (ASA).  Accordingly, the corresponding sides are congruent:\, $AB = DC$\, and\, $AD = BC$. 
\end{proof}

\textbf{Theorem 2.}\, If both pairs of opposite sides of a quadrilateral are congruent, the quadrilateral is a parallelogram.

\textbf{Theorem 3.}\, If one pair of opposite sides of a quadrilateral are both parallel and congruent, the quadrilateral is a parallelogram.

\textbf{Theorem 4.}\, The diagonals of a parallelogram bisect each other.

\textbf{Theorem 5.}\, If the diagonals of a quadrilateral bisect each other, the quadrilateral is a parallelogram.

All of the above theorems hold in Euclidean geometry, but not in hyperbolic geometry.  These theorems do not \PMlinkescapetext{even} make sense in spherical geometry because there are no parallelograms!
%%%%%
%%%%%
\end{document}
