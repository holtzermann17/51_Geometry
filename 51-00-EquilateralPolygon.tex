\documentclass[12pt]{article}
\usepackage{pmmeta}
\pmcanonicalname{EquilateralPolygon}
\pmcreated{2013-03-22 17:12:41}
\pmmodified{2013-03-22 17:12:41}
\pmowner{Wkbj79}{1863}
\pmmodifier{Wkbj79}{1863}
\pmtitle{equilateral polygon}
\pmrecord{10}{39534}
\pmprivacy{1}
\pmauthor{Wkbj79}{1863}
\pmtype{Definition}
\pmcomment{trigger rebuild}
\pmclassification{msc}{51-00}
\pmsynonym{equilateral}{EquilateralPolygon}
\pmrelated{BasicPolygon}

\endmetadata

\usepackage{amssymb}
\usepackage{amsmath}
\usepackage{amsfonts}
\usepackage{pstricks}
\usepackage{psfrag}
\usepackage{graphicx}
\usepackage{amsthm}
%%\usepackage{xypic}

\begin{document}
A polygon is \emph{equilateral} if all of its sides are congruent.

Common examples of equilateral polygons are rhombi and regular polygons such as equilateral triangles and squares.

Let $T$ be a triangle in Euclidean geometry, hyperbolic geometry, or spherical geometry. Then the following are equivalent:

\begin{itemize}
\item $T$ is equilateral;
\item $T$ is equiangular;
\item $T$ is regular.
\end{itemize}

If $ T$ is allowed to be a polygon that has more than three sides, then the above statement is no longer true in any of the indicated geometries.

Below are some pictures of equilateral polygons that are not equiangular.

\begin{center}
\begin{pspicture}(0,0)(17,6)
\pspolygon(0,0)(3,2.25)(6.75,2.25)(3.75,0)
\pspolygon(7.5,0)(10.5,0)(10.5,3)(9,5.598)(7.5,3)
\pspolygon(11.268,1)(13,2)(15,2)(16.732,1)(15,0)(13,0)
\end{pspicture}
\end{center}
%%%%%
%%%%%
\end{document}
