\documentclass[12pt]{article}
\usepackage{pmmeta}
\pmcanonicalname{Polarity}
\pmcreated{2013-03-22 15:57:58}
\pmmodified{2013-03-22 15:57:58}
\pmowner{Algeboy}{12884}
\pmmodifier{Algeboy}{12884}
\pmtitle{polarity}
\pmrecord{12}{37981}
\pmprivacy{1}
\pmauthor{Algeboy}{12884}
\pmtype{Definition}
\pmcomment{trigger rebuild}
\pmclassification{msc}{51A10}
\pmclassification{msc}{51A05}
\pmsynonym{order reversing}{Polarity}
%\pmkeywords{projective geometry}
%\pmkeywords{projective point}
%\pmkeywords{hyperplane}
%\pmkeywords{order preserving}
%\pmkeywords{order reversing}
\pmrelated{SesquilinearFormsOverGeneralFields}
\pmrelated{PolaritiesAndForms}
\pmdefines{polarity}
\pmdefines{duality}
\pmdefines{correlation}
\pmdefines{pole}
\pmdefines{polar}

\endmetadata

\usepackage{latexsym}
\usepackage{amssymb}
\usepackage{amsmath}
\usepackage{amsfonts}
\usepackage{amsthm}

%%\usepackage{xypic}

%-----------------------------------------------------

%       Standard theoremlike environments.

%       Stolen directly from AMSLaTeX sample

%-----------------------------------------------------

%% \theoremstyle{plain} %% This is the default

\newtheorem{thm}{Theorem}

\newtheorem{coro}[thm]{Corollary}

\newtheorem{lem}[thm]{Lemma}

\newtheorem{lemma}[thm]{Lemma}

\newtheorem{prop}[thm]{Proposition}

\newtheorem{conjecture}[thm]{Conjecture}

\newtheorem{conj}[thm]{Conjecture}

\newtheorem{defn}[thm]{Definition}

\newtheorem{remark}[thm]{Remark}

\newtheorem{ex}[thm]{Example}



%\countstyle[equation]{thm}



%--------------------------------------------------

%       Item references.

%--------------------------------------------------


\newcommand{\exref}[1]{Example-\ref{#1}}

\newcommand{\thmref}[1]{Theorem-\ref{#1}}

\newcommand{\defref}[1]{Definition-\ref{#1}}

\newcommand{\eqnref}[1]{(\ref{#1})}

\newcommand{\secref}[1]{Section-\ref{#1}}

\newcommand{\lemref}[1]{Lemma-\ref{#1}}

\newcommand{\propref}[1]{Prop\-o\-si\-tion-\ref{#1}}

\newcommand{\corref}[1]{Cor\-ol\-lary-\ref{#1}}

\newcommand{\figref}[1]{Fig\-ure-\ref{#1}}

\newcommand{\conjref}[1]{Conjecture-\ref{#1}}


% Normal subgroup or equal.

\providecommand{\normaleq}{\unlhd}

% Normal subgroup.

\providecommand{\normal}{\lhd}

\providecommand{\rnormal}{\rhd}
% Divides, does not divide.

\providecommand{\divides}{\mid}

\providecommand{\ndivides}{\nmid}


\providecommand{\union}{\cup}

\providecommand{\bigunion}{\bigcup}

\providecommand{\intersect}{\cap}

\providecommand{\bigintersect}{\bigcap}










\begin{document}
\begin{defn}
\begin{itemize}
\item Given finite dimensional vector spaces $V$ and $W$, a \emph{duality} of the projective geometry $PG(V)$ to $PG(W)$ is an order-reversing bijection 
$f:PG(V)\rightarrow PG(W)$.  If $W=V$ then we can refer to $f$ as a correlation.

\item A correlation of order $2$ is called a \emph{polarity}.

\item The set of correlations and collineations $f:PG(V)\rightarrow PG(V)$ form a group denoted $P\Gamma L^*(V)$ with the operation of composition.
\end{itemize}
\end{defn}

\begin{remark}
Dualities are determined by where they map collinear triples.   Given a map
define on the points of $PG(V)$ to the hyperplanes of $PG(W)$ which maps collinear triples to triples of hyperplanes which intersect in a codimension 2 subspace, this specifies a unique duality.
\end{remark}

\begin{remark}
A polarity/duality necessarily interchanges points with hyperplanes.  In this context points are called ``poles'' and hyperplanes ``polars.''

An alternative definition of a duality is a projectivity (order-preserving map) $f:PG(V)\rightarrow PG(V^*)$.  
\end{remark}

Through the use of the fundamental theorem of projective geometry, dualities and polarities can be identified with non-degenerate sesquilinear forms.  (See \PMlinkname{Polarities and forms}{PolaritiesAndForms}.)

%%%%%
%%%%%
\end{document}
