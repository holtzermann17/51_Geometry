\documentclass[12pt]{article}
\usepackage{pmmeta}
\pmcanonicalname{IsoscelesTriangle}
\pmcreated{2013-03-22 11:44:12}
\pmmodified{2013-03-22 11:44:12}
\pmowner{drini}{3}
\pmmodifier{drini}{3}
\pmtitle{isosceles triangle}
\pmrecord{17}{30149}
\pmprivacy{1}
\pmauthor{drini}{3}
\pmtype{Definition}
\pmcomment{trigger rebuild}
\pmclassification{msc}{51-00}
\pmclassification{msc}{49J20}
\pmclassification{msc}{49J30}
\pmclassification{msc}{49-01}
\pmsynonym{isosceles}{IsoscelesTriangle}
\pmrelated{Triangle}
\pmrelated{RightTriangle}
\pmrelated{EquilateralTriangle}
\pmrelated{EquivalentConditionsForTriangles}
\pmrelated{EquiangularTriangle}
\pmrelated{RegularTriangle}
\pmdefines{base angle}
\pmdefines{vertex angle}

\endmetadata

\usepackage{amssymb}
\usepackage{amsmath}
\usepackage{amsfonts}
\usepackage{graphicx}
%%%%%%%\usepackage{xypic}
\begin{document}
An \emph{isosceles triangle} is a triangle with two congruent sides.  The angles opposite these two sides are the {\em base angles} and the angle between those sides is the {\em vertex angle} of the triangle.

This definition implies that any equilateral triangle is isosceles
too, but there are isosceles triangles that are not equilateral.

In any isosceles triangle, the angles opposite to the congruent sides
are also congruent.  In fact, this condition could be used to give an
alternate definition of isosceles, since a triangle is isosceles if
and only if it has two congruent angles.

In an equilateral triangle, the \PMlinkname{height}{BaseAndHeightOfTriangle}, the median and the bisector to
the third side are the same line.

\includegraphics{trianglebyside}

\PMlinkescapeword{opposite}

%%%%%
%%%%%
%%%%%
%%%%%
%%%%%
%%%%%
%%%%%
\end{document}
