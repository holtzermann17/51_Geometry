\documentclass[12pt]{article}
\usepackage{pmmeta}
\pmcanonicalname{ParallelogramLaw}
\pmcreated{2013-03-22 12:02:23}
\pmmodified{2013-03-22 12:02:23}
\pmowner{drini}{3}
\pmmodifier{drini}{3}
\pmtitle{parallelogram law}
\pmrecord{10}{31082}
\pmprivacy{1}
\pmauthor{drini}{3}
\pmtype{Theorem}
\pmcomment{trigger rebuild}
\pmclassification{msc}{51-00}
\pmrelated{Parallelogram}
\pmrelated{Quadrilateral}
\pmrelated{Rectangle}
\pmrelated{Square}
\pmrelated{Rhombus}
\pmrelated{ApolloniusTheorem}
\pmrelated{Median}
\pmrelated{ParallelogramLaw2}

\usepackage{pstricks}
\begin{document}
Let $ABCD$ be a parallelogram with side lengths $u,v$ and whose diagonals have lengths $d_1$ and $d_2$ then
$$2u^2+2v^2=d_1^2 + d_2^2.$$


\begin{center}
\framebox{
\begin{pspicture*}(-1,-1)(6,3)
\pspolygon(0,0)(4,0)(5,2)(1,2)
\uput[270](2,-0.3){$u$}
\uput[180](0.3,1){$v$}
\qline(0,0)(5,2)
\qline(4,0)(1,2)
\uput[0](1.7,0.5){$d_1$}
\uput[0](1.8,1.5){$d_2$}
\end{pspicture*}
}
\end{center}
%%%%%
%%%%%
%%%%%
\end{document}
