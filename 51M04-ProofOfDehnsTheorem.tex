\documentclass[12pt]{article}
\usepackage{pmmeta}
\pmcanonicalname{ProofOfDehnsTheorem}
\pmcreated{2013-03-22 16:18:07}
\pmmodified{2013-03-22 16:18:07}
\pmowner{paolini}{1187}
\pmmodifier{paolini}{1187}
\pmtitle{proof of Dehn's theorem}
\pmrecord{7}{38423}
\pmprivacy{1}
\pmauthor{paolini}{1187}
\pmtype{Proof}
\pmcomment{trigger rebuild}
\pmclassification{msc}{51M04}
\pmdefines{Dehn invariant}
\pmdefines{Dehn's invariant}

% this is the default PlanetMath preamble.  as your knowledge
% of TeX increases, you will probably want to edit this, but
% it should be fine as is for beginners.

% almost certainly you want these
\usepackage{amssymb}
\usepackage{amsmath}
\usepackage{amsfonts}

% used for TeXing text within eps files
%\usepackage{psfrag}
% need this for including graphics (\includegraphics)
%\usepackage{graphicx}
% for neatly defining theorems and propositions
\usepackage{amsthm}
% making logically defined graphics
%%%\usepackage{xypic}

% there are many more packages, add them here as you need them

% define commands here
\newcommand{\R}{\mathbb R}
\newtheorem{theorem}{Theorem}
\newtheorem{definition}{Definition}
\theoremstyle{remark}
\newtheorem{example}{Example}
\begin{document}
We define the Dehn's invariant, which is a number given to any polyhedron which does not change under scissor-equivalence.

Choose an additive function $f\colon \R\to \R$ 
such that $f(\pi)=f(0)=0$
and define for any polyhedron $P$ 
the number (Dehn's invariant)
\[
  D(P) = \sum_{e \in \{\text{edges of $P$}\}} f(\theta_e) \ell(e)
\]
where $\theta_e$ is the angle between the two faces of $P$ joining in $e$, and $\ell(e)$ is the length of the edge $e$.

We want to prove that if we decompose $P$ into smaller polyhedra $P_1,\ldots,P_N$ as in the definition of scissor-equivalence, we have 
\begin{equation}\label{eq:sum}
  D(P) = \sum_{k=1}^N D(P_k)
\end{equation}
which means that if $P$ is scissor equivalent to $Q$ then $D(P)=D(Q)$.

Let $P_1,\ldots,P_N$ be such a decomposition of $P$. Given any edge $e$ of a piece $P_k$ the following cases arise:
\begin{enumerate}
\item $e$ is contained in the interior of $P$. Since an entire neighbourhood of $e$ is contained in $P$ the angles of the pieces which have $e$ as an edge (or part of an edge) must have sum $2\pi$. So in the right hand side of~\eqref{eq:sum} the edge $e$ gives a contribution of $f(2\pi)\ell(e)$ (recall
that $f$ is additive).

\item
$e$ is contained in a facet of $P$. The same argument as before is valid, only we find that the total contribution is $f(\pi)\ell(e)$.

\item
$e$ is contained in an edge $e'$ of $P$. In this case the total contribution given by $e$ to the right hand side of~\eqref{eq:sum} is given by $f(\theta_{e'})\ell(e)$.
\end{enumerate}

Since we have choosen $f$ so that $f(\pi)=0$ and hence also $f(2\pi)=0$ (since $f$ is additive) we conclude that the equivalence~\eqref{eq:sum} is valid.

Now we are able to prove Dehn's Theorem. Choose $T$ to be a regular tetrahedron with edges of length $1$. Then $D(T)=6 f(\theta)$ where $\theta$ is the angle between two faces of $T$. We know that $\theta/\pi$ is irrational, hence there exists an additive function $f$ such that $f(\theta)=1$ while $f(\pi/2)=0$ (as there exist additive functions which are not linear).

So if $P$ is any parallelepiped we find that $D(P)=0$ (since each angle between facets of $P$ is $\pi/2$ and $f(\pi/2)=0$) while $D(T)=6$. This means that $P$ and $T$ cannot be scissor-equivalent.
%%%%%
%%%%%
\end{document}
