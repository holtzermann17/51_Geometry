\documentclass[12pt]{article}
\usepackage{pmmeta}
\pmcanonicalname{PivotTheorem}
\pmcreated{2013-03-22 12:10:13}
\pmmodified{2013-03-22 12:10:13}
\pmowner{drini}{3}
\pmmodifier{drini}{3}
\pmtitle{pivot theorem}
\pmrecord{8}{31383}
\pmprivacy{1}
\pmauthor{drini}{3}
\pmtype{Theorem}
\pmcomment{trigger rebuild}
\pmclassification{msc}{51M04}

\usepackage{graphicx}
%%%%\usepackage{xypic} 
\usepackage{bbm}
\newcommand{\Z}{\mathbbmss{Z}}
\newcommand{\C}{\mathbbmss{C}}
\newcommand{\R}{\mathbbmss{R}}
\newcommand{\Q}{\mathbbmss{Q}}
\newcommand{\mathbb}[1]{\mathbbmss{#1}}
\newcommand{\figura}[1]{\begin{center}\includegraphics{#1}\end{center}}
\newcommand{\figuraex}[2]{\begin{center}\includegraphics[#2]{#1}\end{center}}
\usepackage{psfrag}
\begin{document}
\psfrag{A}{$A$}
\psfrag{B}{$B$}
\psfrag{C}{$C$}
\psfrag{D}{$D$}
\psfrag{E}{$E$}
\psfrag{F}{$F$}
If $ABC$ is a triangle, $D,E,F$ points on the sides $BC, CA, AB$ respectively, then the circumcircles of the triangles $AEF, BFD$ and  $CDE$ have a common point.
\figura{pivot}
%%%%%
%%%%%
%%%%%
\end{document}
