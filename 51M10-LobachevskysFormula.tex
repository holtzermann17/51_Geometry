\documentclass[12pt]{article}
\usepackage{pmmeta}
\pmcanonicalname{LobachevskysFormula}
\pmcreated{2013-03-22 14:05:53}
\pmmodified{2013-03-22 14:05:53}
\pmowner{vmoraru}{1243}
\pmmodifier{vmoraru}{1243}
\pmtitle{Lobachevsky's formula}
\pmrecord{6}{35484}
\pmprivacy{1}
\pmauthor{vmoraru}{1243}
\pmtype{Definition}
\pmcomment{trigger rebuild}
\pmclassification{msc}{51M10}
\pmdefines{angle of parallelism}

\usepackage{amssymb}
\usepackage{amsmath}
\usepackage{amsfonts}
\usepackage{graphicx}
\usepackage{amsthm}
%%\usepackage{xypic}
\begin{document}
Let $AB$ be a line. Let $M,T$ be two points so that $M$ not lies on $AB$,
$T$ lies on $AB$, and $MT$ perpendicular to $AB$. Let $MD$ be any other line who meets
$AT$ in $D$.In a hyperbolic geometry, as $D$ moves off to infinity
along $AT$ the line $MD$ meets the line $MS$ which is said to be
parallel to $AT$. The angle $\widehat{SMT}$ is called the
\emph{angle of parallelism} for perpendicular distance $d$, and is
given by $$P(d)=2\tan^{-1}(e^{-d}),$$ which is called
\emph{Lobachevsky's formula.}

\begin{center}
\includegraphics{lob}
\end{center}
%%%%%
%%%%%
\end{document}
