\documentclass[12pt]{article}
\usepackage{pmmeta}
\pmcanonicalname{DihedralAngle}
\pmcreated{2013-03-22 18:48:07}
\pmmodified{2013-03-22 18:48:07}
\pmowner{pahio}{2872}
\pmmodifier{pahio}{2872}
\pmtitle{dihedral angle}
\pmrecord{9}{41602}
\pmprivacy{1}
\pmauthor{pahio}{2872}
\pmtype{Definition}
\pmcomment{trigger rebuild}
\pmclassification{msc}{51M04}
\pmrelated{NormalSection}
\pmrelated{PerpendicularityInEuclideanPlane}
\pmrelated{TrihedralAngle}
\pmrelated{SolidAngle}
\pmdefines{edge}
\pmdefines{side}
\pmdefines{normal section}
\pmdefines{concave}
\pmdefines{perpendicular}
\pmdefines{perpendicular planes}

\endmetadata

% this is the default PlanetMath preamble.  as your knowledge
% of TeX increases, you will probably want to edit this, but
% it should be fine as is for beginners.

% almost certainly you want these
\usepackage{amssymb}
\usepackage{amsmath}
\usepackage{amsfonts}

% used for TeXing text within eps files
%\usepackage{psfrag}
% need this for including graphics (\includegraphics)
%\usepackage{graphicx}
% for neatly defining theorems and propositions
 \usepackage{amsthm}
% making logically defined graphics
%%%\usepackage{xypic}
\usepackage{pstricks}
\usepackage{pst-plot}

% there are many more packages, add them here as you need them

% define commands here

\theoremstyle{definition}
\newtheorem*{thmplain}{Theorem}

\begin{document}
Two distinct half-planes, emanating from a same line $l$, \PMlinkescapetext{divide} the space ($\mathbb{R}^3$) into two regions called {\em dihedral angles}.\, The line $l$ is the {\em edge} of the dihedral angle and the bounding half-planes are its {\em sides}.

\begin{center}
\begin{pspicture}(-3,-1)(3,3)
\psline[linewidth=0.05](-2.5,-1)(1.5,-1)(2.5,0)(2,2)(-2,2)
\psline(-2,2)(-1.5,0)(-2.5,-1)
\psline(-2.1,0)(3,0)
\rput(0,0.2){$l$}
\rput(-3,-1){.}
\rput(3,3){.}
\end{pspicture}
\end{center}


The angle, which the sides of a dihedral planes separate from a normal plane of the edge, is the {\em normal section} of the dihedral angle.\, Apparently, all normal sections are equal.\, According to the \PMlinkescapetext{size} of the normal section, the dihedral angle may be called acute, right, obtuse, straight, \PMlinkname{skew}{ConvexAngle}, convex and concave.\, Unlike the angle between two planes, a dihedral angle may be over 90 \PMlinkescapetext{degrees}.

If two planes intersect each other and if one of the four dihedral angles formed is right, then also the others are right.\, Then we say that the planes are {\em perpendicular} to each other.



%%%%%
%%%%%
\end{document}
