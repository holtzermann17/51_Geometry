\documentclass[12pt]{article}
\usepackage{pmmeta}
\pmcanonicalname{ConeInmathbbR3}
\pmcreated{2013-03-22 15:29:54}
\pmmodified{2013-03-22 15:29:54}
\pmowner{pahio}{2872}
\pmmodifier{pahio}{2872}
\pmtitle{cone in $\mathbb{R}^3$}
\pmrecord{22}{37357}
\pmprivacy{1}
\pmauthor{pahio}{2872}
\pmtype{Definition}
\pmcomment{trigger rebuild}
\pmclassification{msc}{51M20}
\pmclassification{msc}{51M04}
\pmsynonym{generalized cone}{ConeInmathbbR3}
\pmsynonym{circular double cone}{ConeInmathbbR3}
\pmsynonym{right circular double cone}{ConeInmathbbR3}
\pmrelated{SurfaceOfRevolution2}
\pmdefines{apex}
\pmdefines{base}
\pmdefines{circular cone}
\pmdefines{cone}
\pmdefines{conical surface}
\pmdefines{mantle}
\pmdefines{pyramid}
\pmdefines{regular pyramid}
\pmdefines{right circular cone}
\pmdefines{solid cone}

\endmetadata

% this is the default PlanetMath preamble.  as your knowledge
% of TeX increases, you will probably want to edit this, but
% it should be fine as is for beginners.

% almost certainly you want these
\usepackage{amssymb}
\usepackage{amsmath}
\usepackage{amsfonts}

% used for TeXing text within eps files
%\usepackage{psfrag}
% need this for including graphics (\includegraphics)
\usepackage{graphicx}
% for neatly defining theorems and propositions
 \usepackage{amsthm}
% making logically defined graphics
%%%\usepackage{xypic}

% there are many more packages, add them here as you need them

% define commands here

\theoremstyle{definition}
\newtheorem*{thmplain}{Theorem}
\begin{document}
When a straight line moves in $\mathbb{R}^3$ passing constantly through a certain point $O$, the ruled surface it sweeps is called a {\em conical surface} or a {\em generalized cone}.\, Formally and more generally, a conical surface $S$ is a ruled surface with the given condition: 
\begin{quote}
there is a point $O$ on $S$, such that any ruling $\ell$ on $S$ passes through $O$.
\end{quote}

By definition, it is readily seen that this point $O$ is unique, for otherwise, all rulings of $S$ that pass through $O$ as well as another point $O^{\prime}$ must all coincide, concluding that $S$ must be nothing more than a straight line, contradicting the fact that $S$ is a surface.\, The point $O$ is commonly known as the \emph{apex} of the conical surface $S$.\\

\begin{figure}
\begin{center}
\includegraphics{cone.eps}
\end{center}
\caption{A circular conical surface}
\end{figure}

\textbf{Remarks.}
\begin{itemize}
\item No plane can be a conical surface, because one can always find a line (ruling) on the plane not passing through $O$.
\item A conical surface is, strictly speaking, not a (regular) surface in the sense of a differentiable manifold.\, This is because of the differentiability at the apex breaks down.\, In fact, no neighborhood of the apex is diffeomorphic to $\mathbb{R}^2$.\, Nevertheless, it is easy to see that any two points on a conical surface can be joined by a simple continuous curve.
\item Any plane passing through the apex $O$ of a conical surface $S$ must contain at least two lines $\ell,m$ through the apex such that $\ell\cap S=m\cap S=\lbrace O\rbrace$.\, In fact, it can be shown that there is a plane with the above property such that $\ell\perp m$.
\item Given a conical surface $S$, if there is a plane $\pi$ in $\mathbb{R}^3$ such that $\pi\cap S=\varnothing$, then it can be shown that $S$ is planar (that is, $S$ lies on a plane).\, This shows that if $S$ is a non-planar surface, it must have non-empty intersection with \emph{any} plane in $\mathbb{R}^3$!
\end{itemize}

Given a plane $\pi$ not passing through the apex $O$ of a non-planar conical surface $S$, the intersection of $\pi$ and $S$ is non-empty, as guaranteed by the previous remark.  Let $c$ be the intersection of $\pi$ and $S$.  It is not hard to see that $c$ is necessarily a curve.  The curve may be bounded or unbounded, and it may have disjoint components.  If there is a plane not passing
through the apex of a conical surface $S$, such that its intersection with $S$ is a bounded, connected closed loop, then we call this surface $S$ a \emph{closed cone}, or \emph{cone} for short.  Intuitively, it can be pictured as a surface swept out by a moving straight line that returns to its starting position.  Any plane not passing through the apex of a closed cone $S$ intersects each ruling of $S$ at exactly one point.

The solid bounded by a closed cone $S$, a plane $\pi$ not passing through the apex $O$ and the apex $O$ itself is called a \emph{solid cone}.\, The portion of the surface of the cone belonging to the conical surface is called the \PMlinkescapetext{{\em lateral surface}} or the \PMlinkescapetext{{\em mantle}} of the solid cone and the portion belonging to the plane is the {\em base} of the solid cone.\,

The intersections of a solid cone and the planes parallel to the
base plane are similar.\, The perpendicular
\PMlinkescapetext{distance} of the apex and the base plane is the
\PMlinkescapetext{{\em height}} of the cone.\, The volume ($V$) of
the cone equals to the third of product of the base area ($A$) and
the \PMlinkescapetext{height} ($h$):
                              $$V = \frac{Ah}{3}.$$
The formula can be derived directly by observing that, if we were to take another parallel slice of the solid cone, the area $A(x)$ of the base from the slice is directly proportional to the square of the corresponding height $x$.  Integrating $A(x)$ with respect to $x$, where $x$ is between $0$ and $h$ gives us the above formula.

If the base of a cone is a polygon, the cone is called a {\em pyramid} (which is a polyhedron).\, The mantle of a pyramid consists of triangular {\em faces} having as common vertex the apex $O$.\,  A pyramid is \PMlinkescapetext{{\em regular}} if its base is a regular polygon and the \PMlinkescapetext{height line connects the apex and the centre} of its base.\,  A tetrahedron is an example of a regular pyramid.\, Note that, in a regular pyramid, all of the faces are isosceles.\, Colloquially, ``pyramid'' typically refers to a regular square pyramid.

If the base is a circle, the cone is called {\em circular}.\, If its \PMlinkescapetext{height line connects the apex and the centre} of the base circle, the circular cone is \PMlinkescapetext{{\em right}}.\, Colloquially, ``cone'' typically refers to a right circular cone. 

In any cone, the line segment of a ruling between the base plane and the apex is a \PMlinkescapetext{{\em side line}} of the cone.\, All \PMlinkescapetext{side lines} are equally long only in a right circular cone.\, If in this case, the \PMlinkescapetext{length} of the side line is $s$ and the radius of the base circle $r$, then the area of the mantle of the right circular cone equals $\pi{rs}$.
%%%%%
%%%%%
\end{document}
