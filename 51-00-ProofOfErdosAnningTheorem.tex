\documentclass[12pt]{article}
\usepackage{pmmeta}
\pmcanonicalname{ProofOfErdosAnningTheorem}
\pmcreated{2013-03-22 13:19:11}
\pmmodified{2013-03-22 13:19:11}
\pmowner{lieven}{1075}
\pmmodifier{lieven}{1075}
\pmtitle{proof of Erd\"os-Anning Theorem}
\pmrecord{4}{33828}
\pmprivacy{1}
\pmauthor{lieven}{1075}
\pmtype{Proof}
\pmcomment{trigger rebuild}
\pmclassification{msc}{51-00}

\endmetadata

% this is the default PlanetMath preamble.  as your knowledge
% of TeX increases, you will probably want to edit this, but
% it should be fine as is for beginners.

% almost certainly you want these
\usepackage{amssymb}
\usepackage{amsmath}
\usepackage{amsfonts}

% used for TeXing text within eps files
%\usepackage{psfrag}
% need this for including graphics (\includegraphics)
%\usepackage{graphicx}
% for neatly defining theorems and propositions
%\usepackage{amsthm}
% making logically defined graphics
%%%\usepackage{xypic}

% there are many more packages, add them here as you need them

% define commands here
\begin{document}
Let $A,B$ and $C$ be three non-collinear points. For an additional point $P$ consider the triangle $ABP$. By using the triangle inequality for the sides $PB$ and $PA$ we find $-|AB|\leq |PB|-|PA|\leq |AB|$. Likewise, for triangle $BCP$ we get $-|BC|\leq |PB|-|PC|\leq |BC|$. Geometrically, this means the point $P$ lies on two hyperbola with $A$ and $B$ or $B$ and $C$ respectively as foci. Since all the lengths involved here are by assumption integer there are only $2|AB|+1$ possibilities for $|PB|-|PA|$ and $2|BC|+1$ possibilities for $|PB|-|PC|$. These hyperbola are distinct since they don't have the same major axis. So for each pair of hyperbola we can have at most $4$ points of intersection and there can be no more than $4(2|AB|+1)(2|BC|+1)$ points satisfying the conditions.
%%%%%
%%%%%
\end{document}
