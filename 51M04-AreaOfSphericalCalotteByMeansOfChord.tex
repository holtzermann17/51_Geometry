\documentclass[12pt]{article}
\usepackage{pmmeta}
\pmcanonicalname{AreaOfSphericalCalotteByMeansOfChord}
\pmcreated{2013-03-22 18:19:20}
\pmmodified{2013-03-22 18:19:20}
\pmowner{pahio}{2872}
\pmmodifier{pahio}{2872}
\pmtitle{area of spherical calotte by means of chord}
\pmrecord{5}{40952}
\pmprivacy{1}
\pmauthor{pahio}{2872}
\pmtype{Derivation}
\pmcomment{trigger rebuild}
\pmclassification{msc}{51M04}
\pmsynonym{alternative way to find area of spherical calotte}{AreaOfSphericalCalotteByMeansOfChord}
%\pmkeywords{spherical calotte}
\pmrelated{ThalesTheorem}
\pmrelated{SimilarityOfTriangles}

\endmetadata

% this is the default PlanetMath preamble.  as your knowledge
% of TeX increases, you will probably want to edit this, but
% it should be fine as is for beginners.

% almost certainly you want these
\usepackage{amssymb}
\usepackage{amsmath}
\usepackage{amsfonts}

% used for TeXing text within eps files
%\usepackage{psfrag}
% need this for including graphics (\includegraphics)
%\usepackage{graphicx}
% for neatly defining theorems and propositions
 \usepackage{amsthm}
% making logically defined graphics
%%%\usepackage{xypic}

% there are many more packages, add them here as you need them

% define commands here

\theoremstyle{definition}
\newtheorem*{thmplain}{Theorem}

\begin{document}
\PMlinkescapeword{height}
Let the arc $PR$ of a circle with radius $r$ rotate about the diameter $PQ$.\, The surface of revolution is a spherical calotte with the height $h$.\, If the \PMlinkescapetext{length} of the chord $PR$ is $k$, we obtain from the right triangle $PQR$ the proportion equation
$$\frac{h}{k} = \frac{k}{2r},$$
i.e. the chord $k$ is the central proportional of the height and the diameter.\, Accordingly, we can substitute\, $2rh = k^2$\, to the expression
$$A = 2\pi rh$$
of the area of the spherical calotte derived in the \PMlinkname{parent entry}{AreaOfSphericalZone}. Thus we have an alternative \PMlinkescapetext{formula} 
\begin{align}
A = \pi{k}^2
\end{align}
for finding the area of a spherical calotte.

\begin{thebibliography}{9}
\bibitem{K.V.} {\sc K. V\"ais\"al\"a}: {\em Geometria}.\, Kymmenennen painoksen muuttamaton lis\"apainos.\, Werner S\"oderstr\"om Osakeyhti\"o, Porvoo \& Helsinki (1971).

\end{thebibliography}
%%%%%
%%%%%
\end{document}
