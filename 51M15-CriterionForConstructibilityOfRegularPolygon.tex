\documentclass[12pt]{article}
\usepackage{pmmeta}
\pmcanonicalname{CriterionForConstructibilityOfRegularPolygon}
\pmcreated{2013-03-22 17:18:40}
\pmmodified{2013-03-22 17:18:40}
\pmowner{Wkbj79}{1863}
\pmmodifier{Wkbj79}{1863}
\pmtitle{criterion for constructibility of regular polygon}
\pmrecord{6}{39660}
\pmprivacy{1}
\pmauthor{Wkbj79}{1863}
\pmtype{Theorem}
\pmcomment{trigger rebuild}
\pmclassification{msc}{51M15}
\pmclassification{msc}{12D15}
\pmrelated{RegularPolygon}
\pmrelated{RootOfUnity}
\pmrelated{TheoremOnConstructibleAngles}

\usepackage{amssymb}
\usepackage{amsmath}
\usepackage{amsfonts}
\usepackage{pstricks}
\usepackage{psfrag}
\usepackage{graphicx}
\usepackage{amsthm}
%%\usepackage{xypic}
\newtheorem{thm*}{Theorem}

\begin{document}
\PMlinkescapeword{adjacent}
\PMlinkescapeword{constructible}
\PMlinkescapeword{regular}

\begin{thm*}
Let $n$ be an integer with $n \ge 3$.  Then a \PMlinkname{regular $n$-gon}{RegularPolygon} is \PMlinkname{constructible}{Constructible2} if and only if a \PMlinkname{primitive $n$th root of unity}{PrimitiveRootOfUnity} is a constructible number.
\end{thm*}

\begin{proof}
First of all, note that a \PMlinkescapetext{primitive $n$th root of unity} is a constructible number if and only if $\displaystyle \cos\left(\frac{2\pi}{n}\right)+i\sin\left(\frac{2\pi}{n}\right)$ is a constructible number.  See the entry on roots of unity for more details.  Therefore, without loss of generality, only the constructibility of the number $\displaystyle \cos\left(\frac{2\pi}{n}\right)+i\sin\left(\frac{2\pi}{n}\right)$ will be considered.

Sufficiency:  If a regular $n$-gon is constructible, then so is the angle whose \PMlinkname{vertex}{Vertex5} is the \PMlinkname{center}{Center9} of the polygon and whose rays pass through adjacent vertices of the polygon.  The \PMlinkname{measure}{AngleMeasure} of this angle is $\displaystyle \frac{2\pi}{n}$.

By the theorem on constructible angles, $\displaystyle \sin\left(\frac{2\pi}{n}\right)$ and $\displaystyle \cos\left(\frac{2\pi}{n}\right)$ are constructible numbers.  Note that $i$ is also a constructible number.  Thus, $\displaystyle \cos\left(\frac{2\pi}{n}\right)+i\sin\left(\frac{2\pi}{n}\right)$ is a constructible number.

Necessity:  If $\displaystyle \omega=\cos\left(\frac{2\pi}{n}\right)+i\sin\left(\frac{2\pi}{n}\right)$ is a constructible number, then so is $\omega^m$ for any integer $m$.

On the complex plane, for every integer $m$ with $0\le m<n$, construct the point corresponding to $\omega^m$.  Use line segments to connect the points corresponding to $\omega^m$ and $\omega^{m+1}$ for every integer $m$ with $0\le m<n$.  (Note that $\omega^0=1=\omega^n$.)  This forms a regular $n$-gon.
\end{proof}
%%%%%
%%%%%
\end{document}
